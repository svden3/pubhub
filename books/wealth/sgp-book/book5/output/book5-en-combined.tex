% Book 5: Retirement Planning - English Version
% Requires XeLaTeX

\documentclass[11pt,openright,twoside]{book}

% Font setup with fontspec (XeLaTeX)
\usepackage{fontspec}

% Use professional fonts
\setmainfont{Georgia}[
  BoldFont=Georgia Bold,
  ItalicFont=Georgia Italic
]
\setsansfont{Helvetica Neue}
\setmonofont{Menlo}[Scale=0.9]

% Page geometry - book format (6x9 inches)
\usepackage{geometry}
\geometry{
  paperwidth=6in,
  paperheight=9in,
  top=0.75in,
  bottom=0.75in,
  inner=0.875in,
  outer=0.625in
}

% Line spacing
\linespread{1.15}

% Colors
\usepackage{xcolor}
\definecolor{chaptercolor}{RGB}{139,0,0}

% Hyperlinks
\usepackage{hyperref}
\hypersetup{
  colorlinks=true,
  linkcolor=chaptercolor,
  urlcolor=blue,
  bookmarks=true,
  bookmarksnumbered=true
}

% Graphics
\usepackage{graphicx}

% Tables
\usepackage{longtable}
\usepackage{booktabs}
\usepackage{array}

% Math symbols
\usepackage{amssymb}

% Pandoc 3.x table support
\usepackage{etoolbox}
\makeatletter
\patchcmd\longtable{\par}{\if@noskipsec\mbox{}\fi\par}{}{}
\@ifundefined{c@none}{\newcounter{none}}{}
\makeatother

% Widow and orphan control
\widowpenalty=10000
\clubpenalty=10000

% Metadata
\title{Retirement Planning}
\author{Jim Xiao}
\date{December 2025}

% Pandoc tight list fix
\providecommand{\tightlist}{%
  \setlength{\itemsep}{0pt}\setlength{\parskip}{0pt}}

\begin{document}

% Front matter
\frontmatter

% Title page
\begin{titlepage}
\thispagestyle{empty}
\vspace*{2cm}
\begin{center}
{\Huge\bfseries Retirement Planning}\\[0.5cm]
{\Large\itshape Silicon Valley Forrest's Investment Memo · Book
Five}\\[2cm]
\vspace{1cm}
{\large Silicon Valley Forrest's Investment Memo Series}\\[1cm]
{\large Book Five}\\[2cm]
{\Large Jim Xiao}\\[1cm]
\vfill
{\large December 2025}
\end{center}
\end{titlepage}

% Copyright page
\thispagestyle{empty}
\vspace*{\fill}
\begin{flushleft}
\small
Copyright \copyright\ \the\year\ Jim Xiao\\[0.5cm]
All rights reserved.\\[1cm]
This book is for educational and informational purposes only\\
and does not constitute investment advice.\\
Investing involves risk. Please consult a professional.
\end{flushleft}
\cleardoublepage

% Table of contents
\tableofcontents
\cleardoublepage

% Main matter
\mainmatter

\section{25. Retirement Is Not the End, It's a Turning
Point}\label{retirement-is-not-the-end-its-a-turning-point}

\begin{quote}
\textbf{Mama said: ``Son, retirement isn't stopping work. It's being
able to choose your work.''}
\end{quote}

\begin{center}\rule{0.5\linewidth}{0.5pt}\end{center}

\subsection{Sarah's Math Lesson}\label{sarahs-math-lesson}

In 2020, Sarah was 48.

``I'm going to retire.''

``Are you crazy? You're only 48.''

``I've done the math.'' She pulled out a piece of paper:

\begin{verbatim}
My calculations:

Google stock: enough for 15 years
Index funds: enough for 10 years
House: paid off
Healthcare: ACA covered
Annual expenses: $80,000

Conclusion: I can live to 95.
\end{verbatim}

``But\ldots{} aren't you afraid of having nothing to do?''

``\textbf{I'm going to paint. I've waited 16 years for this day.}''

\begin{center}\rule{0.5\linewidth}{0.5pt}\end{center}

\subsection{The Three Numbers of
Retirement}\label{the-three-numbers-of-retirement}

\subsubsection{Number One: How Much Do You
Need?}\label{number-one-how-much-do-you-need}

\textbf{Simple Formula:}

\begin{verbatim}
Retirement savings = Annual expenses × 25
\end{verbatim}

This is called the \textbf{4\% Rule} --- withdraw 4\% from your
retirement savings each year, and theoretically it will last 30+ years.

\textbf{Examples:} - \$60,000/year expenses → Need \$1,500,000 -
\$80,000/year expenses → Need \$2,000,000 - \$100,000/year expenses →
Need \$2,500,000

Mama said: \textgreater{} ``Son, this is just theory. In reality, you
won't spend the same amount every year. Expenses at 60 and 80 are
completely different.''

\begin{center}\rule{0.5\linewidth}{0.5pt}\end{center}

\subsubsection{Number Two: How Much Time Do You
Have?}\label{number-two-how-much-time-do-you-have}

\textbf{The Magic of Compound Interest:}

\begin{verbatim}
Starting at 25 vs Starting at 35

Assuming $500/month, 7% annual return:

Start at 25 → $1,200,000 at 65
Start at 35 → $567,000 at 65
\end{verbatim}

\textbf{10 years difference = \$630,000}

Sarah said: \textgreater{} ``Compound interest is a function of time.
The earlier you start, the further you go.''

\begin{center}\rule{0.5\linewidth}{0.5pt}\end{center}

\subsubsection{Number Three: How Many Income Sources Do You
Have?}\label{number-three-how-many-income-sources-do-you-have}

\textbf{Ideal Retirement Income Sources:}

\begin{verbatim}
1. Social Security      — Basic safety net
2. Retirement accounts (401k/IRA) — Tax advantages
3. Investment income (dividends/interest) — Passive income
4. Part-time/consulting — If you want
\end{verbatim}

\textbf{Mike's Lesson:} \textgreater{} ``All my money was in Enron
stock. Retirement fund, 401k, personal accounts. \textgreater{} I
thought I was a millionaire. \textgreater{} Three months later, I had
nothing.''

\textbf{Rule:} Have at least 2-3 independent retirement income sources.

\begin{center}\rule{0.5\linewidth}{0.5pt}\end{center}

\subsection{Retirement Accounts: Silicon Valley's Tax
Weapons}\label{retirement-accounts-silicon-valleys-tax-weapons}

\subsubsection{401(k): Your Company Saves For
You}\label{k-your-company-saves-for-you}

\begin{verbatim}
2024 Limits: $23,000 (age 50+: +$7,500)

Pros:
✓ Pre-tax contributions, save on taxes now
✓ Company match (free money!)
✓ Automatic deduction, no discipline needed

Cons:
✗ 10% penalty if withdrawn before 59.5
✗ Taxed as income when withdrawn
✗ Limited investment choices
\end{verbatim}

\textbf{Mama's Orders:} \textgreater{} ``Son, contribute at least up to
the company match. Not doing so is giving free money to the
government.''

\begin{center}\rule{0.5\linewidth}{0.5pt}\end{center}

\subsubsection{Roth IRA: The Best Choice for Young
People}\label{roth-ira-the-best-choice-for-young-people}

\begin{verbatim}
2024 Limits: $7,000 (age 50+: +$1,000)

Pros:
✓ After-tax contributions, tax-free growth
✓ Completely tax-free withdrawals
✓ High flexibility in retirement
✓ No required minimum distributions

Cons:
✗ Income limits ($161k single / $240k married)
✗ Lower contribution limits
\end{verbatim}

\textbf{Sarah's Advice:} \textgreater{} ``When young, income is low, tax
rate is low. Use Roth IRA. \textgreater{} When old, income is high, tax
rate is high. Use Traditional 401k. \textgreater{} This is called tax
arbitrage.''

\begin{center}\rule{0.5\linewidth}{0.5pt}\end{center}

\subsubsection{Backdoor Roth: The Back Door for High
Earners}\label{backdoor-roth-the-back-door-for-high-earners}

If your income exceeds the Roth IRA limit:

\begin{verbatim}
Steps:
1. Contribute to Traditional IRA (non-deductible)
2. Immediately convert to Roth IRA
3. Pay minimal conversion tax
4. Enjoy lifetime tax-free growth
\end{verbatim}

\textbf{Note:} Consult a tax professional about the Pro Rata rule.

\begin{center}\rule{0.5\linewidth}{0.5pt}\end{center}

\subsubsection{Peter Thiel's Roth IRA
Legend}\label{peter-thiels-roth-ira-legend}

In 2021, ProPublica broke a stunning secret:

\textbf{Peter Thiel's Roth IRA account held \$5 billion.}

``Wait,'' Mike said, ``the Roth IRA limit is only a few thousand a year.
How can it become \$5 billion?''

\textbf{Answer: He bought his own company's stock.}

\begin{verbatim}
Peter Thiel's Move (1999):

1. Open a Roth IRA account
2. Buy PayPal founder shares with the account
3. Price: $0.001 per share (yes, 0.1 cents)
4. Quantity: 1.7 million shares
5. Total cost: $1,700 (within annual limits)

Then...

2002: eBay acquires PayPal for $1.5 billion
Peter Thiel's shares: worth tens of millions

Continued...

Used Roth IRA to invest in Facebook (2004)
Then Palantir
Then other startups

2021: Account value $5 billion
All tax-free.
\end{verbatim}

\begin{center}\rule{0.5\linewidth}{0.5pt}\end{center}

\textbf{Sarah's Analysis:}

\begin{quote}
``Is it legal? Technically, yes.

Can ordinary people replicate it? No.

The key is he could buy his own company's stock at \$0.001. You and I
would have to pay market price for PayPal.

This isn't an investing technique. It's founder privilege.''
\end{quote}

\begin{center}\rule{0.5\linewidth}{0.5pt}\end{center}

\textbf{IRS Response:}

\begin{verbatim}
After 2021, Congress began investigating:

Proposed limits (not yet passed):
- Force withdrawal of Roth IRA amounts over $10M
- Ban purchasing non-publicly traded stock
- Ban purchasing your own company's stock

Current status (2024):
- These limits haven't passed
- But IRS is scrutinizing "undervalued" transactions
- Founder share valuations must be reasonable
\end{verbatim}

\begin{center}\rule{0.5\linewidth}{0.5pt}\end{center}

\textbf{Mama's Comment:}

\begin{quote}
``Son, Peter Thiel is a genius, but you can't copy his method.

Instead of envying someone's \$5 billion, just faithfully max out your
\$7,000 each year.

After 40 years, you'll have \$1 million too. \$1 million tax-free is
pretty sweet.''
\end{quote}

\begin{center}\rule{0.5\linewidth}{0.5pt}\end{center}

\textbf{Forrest's Roth IRA Strategy:}

\begin{verbatim}
The Right Approach for Regular People:

1. Max out contributions each year ($7,000)
2. Invest in index funds (don't overthink it)
3. Don't touch it for 40 years (let compound interest work)
4. Withdraw tax-free in retirement

Expected result (7% annual return):
Start at 25 → $1,500,000 at 65
All tax-free.

Not $5 billion, but enough.
\end{verbatim}

\begin{center}\rule{0.5\linewidth}{0.5pt}\end{center}

\subsection{The Four Phases of
Retirement}\label{the-four-phases-of-retirement}

\subsubsection{Phase One: Accumulation
(25-45)}\label{phase-one-accumulation-25-45}

\textbf{Goal:} Save as much as possible

\begin{verbatim}
Strategy:
- Maximize 401k + Roth IRA
- Invest primarily in stocks
- Tolerate volatility, pursue growth
- Don't touch retirement accounts
\end{verbatim}

\begin{center}\rule{0.5\linewidth}{0.5pt}\end{center}

\subsubsection{Phase Two: Transition
(45-55)}\label{phase-two-transition-45-55}

\textbf{Goal:} Reduce risk, protect principal

\begin{verbatim}
Strategy:
- Gradually shift from stocks to bonds
- Start planning post-retirement life
- Calculate your "number"
- Consider early retirement possibility
\end{verbatim}

Sarah at 45: \textgreater{} ``I started doing serious math. Stocks went
from 80\% to 60\%. Bonds from 10\% to 30\%. Cash stayed at 10\%.''

\begin{center}\rule{0.5\linewidth}{0.5pt}\end{center}

\subsubsection{Phase Three: Early Retirement
(55-70)}\label{phase-three-early-retirement-55-70}

\textbf{Goal:} Active, free, with choices

\begin{verbatim}
Strategy:
- Health is the biggest investment
- Do what you want to do
- Expenses may be higher than when working (travel, hobbies)
- Keep some stock allocation (fight inflation)
\end{verbatim}

\begin{center}\rule{0.5\linewidth}{0.5pt}\end{center}

\subsubsection{Phase Four: Late Retirement
(70+)}\label{phase-four-late-retirement-70}

\textbf{Goal:} Simplify, peace of mind, legacy

\begin{verbatim}
Strategy:
- Expenses usually decrease
- Medical costs increase
- Consider estate planning
- Simplify the portfolio
\end{verbatim}

Mama said: \textgreater{} ``Son, at this age, money isn't the most
important thing. What matters is having someone to eat dinner with.''

\begin{center}\rule{0.5\linewidth}{0.5pt}\end{center}

\subsection{Early Retirement? The FIRE
Movement}\label{early-retirement-the-fire-movement}

\textbf{FIRE = Financial Independence, Retire Early}

\begin{verbatim}
Core Formula:
Savings rate determines retirement timeline

10% savings rate → Work 51 years
25% savings rate → Work 32 years
50% savings rate → Work 17 years
75% savings rate → Work 7 years
\end{verbatim}

\textbf{Sarah's View:} \textgreater{} ``FIRE is great, but it's not for
everyone. \textgreater{} Some people need the meaning of work.
\textgreater{} Some people need the social aspect of colleagues.
\textgreater{} Financial independence is just a tool, not the goal.''

\begin{center}\rule{0.5\linewidth}{0.5pt}\end{center}

\subsection{The Bitcoin Pizza: The Most Expensive Meal in
History}\label{the-bitcoin-pizza-the-most-expensive-meal-in-history}

On May 22, 2010, programmer Laszlo Hanyecz posted on Bitcoin Talk forum:

\begin{quote}
``I'll pay 10,000 bitcoins for two pizzas.''
\end{quote}

Someone took him up on it. Two Papa John's pizzas, worth \$41.

\textbf{How much is 10,000 bitcoin worth today?}

\begin{verbatim}
The Cost of Bitcoin Pizza:

2010: 10,000 BTC = $41 (two pizzas)
2013: 10,000 BTC = $1,000,000
2017: 10,000 BTC = $200,000,000
2021: 10,000 BTC = $690,000,000
2024: 10,000 BTC = $1,000,000,000

This is the most expensive pizza in human history.
\end{verbatim}

\begin{center}\rule{0.5\linewidth}{0.5pt}\end{center}

\textbf{Mike After Hearing This Story:}

``So I should buy bitcoin?''

Sarah said:

\begin{quote}
``You're missing the point.

In 2010, Laszlo had 10,000 bitcoins. In 2024, does he still have
bitcoin?

\textbf{Answer: Yes. He still holds some.}

But most early holders? Sold at \$100. Sold at \$1,000. Sold at
\$10,000.

\textbf{Buying right isn't hard. Holding on is.}''
\end{quote}

\begin{center}\rule{0.5\linewidth}{0.5pt}\end{center}

\textbf{Mama's Comment:}

\begin{quote}
``Son, everyone tells `if only I had bought XXX back then' stories.

But no one tells you: - 99\% of people who bought back then sold long
ago - Those who held for 14 years are one in ten thousand - The people
telling these stories didn't buy either

\textbf{Instead of regretting not buying bitcoin, think about what you
can hold for 14 years.}''
\end{quote}

\begin{center}\rule{0.5\linewidth}{0.5pt}\end{center}

\textbf{Forrest's Lesson:}

\begin{verbatim}
The Real Lessons from Bitcoin Pizza:

1. Buying early isn't as important as holding on
2. If it goes up 10x you'll sell, 100x you'll have run long ago
3. True wealth comes from time, not timing
4. Index funds grow slowly, but you can actually hold them for 40 years
\end{verbatim}

\begin{center}\rule{0.5\linewidth}{0.5pt}\end{center}

\subsection{The Chinese Capital Exodus: Who Pumped Up Bay Area
Housing?}\label{the-chinese-capital-exodus-who-pumped-up-bay-area-housing}

In 2012, Sarah wanted to buy a house in Palo Alto.

At the Open House, she witnessed something:

\begin{quote}
``House listed at \$1,200,000. 30 groups came to view. 20 of them were
Chinese. Final sale price: \$1,680,000. \textbf{All cash, no loan.}''
\end{quote}

\begin{center}\rule{0.5\linewidth}{0.5pt}\end{center}

\subsubsection{2010-2016: Chinese Hot Money Floods
In}\label{chinese-hot-money-floods-in}

\begin{verbatim}
Background:

2010: China's GDP surpasses Japan, becomes world's second largest
2012: Xi Jinping takes power, anti-corruption begins
2013: Wealthy Chinese start "moving assets overseas"
2015: RMB depreciates, capital flight accelerates

Destinations:
- Vancouver (too expensive already)
- Los Angeles (too many Chinese)
- Bay Area (kids need school + tech hub)

Especially: Palo Alto, Los Altos, Cupertino
Why? School districts.
\end{verbatim}

\begin{center}\rule{0.5\linewidth}{0.5pt}\end{center}

\subsubsection{Why Palo Alto and Los
Altos?}\label{why-palo-alto-and-los-altos}

\textbf{Mike's Observation:}

\begin{quote}
``In 2014, I went to look at houses in Los Altos.

A 1960s house, 3000 sqft. Listed at \$2,800,000.

I asked the agent: `Is this house worth that much?'

She said: `You're not buying a house. You're buying a school district.
\textbf{Los Altos High School, API 950+. Houses in this district always
have buyers.}'\,''
\end{quote}

\begin{center}\rule{0.5\linewidth}{0.5pt}\end{center}

\textbf{Chinese Buyers' Logic:}

\begin{verbatim}
Chinese Parents' Algorithm:

1. Kid needs to go to a good university
2. Good university needs good high school
3. Good high school needs good school district
4. Good school district = Palo Alto / Los Altos / Cupertino

Price? Doesn't matter.
Mortgage? Don't need one.
House is old? Can tear it down and rebuild.

Goal: Get kid into Stanford / Berkeley / MIT
\end{verbatim}

\begin{center}\rule{0.5\linewidth}{0.5pt}\end{center}

\subsubsection{The Crazy Housing Years}\label{the-crazy-housing-years}

\begin{verbatim}
Palo Alto Median Home Price:

2010: $1,200,000
2012: $1,500,000
2014: $2,200,000
2016: $2,800,000
2018: $3,200,000

167% increase in 8 years.
S&P 500 rose 150% in the same period.

But stocks can't get your kid into a good school.
\end{verbatim}

\begin{center}\rule{0.5\linewidth}{0.5pt}\end{center}

\textbf{Sarah's Analysis:}

\begin{quote}
``Chinese capital changed the rules of Bay Area real estate.

Before: Price = Income × reasonable multiple Now: Price = global capital
bidding war

You use a mortgage, they pay all cash. You buy based on income, they buy
based on assets. You calculate investment returns, they calculate their
child's future.

\textbf{Different game, different rules.}''
\end{quote}

\begin{center}\rule{0.5\linewidth}{0.5pt}\end{center}

\subsubsection{After 2017: The Tide
Recedes}\label{after-2017-the-tide-recedes}

\begin{verbatim}
Turning Points:

2017: China tightens foreign exchange controls, $50,000 per person per year limit
2018: US-China trade war begins
2020: Pandemic, Chinese buyers can't come to view houses
2022: Interest rates surge, prices finally stall

Results:
- Chinese buyers decrease significantly
- But prices didn't fall much
- Because inventory is also low
- Old homeowners have Prop 13 protection, don't want to sell
\end{verbatim}

\begin{center}\rule{0.5\linewidth}{0.5pt}\end{center}

\textbf{David's Reflection:}

\begin{quote}
``I paid \$300,000 for my house in 1995. In 2015 it was worth
\$2,500,000.

I thought I had great investment vision. Actually, the Chinese helped
pump up my price.

\textbf{I'm not a stock god. I was just lucky.}''
\end{quote}

\begin{center}\rule{0.5\linewidth}{0.5pt}\end{center}

\textbf{Mama's Summary:}

\begin{quote}
``Son, housing prices are short-term driven by capital, long-term driven
by population.

Chinese capital comes, prices rise. Chinese capital leaves, prices
stabilize.

But Silicon Valley's job opportunities are still here. Good school
districts' attraction is still here. People from all over the world
still want to come to America.

\textbf{You can't afford it? Normal. This game wasn't made for the
working class.}''
\end{quote}

\begin{center}\rule{0.5\linewidth}{0.5pt}\end{center}

\subsection{California Real Estate: The Hidden Bomb of
Retirement}\label{california-real-estate-the-hidden-bomb-of-retirement}

In 2022, Mike's neighbor David was about to retire.

``I bought my house in 1995 for \$300,000. Now it's worth \$2,500,000.''

``Congratulations!''

``\textbf{Congratulations for what? I want to sell and move to Texas,
but the federal capital gains tax would be \$440,000.}''

This is the first lesson of California retirement: \textbf{Your house
might be golden handcuffs.}

\begin{center}\rule{0.5\linewidth}{0.5pt}\end{center}

\subsubsection{Prop 13: California's Greatest
Law}\label{prop-13-californias-greatest-law}

\textbf{The 1978 Rules:}

\begin{verbatim}
Prop 13 Core:
- Property tax based on purchase price, not market value
- Maximum 2% increase per year
- If you don't sell, tax basis doesn't change
\end{verbatim}

\textbf{David's Example:}

\begin{verbatim}
1995 purchase: $300,000
2024 market value: $2,500,000

Under Prop 13: Annual property tax ≈ $5,000 (based on adjusted purchase price ~$420,000)
At market value: Annual property tax ≈ $30,000

Savings per year: $25,000
\end{verbatim}

Mama said: \textgreater{} ``Son, buying a house in California is like
getting a lifetime discount card. But once you sell, the discount is
gone.''

\begin{center}\rule{0.5\linewidth}{0.5pt}\end{center}

\subsubsection{1031 Exchange: The Art of Tax
Deferral}\label{exchange-the-art-of-tax-deferral}

\textbf{IRS Code Section 1031:}

\begin{verbatim}
Rules:
- Sell investment property, buy similar property
- Capital gains tax can be deferred
- Not tax-free, just deferred

Time Requirements:
- Identify new property within 45 days
- Complete transaction within 180 days

Limitations:
- Only applies to investment properties
- Doesn't apply to primary residence (has other exemptions)
- Must be "like-kind" exchange
\end{verbatim}

\textbf{David's Thought:} \textgreater{} ``Can I use a 1031 to exchange
my Palo Alto house for a Texas house?''

\textbf{Answer:} Yes. But you'll lose Prop 13 protection. And although
Texas has no state income tax, property taxes are 3x California's.

\begin{center}\rule{0.5\linewidth}{0.5pt}\end{center}

\subsubsection{Prop 19 Trap: The 2020 Sugar-Coated
Poison}\label{prop-19-trap-the-2020-sugar-coated-poison}

\textbf{The Surface ``Benefits'':}

\begin{verbatim}
Prop 19 (Effective 2020):

"Benefits":
- Age 55+ can transfer tax basis when moving (statewide)
- Also applies to disabled, disaster victims
- Can use up to 3 times

Sounds good? Wait...
\end{verbatim}

\textbf{The Real Trap:}

\begin{verbatim}
The Cost of Prop 19:

Old Rules (Prop 58):
- Parent's property passes to child, tax basis unchanged
- Child inherits $300,000 tax basis
- Continues enjoying low property taxes

New Rules (Prop 19):
- If inherited property isn't primary residence, reassess to market value
- Child inherits $2,500,000 market value house
- Property tax jumps from $5,000 to $30,000
- Unless child moves in
\end{verbatim}

\textbf{Mike's Analysis:} \textgreater{} ``Prop 19 is the government's
trick. \textgreater{} On the surface, giving seniors convenience when
moving. \textgreater{} Actually, eliminating tax basis protection for
children inheriting. \textgreater{} California government collects tens
of billions more in property taxes each year.''

\begin{center}\rule{0.5\linewidth}{0.5pt}\end{center}

\subsubsection{Retirement Property Strategy: Three
Paths}\label{retirement-property-strategy-three-paths}

\textbf{Path One: Stay in California}

\begin{verbatim}
Suitable for:
- Like California weather
- Children are local
- House is paid off

Strategy:
- Continue enjoying Prop 13
- Use house for HELOC emergency fund
- Let children inherit (if they're willing to live there)
\end{verbatim}

\textbf{Path Two: 1031 Exchange}

\begin{verbatim}
Suitable for:
- Have investment properties
- Want to move to another state
- Can accept managing new properties

Strategy:
- Use 1031 for investment properties
- Use $250k/$500k exemption for primary residence
- Process in batches, spread out tax impact
\end{verbatim}

\textbf{Path Three: Just Sell}

\begin{verbatim}
Suitable for:
- Want to simplify completely
- Don't care about taxes
- Children don't need the house

Strategy:
- Sell house, pay taxes, take cash
- Move to low-tax state, rent or buy cheaper house
- Use saved living expenses to cover taxes

The Math:
Primary residence sells for $2,500,000
Cost basis $300,000
Appreciation $2,200,000
Married exemption -$500,000
Taxable gain $1,700,000
Federal tax (20%) $340,000
California tax (13.3%) $226,000
Total tax ~$566,000

Sounds like a lot? But if you're 75,
remaining $1,934,000 in cash,
move to Texas and save $25,000/year in property taxes,
live to 90 and save $375,000.
\end{verbatim}

\begin{center}\rule{0.5\linewidth}{0.5pt}\end{center}

\subsubsection{David's Final Choice}\label{davids-final-choice}

In 2023, David was 65.

He calculated for three months.

Finally he said:

\begin{quote}
``\textbf{I'm not moving.}

I've lived in this house for 30 years. My wife's ashes are scattered in
the backyard. My daughter comes for dinner every week.

Prop 13 saves me \$25,000 every year. That money is enough for me to
live well.

After I die, let the kids decide. Either move in, or sell and pay taxes.
That's their problem, not mine.''
\end{quote}

Mama said: \textgreater{} ``Son, sometimes the best tax strategy is not
to sell at all.''

\begin{center}\rule{0.5\linewidth}{0.5pt}\end{center}

\subsection{The True Meaning of
Retirement}\label{the-true-meaning-of-retirement}

In 2024, two years after Sarah retired.

I asked her: ``What's the biggest discovery after retirement?''

She said:

\begin{quote}
``\textbf{Time is more valuable than money.}

I used to think retirement was about money. Now I know retirement is
about time.

Money can be earned again. Time cannot.

I now paint four hours every day, a hundred times happier than meetings
at Google.''
\end{quote}

\begin{center}\rule{0.5\linewidth}{0.5pt}\end{center}

\subsection{Forrest's Retirement
Checklist}\label{forrests-retirement-checklist}

\begin{verbatim}
Retirement Accounts:
□ Contribute at least to company 401k match
□ Open Roth IRA or Backdoor Roth
□ Know your retirement "number" (annual expenses × 25)
□ Have at least 2-3 income sources

Investment Allocation:
□ Adjust stock/bond mix with age
□ Don't put all retirement in one stock
□ Index funds as core, hold long-term

California Real Estate (if applicable):
□ Understand the value of Prop 13
□ Know Prop 19's impact on inheritance
□ Consider 1031 Exchange pros and cons
□ Calculate selling vs holding tax costs

Mental Preparation:
□ Know what you want to do after retirement
□ Buying right isn't hard, holding on is
□ Time is more valuable than money
\end{verbatim}

\begin{center}\rule{0.5\linewidth}{0.5pt}\end{center}

\subsection{One-Sentence Summary}\label{one-sentence-summary}

\begin{quote}
\textbf{Mama said: ``Son, retirement isn't being done living. It's
finally being able to live well.''}
\end{quote}

\begin{center}\rule{0.5\linewidth}{0.5pt}\end{center}

\subsection{The Three Teachers Say}\label{the-three-teachers-say}

{\def\LTcaptype{none} % do not increment counter
\begin{longtable}[]{@{}
  >{\raggedright\arraybackslash}p{(\linewidth - 2\tabcolsep) * \real{0.5294}}
  >{\raggedright\arraybackslash}p{(\linewidth - 2\tabcolsep) * \real{0.4706}}@{}}
\toprule\noalign{}
\begin{minipage}[b]{\linewidth}\raggedright
Teacher
\end{minipage} & \begin{minipage}[b]{\linewidth}\raggedright
Wisdom
\end{minipage} \\
\midrule\noalign{}
\endhead
\bottomrule\noalign{}
\endlastfoot
\textbf{Sun Tzu} & ``The good warrior takes his stand on ground where he
cannot lose'' --- Retirement planning is standing on unloseable
ground \\
\textbf{Graham} & ``The purpose of investing is to let compound interest
work for you'' --- 401k and Roth let compound interest work tax-free \\
\textbf{Bible} & ``Six days you shall labor and do all your work, but
the seventh day is a Sabbath'' (Exodus 20:9-10) --- Retirement is life's
Sabbath \\
\end{longtable}
}

\begin{center}\rule{0.5\linewidth}{0.5pt}\end{center}

\subsection{Further Reading}\label{further-reading}

\begin{itemize}
\tightlist
\item
  \href{01-never-all-in.md}{Chapter 1: Never Put All Eggs in One Basket}
  - Diversification
\item
  \href{05-index-funds.md}{Chapter 5: If You Don't Know What to Buy, Buy
  Index Funds} - Simple Strategy
\item
  \href{22-compound-time.md}{Chapter 22: Compound Interest Needs Time} -
  Patience
\item
  \href{23-real-wealth.md}{Chapter 23: True Wealth Is Time} - Freedom
\end{itemize}

\begin{center}\rule{0.5\linewidth}{0.5pt}\end{center}

\textbf{Previous Chapter:} \href{24-listen-to-mama.md}{Chapter 24:
Listen to Mama} \textbf{Next Chapter:} \href{26-healthcare.md}{Chapter
26: Healthcare Is Hidden Wealth}

\begin{center}\rule{0.5\linewidth}{0.5pt}\end{center}

\textbf{Version:} v0.1 \textbf{Updated:} 2025-12-30

\newpage

\section{26. Healthcare Is Hidden
Wealth}\label{healthcare-is-hidden-wealth}

\begin{quote}
\textbf{Mama said: ``Son, a healthy person has a thousand wishes. A sick
person has only one.''}
\end{quote}

\begin{center}\rule{0.5\linewidth}{0.5pt}\end{center}

\subsection{Sarah's Math Lesson
(Continued)}\label{sarahs-math-lesson-continued}

In 2020, Sarah was 48, preparing to retire.

``Wait,'' I said, ``you're only 48. Medicare doesn't start until 65.
What about the 17 years in between?''

Sarah pulled out a paper:

\begin{verbatim}
Healthcare options for ages 48-65:

Option 1: COBRA (Continue company insurance after leaving)
- $2,200/month (family)
- Only lasts 18 months
- Too expensive

Option 2: ACA (Obamacare)
- $800-1,500/month (depends on income and subsidies)
- Can use until 65
- This is my choice

Option 3: Spouse's insurance
- If spouse is still working
- This is the most cost-effective
\end{verbatim}

``\textbf{Healthcare is the biggest barrier to early retirement.} Many
people don't lack money to retire --- they lack insurance and are afraid
to retire.''

\begin{center}\rule{0.5\linewidth}{0.5pt}\end{center}

\subsection{The American Healthcare System: One Chart to Understand It
All}\label{the-american-healthcare-system-one-chart-to-understand-it-all}

\begin{verbatim}
                    American Healthcare Overview

┌─────────────────────────────────────────────────────┐
│                    Your Age                          │
├─────────────────────────────────────────────────────┤
│  0-26 years    │  Can stay on parents' insurance    │
│  26-65 years   │  Work insurance / ACA / Spouse     │
│  65+ years     │  Medicare (government insurance)   │
└─────────────────────────────────────────────────────┘

                    Special Cases
┌─────────────────────────────────────────────────────┐
│  Low income    │  Medicaid (varies by state)        │
│  Veterans      │  VA healthcare system              │
│  Disabled      │  Medicare (don't need to wait to 65)│
└─────────────────────────────────────────────────────┘
\end{verbatim}

\begin{center}\rule{0.5\linewidth}{0.5pt}\end{center}

\subsection{Employer Insurance: Silicon Valley's Hidden
Benefit}\label{employer-insurance-silicon-valleys-hidden-benefit}

\subsubsection{Why Big Company Insurance Is So
Good}\label{why-big-company-insurance-is-so-good}

\textbf{Google's Healthcare (Sarah's Example):}

\begin{verbatim}
Google Employee Healthcare Benefits:

- Premiums: Company pays 90%+, employee pays very little
- Deductible: $500 or less
- Out-of-pocket max: $3,000
- Includes: Dental, vision, mental health
- Extras: Gym, massage, nutrition counseling

Annual value: $20,000-30,000 (if you bought it yourself)
\end{verbatim}

\textbf{This is hidden salary.}

Many people only look at base salary and forget how much healthcare
benefits are worth.

\begin{center}\rule{0.5\linewidth}{0.5pt}\end{center}

\subsubsection{Small Company vs Big
Company}\label{small-company-vs-big-company}

{\def\LTcaptype{none} % do not increment counter
\begin{longtable}[]{@{}lll@{}}
\toprule\noalign{}
Item & Big Company (Google) & Small Company/Startup \\
\midrule\noalign{}
\endhead
\bottomrule\noalign{}
\endlastfoot
Premium split & Company 90\%+ & Company 50-70\% \\
Deductible & \$500 & \$2,000-5,000 \\
Network & PPO (choose any doctor) & HMO (restricted) \\
Extra benefits & Many & Few \\
\end{longtable}
}

\textbf{Mike's Lesson:}

\begin{quote}
``I joined a startup, got a 20\% raise. But healthcare went from PPO to
High Deductible. First year my son was hospitalized, I paid \$8,000 out
of pocket. That 20\% raise was gone.''
\end{quote}

\begin{center}\rule{0.5\linewidth}{0.5pt}\end{center}

\subsection{HSA: The Triple Tax-Free Super
Account}\label{hsa-the-triple-tax-free-super-account}

\subsubsection{What Is an HSA}\label{what-is-an-hsa}

\textbf{Health Savings Account}

Only available with High Deductible Health Plan (HDHP).

\subsubsection{Triple Tax-Free}\label{triple-tax-free}

\begin{verbatim}
HSA's Superpowers:

1. Contributions are tax-free — Deducted from pre-tax income
2. Growth is tax-free — Investment gains aren't taxed
3. Withdrawals are tax-free — Medical expenses aren't taxed

This is the ONLY "triple tax-free" account in US tax law.
Better than 401k (401k withdrawals are taxed).
Better than Roth IRA (Roth contributions are taxed).
\end{verbatim}

\subsubsection{2024 Limits}\label{limits}

{\def\LTcaptype{none} % do not increment counter
\begin{longtable}[]{@{}ll@{}}
\toprule\noalign{}
Type & Limit \\
\midrule\noalign{}
\endhead
\bottomrule\noalign{}
\endlastfoot
Individual & \$4,150 \\
Family & \$8,300 \\
Age 55+ catch-up & +\$1,000 \\
\end{longtable}
}

\subsubsection{The Hidden Retirement
Account}\label{the-hidden-retirement-account}

\textbf{Sarah's HSA Strategy:}

\begin{quote}
``I don't use HSA to pay current medical expenses. I pay with cash and
keep receipts. All HSA money goes into index fund investments.

After 20 years, this money grows tax-free. After 65, it can be used for
any expense (non-medical expenses taxed as income, but no penalty).

\textbf{HSA is a hidden Super Roth IRA.}''
\end{quote}

\begin{center}\rule{0.5\linewidth}{0.5pt}\end{center}

\subsection{ACA (Obamacare): The Savior of Early
Retirement}\label{aca-obamacare-the-savior-of-early-retirement}

\subsubsection{What Is ACA}\label{what-is-aca}

\textbf{Affordable Care Act}, passed in 2010.

Core rules: - Cannot deny coverage for pre-existing conditions - Must
cover essential health services - Subsidies for low income

\subsubsection{How Subsidies Are
Calculated}\label{how-subsidies-are-calculated}

\begin{verbatim}
ACA Subsidy Rules (2024):

Your income vs Federal Poverty Level (FPL)

Income < 150% FPL  →  Pay max 0-2% of income
Income 150-200% FPL →  Pay max 2-4% of income
Income 200-250% FPL →  Pay max 4-6% of income
Income 250-400% FPL →  Pay max 6-8.5% of income
Income > 400% FPL  →  Currently no cap (may change in 2025)

2024 FPL (48 states):
- Individual: $15,060
- 2-person family: $20,440
- 4-person family: $31,200
\end{verbatim}

\subsubsection{The Key to Early
Retirement}\label{the-key-to-early-retirement}

\textbf{Sarah's Strategy:}

\begin{verbatim}
When retiring in 2020:

Google stock: Mostly long-term capital gains
Sell a little each year: Control at around $60,000
Family income: $60,000 (about 192% FPL)

ACA premiums (2-person family):
- Market price: $1,800/month
- After subsidies: $400/month

Annual savings: $16,800
\end{verbatim}

\textbf{Key Technique: Control AGI (Adjusted Gross Income)}

\begin{itemize}
\tightlist
\item
  Selling stock counts as capital gains
\item
  401k withdrawals count as income
\item
  Roth IRA withdrawals don't count as income
\end{itemize}

``\textbf{Roth IRA is early retirement's best friend. Withdrawals don't
count as income and don't affect ACA subsidies.}''

\begin{center}\rule{0.5\linewidth}{0.5pt}\end{center}

\subsection{Medicare: The Gift at 65}\label{medicare-the-gift-at-65}

\subsubsection{Four Parts}\label{four-parts}

\begin{verbatim}
Medicare Structure:

Part A (Hospital)
- Free for most people (if worked 10+ years)
- Covers: Hospital stays, nursing facilities, hospice
- Has deductibles and copays

Part B (Outpatient)
- $174.70/month (2024 standard)
- Higher for high earners (IRMAA surcharge)
- Covers: Doctors, tests, outpatient surgery

Part C (Medicare Advantage)
- Offered by private insurance companies
- Combines A + B + usually includes D
- May have extra benefits (dental, vision)

Part D (Prescription Drugs)
- Purchased separately or included in Part C
- Different plans cover different drugs
\end{verbatim}

\subsubsection{IRMAA: The High-Income
Penalty}\label{irmaa-the-high-income-penalty}

\textbf{Income-Related Monthly Adjustment Amount}

\begin{verbatim}
2024 IRMAA Thresholds (Individual):

Income ≤ $103,000        →  Part B: $174.70
$103,000 - $129,000      →  Part B: $244.60
$129,000 - $161,000      →  Part B: $349.40
$161,000 - $193,000      →  Part B: $454.20
$193,000 - $500,000      →  Part B: $559.00
> $500,000               →  Part B: $594.00

Part D also has similar surcharges.
\end{verbatim}

\textbf{Note: It uses income from 2 years ago!}

Sarah: \textgreater{} ``I retired from Google in 2020, sold a lot of
stock that year. \textgreater{} In 2022 when I turned 65, Medicare used
my 2020 income. \textgreater{} Part B premiums went from \$170 to over
\$500. \textgreater{} \textgreater{} \textbf{Lesson: Plan your income
for the two years before retirement.}''

\begin{center}\rule{0.5\linewidth}{0.5pt}\end{center}

\subsection{Dental and Vision: The Forgotten
Corners}\label{dental-and-vision-the-forgotten-corners}

\subsubsection{The Problem}\label{the-problem}

Medicare doesn't cover: - Routine dental - Routine vision (eye exams,
glasses) - Hearing aids

These require separate insurance or out-of-pocket payment.

\subsubsection{Solutions}\label{solutions}

\begin{verbatim}
Option 1: Buy separate dental/vision insurance
- $30-50/month
- Limited coverage, mainly preventive checkups

Option 2: Medicare Advantage (Part C)
- Many plans include dental/vision
- Look carefully when choosing a plan

Option 3: Pay out of pocket
- Two cleanings per year: $200-400
- Eye exam + glasses: $300-500
- Might be cheaper than premiums

Option 4: Dental schools
- 50-70% cheaper
- But requires time and patience
\end{verbatim}

\begin{center}\rule{0.5\linewidth}{0.5pt}\end{center}

\subsection{Long-Term Care: The Biggest Hidden
Risk}\label{long-term-care-the-biggest-hidden-risk}

\subsubsection{The Numbers Speak}\label{the-numbers-speak}

\begin{verbatim}
Long-Term Care Costs (2024):

Nursing Home
- National average: $9,000/month
- California: $12,000/month
- New York: $14,000/month

Assisted Living
- National average: $4,500/month

Home Care
- $25-35/hour
- Full-time: $8,000-12,000/month

Probability of needing long-term care:
- Age 65+: About 70% will need some form of long-term care
- Average duration: Women 3.7 years, Men 2.2 years
\end{verbatim}

\textbf{Medicare doesn't cover long-term care!}

Only covers short-term rehabilitation (max 100 days).

\subsubsection{Options}\label{options}

\begin{verbatim}
Option 1: Long-Term Care Insurance (LTCI)
- Best to buy at 50-60
- Premiums increase with age
- Insurance companies may raise rates or go bankrupt

Option 2: Self-fund
- Assume needing $300,000-500,000
- Keep in separate account

Option 3: Medicaid (last resort)
- Must "spend down" assets
- Only eligible with under $2,000
- Limited choices

Option 4: Hybrid Insurance
- Life insurance + long-term care
- If unused, goes to heirs
\end{verbatim}

\textbf{Mama's Attitude:}

\begin{quote}
``Son, I don't want to go to a nursing home. I want to pass away in my
own home. Just hire someone for me.''
\end{quote}

\begin{center}\rule{0.5\linewidth}{0.5pt}\end{center}

\subsection{Medical Tourism: Silicon Valley's
Secret}\label{medical-tourism-silicon-valleys-secret}

\subsubsection{Why}\label{why}

US medical costs are the highest in the world.

\begin{verbatim}
Same Surgery:

Knee Replacement:
- US: $50,000
- Mexico: $12,000
- Thailand: $10,000
- India: $7,000

Dental Implant (single):
- US: $5,000
- Mexico: $1,500
- Costa Rica: $1,000
\end{verbatim}

\subsubsection{Sarah's Experience}\label{sarahs-experience}

\begin{quote}
``I got dental work done in Mexico. Los Algodones, a border town. There
are over 300 dental clinics there.

Four crowns + cleaning: \$2,000 US quote: \$8,000

The dentist graduated from an American dental school. The equipment was
imported from Germany. The only difference was the price.''
\end{quote}

\subsubsection{Considerations}\label{considerations}

\begin{itemize}
\tightlist
\item
  Do your research (check reviews, certifications)
\item
  Think twice about complex surgeries
\item
  Calculate travel costs
\item
  Consider post-operative care
\end{itemize}

\begin{center}\rule{0.5\linewidth}{0.5pt}\end{center}

\subsection{Health Is the Greatest
Wealth}\label{health-is-the-greatest-wealth}

\subsubsection{Mike's Story}\label{mikes-story}

In 2018, Mike was 52.

Physical exam found: Type 2 diabetes.

\begin{verbatim}
Mike's Medical Bills (Annual):

Medications: $3,000
Tests: $1,500
Specialists: $2,000
Total: $6,500/year

What if he had started exercising 10 years ago?
\end{verbatim}

\textbf{Mama said:}

\begin{quote}
``Son, gym membership is \$50/month. Diabetes medication is \$300/month.
You do the math.''
\end{quote}

\subsubsection{Prevention vs Treatment}\label{prevention-vs-treatment}

\begin{verbatim}
Returns on Investing in Health:

Exercise (30 minutes daily)
- Cost: $0-50/month
- Return: Reduced risk of heart disease, diabetes, cancer

Healthy Diet
- Cost: Maybe 20% more than junk food
- Return: Fewer medical expenses

Annual Physical
- Cost: Usually fully covered by insurance
- Return: Early detection, early treatment

This is true "investing."
\end{verbatim}

\begin{center}\rule{0.5\linewidth}{0.5pt}\end{center}

\subsection{Forrest's Healthcare
Checklist}\label{forrests-healthcare-checklist}

\begin{verbatim}
While Working:
□ Maximize company healthcare benefits
□ If you have HDHP, open and max out HSA
□ Understand COBRA rules (18 months after leaving job)
□ Consider spouse's insurance as backup

Early Retirement (Before 65):
□ Calculate ACA premiums and subsidies
□ Plan income to maximize subsidies
□ Use Roth IRA withdrawals (don't count as income)
□ Keep HSA, let it continue growing

After 65:
□ Apply for Medicare 3 months before 65th birthday
□ Choose Original Medicare or Medicare Advantage
□ Consider Medigap supplemental insurance
□ Watch out for IRMAA (uses income from 2 years ago)

Long-Term Planning:
□ Consider long-term care risk
□ Prevention is better than treatment
□ Health is the best investment
\end{verbatim}

\begin{center}\rule{0.5\linewidth}{0.5pt}\end{center}

\subsection{One-Sentence Summary}\label{one-sentence-summary-1}

\begin{quote}
\textbf{Mama said: ``Son, rich without health is nothing. Healthy
without money, you can still earn slowly.''}
\end{quote}

\begin{center}\rule{0.5\linewidth}{0.5pt}\end{center}

\subsection{The Three Teachers Say}\label{the-three-teachers-say-1}

{\def\LTcaptype{none} % do not increment counter
\begin{longtable}[]{@{}
  >{\raggedright\arraybackslash}p{(\linewidth - 2\tabcolsep) * \real{0.5294}}
  >{\raggedright\arraybackslash}p{(\linewidth - 2\tabcolsep) * \real{0.4706}}@{}}
\toprule\noalign{}
\begin{minipage}[b]{\linewidth}\raggedright
Teacher
\end{minipage} & \begin{minipage}[b]{\linewidth}\raggedright
Wisdom
\end{minipage} \\
\midrule\noalign{}
\endhead
\bottomrule\noalign{}
\endlastfoot
\textbf{Sun Tzu} & ``Those who arrive first at the battlefield await the
enemy at ease'' --- Plan healthcare early, don't panic later \\
\textbf{Graham} & ``Margin of safety is the foundation of investing''
--- Health insurance is life's margin of safety \\
\textbf{Bible} & ``Do you not know that your bodies are temples of the
Holy Spirit?'' (1 Corinthians 6:19) --- Health is the greatest wealth \\
\end{longtable}
}

\begin{center}\rule{0.5\linewidth}{0.5pt}\end{center}

\subsection{Further Reading}\label{further-reading-1}

\begin{itemize}
\tightlist
\item
  \href{25-retirement.md}{Chapter 25: Retirement Is Not the End, It's a
  Turning Point} - 401k/IRA/4\% Rule
\item
  \href{02-emergency-fund.md}{Chapter 2: Save Six Months of Living
  Expenses First} - Safety Net
\item
  \href{23-real-wealth.md}{Chapter 23: True Wealth Is Time} - Freedom
\end{itemize}

\begin{center}\rule{0.5\linewidth}{0.5pt}\end{center}

\textbf{Previous Chapter:} \href{25-retirement.md}{Chapter 25:
Retirement Is Not the End, It's a Turning Point} \textbf{Appendix:}
\href{appendix-b-rules.md}{Forrest's 10 Iron Rules}

\begin{center}\rule{0.5\linewidth}{0.5pt}\end{center}

\textbf{Version:} v0.1 \textbf{Updated:} 2025-12-30

\newpage

\end{document}
