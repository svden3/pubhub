% SGP Book Template - Chinese Version (Simple)
% Requires XeLaTeX for CJK support

\documentclass[11pt,openright,twoside]{book}

% Font setup with fontspec (XeLaTeX)
\usepackage{fontspec}

% Use macOS system fonts that support CJK
\setmainfont{Songti SC}[
  BoldFont=Songti SC Bold,
  ItalicFont=Kaiti SC
]
\setsansfont{Heiti SC}
\setmonofont{Menlo}

% Enable CJK line breaking
\XeTeXlinebreaklocale "zh"
\XeTeXlinebreakskip = 0pt plus 1pt minus 0.1pt

% Page geometry - book format (6x9 inches)
\usepackage{geometry}
\geometry{
  paperwidth=6in,
  paperheight=9in,
  top=0.75in,
  bottom=0.75in,
  inner=0.875in,
  outer=0.625in
}

% Line spacing
\linespread{1.3}

% Colors
\usepackage{xcolor}
\definecolor{chaptercolor}{RGB}{139,0,0}

% Hyperlinks
\usepackage{hyperref}
\hypersetup{
  colorlinks=true,
  linkcolor=chaptercolor,
  urlcolor=blue,
  bookmarks=true,
  bookmarksnumbered=true,
  pdfencoding=auto
}

% Graphics
\usepackage{graphicx}

% Tables
\usepackage{longtable}
\usepackage{booktabs}
\usepackage{array}

% Pandoc 3.x table support
\usepackage{etoolbox}
\makeatletter
\patchcmd\longtable{\par}{\if@noskipsec\mbox{}\fi\par}{}{}
\@ifundefined{c@none}{\newcounter{none}}{}
\makeatother

% Widow and orphan control
\widowpenalty=10000
\clubpenalty=10000

% Metadata
\title{阿甘的投资备忘录}
\author{Jim Xiao}
\date{2025年12月}

% Pandoc tight list fix
\providecommand{\tightlist}{%
  \setlength{\itemsep}{0pt}\setlength{\parskip}{0pt}}

\begin{document}

% Front matter
\frontmatter

% Title page
\begin{titlepage}
\thispagestyle{empty}
\vspace*{3cm}
\begin{center}
{\Huge\bfseries 阿甘的投资备忘录}\\[1cm]
{\Large\itshape 25年硅谷生存智慧 · 简单实用版}\\[2cm]
{\Large Jim Xiao}\\[1cm]
\vfill
{\large 2025年12月}
\end{center}
\end{titlepage}

% Copyright page
\thispagestyle{empty}
\vspace*{\fill}
\begin{flushleft}
\small
版权所有 \copyright\ \the\year\ Jim Xiao\\
保留所有权利。
\end{flushleft}
\cleardoublepage

% Table of contents
\tableofcontents
\cleardoublepage

% Main matter
\mainmatter

\section{阿甘的投资备忘录}\label{ux963fux7518ux7684ux6295ux8d44ux5907ux5fd8ux5f55}

\begin{quote}
\textbf{25 年硅谷生存智慧 · 简单实用版}
\end{quote}

\begin{center}\rule{0.5\linewidth}{0.5pt}\end{center}

\subsection{关于本书}\label{ux5173ux4e8eux672cux4e66}

这是《硅谷阿甘》的姊妹篇。

如果说《硅谷阿甘》是一部电影,那这本书就是\textbf{电影手册}------把阿甘
25 年的故事,提炼成你可以直接用的\textbf{简单规则}。

\textbf{不讲理论,只讲做法。} \textbf{不讲为什么,只讲怎么做。}

\begin{center}\rule{0.5\linewidth}{0.5pt}\end{center}

\subsection{书籍信息}\label{ux4e66ux7c4dux4fe1ux606f}

\begin{itemize}
\tightlist
\item
  \textbf{定位}:实用手册 · 新手友好
\item
  \textbf{阅读时间}:2 小时
\item
  \textbf{配套阅读}:\href{../silicon-jim.md}{硅谷阿甘(完整故事)}
\item
  \textbf{进阶阅读}:\href{../00-index.md}{兵法投资道(系统理论)}
\end{itemize}

\begin{center}\rule{0.5\linewidth}{0.5pt}\end{center}

\subsection{目录}\label{ux76eeux5f55}

\subsubsection{第一部分:基本功}\label{ux7b2cux4e00ux90e8ux5206ux57faux672cux529f}

\begin{quote}
``妈妈说:先学会不输,再学怎么赢。''
\end{quote}

{\def\LTcaptype{none} % do not increment counter
\begin{longtable}[]{@{}lll@{}}
\toprule\noalign{}
章节 & 标题 & 核心 \\
\midrule\noalign{}
\endhead
\bottomrule\noalign{}
\endlastfoot
01 & \href{01-never-all-in.md}{永远不要全押一个篮子} & 分散投资 \\
02 & \href{02-emergency-fund.md}{存够六个月生活费再说} & 安全垫 \\
03 & \href{03-understand-first.md}{听不懂就不要投} & 能力圈 \\
\end{longtable}
}

\subsubsection{第二部分:怎么买}\label{ux7b2cux4e8cux90e8ux5206ux600eux4e48ux4e70}

\begin{quote}
``Sarah 说:买什么不重要,重要的是什么时候买。''
\end{quote}

{\def\LTcaptype{none} % do not increment counter
\begin{longtable}[]{@{}lll@{}}
\toprule\noalign{}
章节 & 标题 & 核心 \\
\midrule\noalign{}
\endhead
\bottomrule\noalign{}
\endlastfoot
04 & \href{04-fear-and-greed.md}{别人恐惧时贪婪} & 逆向投资 \\
05 & \href{05-index-funds.md}{不知道买什么就买指数} & 简单策略 \\
06 & \href{06-dollar-cost-averaging.md}{分批买入不择时} & 定投 \\
\end{longtable}
}

\subsubsection{第三部分:怎么卖}\label{ux7b2cux4e09ux90e8ux5206ux600eux4e48ux5356}

\begin{quote}
``妈妈说:会买的是徒弟,会卖的才是师傅。''
\end{quote}

{\def\LTcaptype{none} % do not increment counter
\begin{longtable}[]{@{}lll@{}}
\toprule\noalign{}
章节 & 标题 & 核心 \\
\midrule\noalign{}
\endhead
\bottomrule\noalign{}
\endlastfoot
07 & \href{07-hold-through.md}{持有是最难的事} & 长期持有 \\
08 & \href{08-when-to-sell.md}{什么时候该走} & 卖出信号 \\
09 & \href{09-no-chasing.md}{不要追涨杀跌} & 纪律 \\
\end{longtable}
}

\subsubsection{第四部分:避坑指南}\label{ux7b2cux56dbux90e8ux5206ux907fux5751ux6307ux5357}

\begin{quote}
``Mike 说:我踩过的坑,你就不用再踩了。''
\end{quote}

{\def\LTcaptype{none} % do not increment counter
\begin{longtable}[]{@{}lll@{}}
\toprule\noalign{}
章节 & 标题 & 真实案例 \\
\midrule\noalign{}
\endhead
\bottomrule\noalign{}
\endlastfoot
10 & \href{10-sock-puppet.md}{袜子木偶的教训} & 2000 互联网泡沫 \\
11 & \href{11-enron.md}{安然式的创新} & 2001 安然 \\
12 & \href{12-lehman.md}{大到不能倒也会倒} & 2008 雷曼 \\
13 & \href{13-meme-stocks.md}{表情包不是投资} & 2021 GME/Crypto \\
14 & \href{14-ftx.md}{有效利他还是有效挪用} & 2022 FTX \\
\end{longtable}
}

\subsubsection{第五部分:职场生存}\label{ux7b2cux4e94ux90e8ux5206ux804cux573aux751fux5b58}

\begin{quote}
``妈妈说:不是每场仗都值得打。''
\end{quote}

{\def\LTcaptype{none} % do not increment counter
\begin{longtable}[]{@{}
  >{\raggedright\arraybackslash}p{(\linewidth - 4\tabcolsep) * \real{0.3125}}
  >{\raggedright\arraybackslash}p{(\linewidth - 4\tabcolsep) * \real{0.3750}}
  >{\raggedright\arraybackslash}p{(\linewidth - 4\tabcolsep) * \real{0.3125}}@{}}
\toprule\noalign{}
\begin{minipage}[b]{\linewidth}\raggedright
章节
\end{minipage} & \begin{minipage}[b]{\linewidth}\raggedright
标题
\end{minipage} & \begin{minipage}[b]{\linewidth}\raggedright
核心
\end{minipage} \\
\midrule\noalign{}
\endhead
\bottomrule\noalign{}
\endlastfoot
15 & \href{15-dont-buy-company-stock.md}{别把人力资本和金融资本放一起} &
工作与投资分开 \\
16 & \href{16-plan-b.md}{永远要有 Plan B} & 存款、技能、人脉 \\
17 & \href{17-best-revenge.md}{最好的报复是过得好} & 离开的智慧 \\
\end{longtable}
}

\subsubsection{第六部分:子女教育}\label{ux7b2cux516dux90e8ux5206ux5b50ux5973ux6559ux80b2}

\begin{quote}
``妈妈说:最好的投资,是投资在孩子身上------但不是你想的那种。''
\end{quote}

{\def\LTcaptype{none} % do not increment counter
\begin{longtable}[]{@{}lll@{}}
\toprule\noalign{}
章节 & 标题 & 核心 \\
\midrule\noalign{}
\endhead
\bottomrule\noalign{}
\endlastfoot
18 & \href{18-529-plan.md}{529 计划:税务利器} & 教育储蓄 \\
19 & \href{19-public-vs-private.md}{公立还是私立} & 选择智慧 \\
20 & \href{20-beyond-ivy.md}{爬藤不是唯一的路} & 多元路径 \\
21 & \href{21-kids-financial-literacy.md}{财商教育从小开始} &
言传身教 \\
\end{longtable}
}

\subsubsection{第七部分:人生智慧}\label{ux7b2cux4e03ux90e8ux5206ux4ebaux751fux667aux6167}

\begin{quote}
``25 年后,我学到的最重要的事。''
\end{quote}

{\def\LTcaptype{none} % do not increment counter
\begin{longtable}[]{@{}lll@{}}
\toprule\noalign{}
章节 & 标题 & 核心 \\
\midrule\noalign{}
\endhead
\bottomrule\noalign{}
\endlastfoot
22 & \href{22-compound-time.md}{复利需要时间} & 耐心 \\
23 & \href{23-real-wealth.md}{真正的财富是时间} & 自由 \\
24 & \href{24-listen-to-mama.md}{听妈妈的话} & 常识 \\
\end{longtable}
}

\subsubsection{第八部分:退休规划}\label{ux7b2cux516bux90e8ux5206ux9000ux4f11ux89c4ux5212}

\begin{quote}
``Sarah 说:退休不是停止工作,是可以选择工作。''
\end{quote}

{\def\LTcaptype{none} % do not increment counter
\begin{longtable}[]{@{}lll@{}}
\toprule\noalign{}
章节 & 标题 & 核心 \\
\midrule\noalign{}
\endhead
\bottomrule\noalign{}
\endlastfoot
25 & \href{25-retirement.md}{退休不是终点,是转折点} &
401k/IRA/4\%法则 \\
\end{longtable}
}

\subsubsection{附录}\label{ux9644ux5f55}

{\def\LTcaptype{none} % do not increment counter
\begin{longtable}[]{@{}ll@{}}
\toprule\noalign{}
附录 & 内容 \\
\midrule\noalign{}
\endhead
\bottomrule\noalign{}
\endlastfoot
\href{appendix-a-checklist.md}{附录 A} & 投资决策检查清单 \\
\href{appendix-b-rules.md}{附录 B} & 阿甘的 10 条铁律 \\
\href{appendix-c-mama-says.md}{附录 C} & 妈妈语录全集 \\
\href{appendix-d-sarah-says.md}{附录 D} & Sarah 语录全集 \\
\end{longtable}
}

\begin{center}\rule{0.5\linewidth}{0.5pt}\end{center}

\subsection{怎么读这本书}\label{ux600eux4e48ux8bfbux8fd9ux672cux4e66}

\textbf{如果你是新手:} - 从第一部分开始,一章章读 - 每章只有 3-5 分钟

\textbf{如果你已经踩过坑:} - 直接翻到第四部分,看看你踩的是哪个 -
然后读对应的解决方案

\textbf{如果你只有 10 分钟:} - 直接翻到附录 B:《阿甘的 10 条铁律》

\begin{center}\rule{0.5\linewidth}{0.5pt}\end{center}

\subsection{开始之前}\label{ux5f00ux59cbux4e4bux524d}

妈妈说:

\begin{quote}
``儿子,投资很简单。难的是做到。''
\end{quote}

好了,翻开下一页。

\begin{center}\rule{0.5\linewidth}{0.5pt}\end{center}

\textbf{下一章}:\href{01-never-all-in.md}{第一章:永远不要全押一个篮子}

\begin{center}\rule{0.5\linewidth}{0.5pt}\end{center}

\textbf{版本}:v0.1 \textbf{更新日期}:2025-12-30

\newpage

\section{第一章:永远不要全押一个篮子}\label{ux7b2cux4e00ux7ae0ux6c38ux8fdcux4e0dux8981ux5168ux62bcux4e00ux4e2aux7beeux5b50}

\begin{quote}
``妈妈说:鸡蛋放一个篮子里,篮子掉了,全碎了。''
\end{quote}

\begin{center}\rule{0.5\linewidth}{0.5pt}\end{center}

\subsection{故事}\label{ux6545ux4e8b}

Mike 在安然工作。

公司给了他 5 万美元的 signing bonus。他全买了公司股票。

``这是全美最创新的公司!'' 他说。

2001 年,安然破产。他那 5 万块的股票,变成了 37 美分。

后来他去了雷曼。又把所有钱押在曼哈顿的房子上。

2008 年,雷曼倒闭。房子亏了 40\%。

\textbf{Mike 犯了同一个错误两次。}

\begin{center}\rule{0.5\linewidth}{0.5pt}\end{center}

\subsection{规则}\label{ux89c4ux5219}

\subsubsection{规则
1:不要把所有钱放一个股票}\label{ux89c4ux5219-1ux4e0dux8981ux628aux6240ux6709ux94b1ux653eux4e00ux4e2aux80a1ux7968}

不管你多相信一家公司,最多放 10\% 的钱。

\subsubsection{规则
2:不要买你工作的公司的股票}\label{ux89c4ux5219-2ux4e0dux8981ux4e70ux4f60ux5de5ux4f5cux7684ux516cux53f8ux7684ux80a1ux7968}

你的工资已经押在那里了。如果公司倒了,你失业 +
股票归零。这叫\textbf{双重风险}。

\subsubsection{规则
3:不要把所有钱放一家银行}\label{ux89c4ux5219-3ux4e0dux8981ux628aux6240ux6709ux94b1ux653eux4e00ux5bb6ux94f6ux884c}

2023 年,硅谷银行 48 小时倒闭。很多创业公司差点发不出工资。

FDIC 只保 25 万美元。超过这个数,分到两家银行。

\begin{center}\rule{0.5\linewidth}{0.5pt}\end{center}

\subsection{怎么做}\label{ux600eux4e48ux505a}

\subsubsection{最简单的分散法}\label{ux6700ux7b80ux5355ux7684ux5206ux6563ux6cd5}

\begin{verbatim}
你的投资 = 美股指数 + 美债 + 现金

比例:
- 年轻人:70% 股 + 20% 债 + 10% 现金
- 中年人:50% 股 + 30% 债 + 20% 现金
- 退休前:30% 股 + 50% 债 + 20% 现金
\end{verbatim}

\subsubsection{一句话}\label{ux4e00ux53e5ux8bdd}

\begin{quote}
\textbf{不要把所有鸡蛋放在一个篮子里,尤其是你工作的那个篮子。}
\end{quote}

\begin{center}\rule{0.5\linewidth}{0.5pt}\end{center}

\subsection{检查清单}\label{ux68c0ux67e5ux6e05ux5355}

\begin{itemize}
\tightlist
\item[$\square$]
  单只股票占比不超过 10\%
\item[$\square$]
  没有买自己公司的股票(或者很少)
\item[$\square$]
  存款分在两家银行以上
\item[$\square$]
  有股票、有债券、有现金
\end{itemize}

\begin{center}\rule{0.5\linewidth}{0.5pt}\end{center}

\textbf{下一章}:\href{02-emergency-fund.md}{存够六个月生活费再说}

\newpage

\section{第二章:存够六个月生活费再说}\label{ux7b2cux4e8cux7ae0ux5b58ux591fux516dux4e2aux6708ux751fux6d3bux8d39ux518dux8bf4}

\begin{quote}
``Sarah 说:真正的安全感来自你银行账户里的数字。''
\end{quote}

\begin{center}\rule{0.5\linewidth}{0.5pt}\end{center}

\subsection{故事}\label{ux6545ux4e8b-1}

2011 年,NeuralMind 倒闭了。

我的 0.05\% 期权变成了废纸。

十一年的人生,装进一个纸箱。

但我没有慌。

因为我有存款。因为我的指数基金还在赚钱。

Sarah
说:``\textbf{看,这就是分散投资的意义。你的人力资本归零了,但你的金融资本还在。}''

\begin{center}\rule{0.5\linewidth}{0.5pt}\end{center}

\subsection{规则}\label{ux89c4ux5219-1}

\subsubsection{规则
1:先存紧急备用金}\label{ux89c4ux5219-1ux5148ux5b58ux7d27ux6025ux5907ux7528ux91d1}

在投资任何东西之前,先存够 \textbf{6 个月的生活费}。

这笔钱: - 放在随时能取的地方(货币基金或高息储蓄账户) - 不投资股票 -
只在紧急情况用

\subsubsection{规则
2:紧急情况的定义}\label{ux89c4ux5219-2ux7d27ux6025ux60c5ux51b5ux7684ux5b9aux4e49}

✅ 失业 ✅ 生病 ✅ 意外支出

❌ 股市大跌想抄底 ❌ 想买新车 ❌ 想去旅游

\subsubsection{规则
3:存够再投资}\label{ux89c4ux5219-3ux5b58ux591fux518dux6295ux8d44}

顺序很重要:

\begin{verbatim}
第一步:还清高息债务(信用卡)
    ↓
第二步:存够 6 个月紧急备用金
    ↓
第三步:开始投资
\end{verbatim}

\begin{center}\rule{0.5\linewidth}{0.5pt}\end{center}

\subsection{怎么算}\label{ux600eux4e48ux7b97}

\subsubsection{你的月支出}\label{ux4f60ux7684ux6708ux652fux51fa}

{\def\LTcaptype{none} % do not increment counter
\begin{longtable}[]{@{}ll@{}}
\toprule\noalign{}
项目 & 金额 \\
\midrule\noalign{}
\endhead
\bottomrule\noalign{}
\endlastfoot
房租/房贷 & \$\_\_\_\_\_\_ \\
吃饭 & \$\_\_\_\_\_\_ \\
交通 & \$\_\_\_\_\_\_ \\
保险 & \$\_\_\_\_\_\_ \\
其他必要支出 & \(______ |
| **月总支出** | **\)\_\_\_\_\_\_** \\
\end{longtable}
}

\subsubsection{你需要的紧急备用金}\label{ux4f60ux9700ux8981ux7684ux7d27ux6025ux5907ux7528ux91d1}

\textbf{月总支出 × 6 = 紧急备用金}

例子: - 月支出 \$5,000 - 紧急备用金 = \$5,000 × 6 = \textbf{\$30,000}

\begin{center}\rule{0.5\linewidth}{0.5pt}\end{center}

\subsection{一句话}\label{ux4e00ux53e5ux8bdd-1}

\begin{quote}
\textbf{没有安全垫,别上战场。}
\end{quote}

\begin{center}\rule{0.5\linewidth}{0.5pt}\end{center}

\subsection{检查清单}\label{ux68c0ux67e5ux6e05ux5355-1}

\begin{itemize}
\tightlist
\item[$\square$]
  我知道我的月支出是多少
\item[$\square$]
  我有 6 个月的紧急备用金
\item[$\square$]
  这笔钱在我能随时取到的地方
\item[$\square$]
  我没有动用这笔钱去投资
\end{itemize}

\begin{center}\rule{0.5\linewidth}{0.5pt}\end{center}

\textbf{下一章}:\href{03-understand-first.md}{听不懂就不要投}

\newpage

\section{第三章:听不懂就不要投}\label{ux7b2cux4e09ux7ae0ux542cux4e0dux61c2ux5c31ux4e0dux8981ux6295}

\begin{quote}
``妈妈说:如果一个人说的话你听不懂,要么他是天才,要么他是骗子。天才很少。''
\end{quote}

\begin{center}\rule{0.5\linewidth}{0.5pt}\end{center}

\subsection{故事}\label{ux6545ux4e8b-2}

\subsubsection{安然}\label{ux5b89ux7136}

Mike 问 Sarah:``安然是做什么的?''

``他们\ldots{} 创造价值。''

``怎么创造?''

``通过复杂的金融工具。''

Sarah 的爸爸是
CPA。他说:``\textbf{如果一个公司的财报需要天才才能看懂,要么他们在做诺贝尔奖级别的创新,要么他们在做牢狱级别的造假。}''

安然倒闭了。CEO 进监狱了。

\subsubsection{Theranos}\label{theranos}

Elizabeth Holmes 说:``一滴血检测所有疾病!''

猎头找我:``Theranos 在招 AI 工程师。''

Sarah
问她爸。她爸说:``这家医疗科技公司,没有发过一篇经过同行评审的论文。''

我拒绝了那个 offer。

Theranos 倒闭了。Holmes 被判欺诈。

那台机器根本不工作。

\begin{center}\rule{0.5\linewidth}{0.5pt}\end{center}

\subsection{规则}\label{ux89c4ux5219-2}

\subsubsection{规则
1:三句话原则}\label{ux89c4ux5219-1ux4e09ux53e5ux8bddux539fux5219}

如果你不能用三句话解释一个投资是怎么赚钱的,\textbf{不要买}。

✅ 能通过: - ``苹果卖 iPhone,用户越多越赚钱'' -
``指数基金买一篮子股票,经济涨它就涨'' - ``这套房子出租,每月租金 2000''

❌ 不能通过: - ``它通过复杂的金融工具创造价值'' -
``这是去中心化的元宇宙生态系统'' - ``他们用 AI 颠覆传统行业''

\subsubsection{规则
2:问三个问题}\label{ux89c4ux5219-2ux95eeux4e09ux4e2aux95eeux9898}

\begin{enumerate}
\def\labelenumi{\arabic{enumi}.}
\tightlist
\item
  \textbf{它怎么赚钱?}
\item
  \textbf{为什么是它,不是别人?}
\item
  \textbf{如果我错了,会亏多少?}
\end{enumerate}

答不上来,就不要投。

\subsubsection{规则 3:复杂 ≠
高级}\label{ux89c4ux5219-3ux590dux6742-ux9ad8ux7ea7}

骗子喜欢用复杂的词。

真正懂的人可以把复杂的事情说简单。

\textbf{如果对方说不清楚,要么他不懂,要么他在骗你。}

\begin{center}\rule{0.5\linewidth}{0.5pt}\end{center}

\subsection{避开的信号}\label{ux907fux5f00ux7684ux4fe1ux53f7}

当你听到这些话时,要小心:

{\def\LTcaptype{none} % do not increment counter
\begin{longtable}[]{@{}ll@{}}
\toprule\noalign{}
危险信号 & 翻译 \\
\midrule\noalign{}
\endhead
\bottomrule\noalign{}
\endlastfoot
``太复杂了你不会懂的'' & 我解释不清楚 \\
``这是颠覆性创新'' & 我们还没赚钱 \\
``错过这次就没有了'' & 我要你快点做决定 \\
``回报率保证 20\%'' & 这很可能是骗局 \\
``有效利他主义'' & 我用你的钱做慈善 \\
\end{longtable}
}

\begin{center}\rule{0.5\linewidth}{0.5pt}\end{center}

\subsection{一句话}\label{ux4e00ux53e5ux8bdd-2}

\begin{quote}
\textbf{能力圈之外的钱,不是你的钱。}
\end{quote}

\begin{center}\rule{0.5\linewidth}{0.5pt}\end{center}

\subsection{检查清单}\label{ux68c0ux67e5ux6e05ux5355-2}

\begin{itemize}
\tightlist
\item[$\square$]
  我能用三句话解释这个投资怎么赚钱
\item[$\square$]
  我知道如果错了会亏多少
\item[$\square$]
  我没有被''复杂''或''专业''的词吓住
\item[$\square$]
  我没有被''错过就没有了''逼着做决定
\end{itemize}

\begin{center}\rule{0.5\linewidth}{0.5pt}\end{center}

\textbf{下一章}:\href{04-fear-and-greed.md}{别人恐惧时贪婪}

\newpage

\section{第四章:别人恐惧时贪婪}\label{ux7b2cux56dbux7ae0ux522bux4ebaux6050ux60e7ux65f6ux8d2aux5a6a}

\begin{quote}
``Sarah 说:恐惧是最好的买入信号。''
\end{quote}

\begin{center}\rule{0.5\linewidth}{0.5pt}\end{center}

\subsection{故事}\label{ux6545ux4e8b-3}

2008 年 9 月,雷曼兄弟破产。

全世界都在恐慌。

Sarah 打电话给我:

``Jim,你手上有多少现金?''

``大概 15 万。''

``现在是买股票的时候。''

``你疯了?世界要完蛋了!''

``\textbf{世界不会完蛋。恐惧是最好的买入信号。}''

我用 10 万美元买了 S\&P 500 指数基金。那时候是 1,200 点。

2009 年 3 月,跌到 666 点。我的 10 万变成了 5.5 万。

我又加了 5 万。

2024 年,S\&P 500 超过 5,000 点。

\textbf{那 15 万,变成了 60 多万。}

\begin{center}\rule{0.5\linewidth}{0.5pt}\end{center}

\subsection{规则}\label{ux89c4ux5219-3}

\subsubsection{规则
1:别人恐惧时贪婪,别人贪婪时恐惧}\label{ux89c4ux5219-1ux522bux4ebaux6050ux60e7ux65f6ux8d2aux5a6aux522bux4ebaux8d2aux5a6aux65f6ux6050ux60e7}

这是巴菲特最有名的话。

意思是:

\begin{itemize}
\tightlist
\item
  \textbf{大家都在卖} → 你应该考虑买
\item
  \textbf{大家都在买} → 你应该考虑卖
\end{itemize}

\subsubsection{规则
2:但首先,你要有现金}\label{ux89c4ux5219-2ux4f46ux9996ux5148ux4f60ux8981ux6709ux73b0ux91d1}

Sarah
说:``\textbf{别人恐惧时贪婪。但首先你要有贪婪的资本------现金。}''

如果你没有现金,股市跌的时候你只能看着。

所以永远保持 10-20\% 的现金。

\subsubsection{规则
3:恐惧的信号}\label{ux89c4ux5219-3ux6050ux60e7ux7684ux4fe1ux53f7}

当你看到这些,说明市场很恐惧:

{\def\LTcaptype{none} % do not increment counter
\begin{longtable}[]{@{}ll@{}}
\toprule\noalign{}
信号 & 例子 \\
\midrule\noalign{}
\endhead
\bottomrule\noalign{}
\endlastfoot
新闻头条全是坏消息 & ``股市崩盘!'' \\
你的朋友都在卖股票 & ``我全清仓了'' \\
专家说''这次不一样'' & ``经济要完蛋了'' \\
VIX 恐慌指数 \textgreater{} 40 & 极度恐慌 \\
\end{longtable}
}

\begin{center}\rule{0.5\linewidth}{0.5pt}\end{center}

\subsection{怎么做}\label{ux600eux4e48ux505a-1}

\subsubsection{简单的方法}\label{ux7b80ux5355ux7684ux65b9ux6cd5}

不用猜什么时候恐惧、什么时候贪婪。

用\textbf{定投}:

\begin{itemize}
\tightlist
\item
  每个月固定买入一定金额
\item
  不管市场涨还是跌
\item
  自动做到''跌多买多,涨多买少''
\end{itemize}

\subsubsection{进阶方法}\label{ux8fdbux9636ux65b9ux6cd5}

当市场大跌(\textgreater{} 20\%)时:

\begin{enumerate}
\def\labelenumi{\arabic{enumi}.}
\tightlist
\item
  检查:这是暂时的恐慌,还是公司/经济真的出问题了?
\item
  如果是暂时的恐慌:用你的现金储备加仓
\item
  不要一次 all in,分批买入
\end{enumerate}

\begin{center}\rule{0.5\linewidth}{0.5pt}\end{center}

\subsection{真实案例}\label{ux771fux5b9eux6848ux4f8b}

{\def\LTcaptype{none} % do not increment counter
\begin{longtable}[]{@{}lll@{}}
\toprule\noalign{}
恐惧时刻 & 当时 S\&P 500 & 5 年后 \\
\midrule\noalign{}
\endhead
\bottomrule\noalign{}
\endlastfoot
2008 金融危机 & 666 & 2,100 (+215\%) \\
2020 新冠暴跌 & 2,200 & 4,800 (+118\%) \\
\end{longtable}
}

\textbf{每一次''世界要完蛋了'',最后都没有完蛋。}

\begin{center}\rule{0.5\linewidth}{0.5pt}\end{center}

\subsection{一句话}\label{ux4e00ux53e5ux8bdd-3}

\begin{quote}
\textbf{恐惧是打折的信号,但你要有钱才能买。}
\end{quote}

\begin{center}\rule{0.5\linewidth}{0.5pt}\end{center}

\subsection{检查清单}\label{ux68c0ux67e5ux6e05ux5355-3}

\begin{itemize}
\tightlist
\item[$\square$]
  我手上有 10-20\% 的现金
\item[$\square$]
  我不会因为新闻头条恐慌而卖出
\item[$\square$]
  我有一个定投计划
\item[$\square$]
  当市场大跌时,我会考虑加仓,而不是清仓
\end{itemize}

\begin{center}\rule{0.5\linewidth}{0.5pt}\end{center}

\textbf{下一章}:\href{05-index-funds.md}{不知道买什么就买指数}

\newpage

\section{第五章:不知道买什么就买指数}\label{ux7b2cux4e94ux7ae0ux4e0dux77e5ux9053ux4e70ux4ec0ux4e48ux5c31ux4e70ux6307ux6570}

\begin{quote}
``Sarah 说:当你不知道该买什么的时候,就买整个市场。''
\end{quote}

\begin{center}\rule{0.5\linewidth}{0.5pt}\end{center}

\subsection{故事}\label{ux6545ux4e8b-4}

2008 年,Sarah 给我建议:

``买指数基金。别选股,就买 S\&P 500。''

``为什么不选个股?''

``\textbf{因为你不是巴菲特。99\% 的人也不是。}''

``可是\ldots{}''

``听着,职业基金经理有整个团队做研究,95\%
的人长期跑不过指数。你一个人,能比他们强?''

我买了 Vanguard S\&P 500 指数基金(VOO)。

2024 年,Sarah 退休了。

``你真的在 Google 干了 16 年?''

``是啊。我的投资策略很无聊------\textbf{买指数,持有,不动。}''

``就这样?''

``\textbf{就这样。无聊才能赚钱。}''

\begin{center}\rule{0.5\linewidth}{0.5pt}\end{center}

\subsection{规则}\label{ux89c4ux5219-4}

\subsubsection{规则 1:指数基金 \textgreater{}
个股}\label{ux89c4ux5219-1ux6307ux6570ux57faux91d1-ux4e2aux80a1}

{\def\LTcaptype{none} % do not increment counter
\begin{longtable}[]{@{}ll@{}}
\toprule\noalign{}
个股 & 指数基金 \\
\midrule\noalign{}
\endhead
\bottomrule\noalign{}
\endlastfoot
要选对公司 & 买一篮子公司 \\
可能归零 & 很难归零 \\
需要研究 & 不需要研究 \\
95\% 的人跑输市场 & 保证拿到市场平均回报 \\
\end{longtable}
}

\subsubsection{规则
2:费用很重要}\label{ux89c4ux5219-2ux8d39ux7528ux5f88ux91cdux8981}

指数基金的费用差别很大:

{\def\LTcaptype{none} % do not increment counter
\begin{longtable}[]{@{}lll@{}}
\toprule\noalign{}
类型 & 年费率 & 10 万美元,30 年后损失 \\
\midrule\noalign{}
\endhead
\bottomrule\noalign{}
\endlastfoot
便宜的指数基金 & 0.03\% & \$900 \\
贵的主动基金 & 1.5\% & \$45,000 \\
\end{longtable}
}

\textbf{差了 5 万美元。}

所以选低费率的: - Vanguard VOO: 0.03\% - Fidelity FXAIX: 0.015\% -
iShares IVV: 0.03\%

\subsubsection{规则
3:不要过度分散}\label{ux89c4ux5219-3ux4e0dux8981ux8fc7ux5ea6ux5206ux6563}

你只需要 2-4 个基金:

\begin{verbatim}
最简单版(1 个基金):
- 100%:VT(全球股票指数)

简单版(2 个基金):
- 80%:VOO(美股指数)
- 20%:BND(美债指数)

进阶版(3 个基金):
- 60%:VOO(美股指数)
- 20%:VXUS(国际股票)
- 20%:BND(美债指数)
\end{verbatim}

\begin{center}\rule{0.5\linewidth}{0.5pt}\end{center}

\subsection{常见问题}\label{ux5e38ux89c1ux95eeux9898}

\subsubsection{Q:我应该买哪个指数基金?}\label{qux6211ux5e94ux8be5ux4e70ux54eaux4e2aux6307ux6570ux57faux91d1}

A:如果只买一个,买 \textbf{VOO}(或 SPY、IVV)。

\subsubsection{Q:什么时候买?}\label{qux4ec0ux4e48ux65f6ux5019ux4e70}

A:现在。然后每个月继续买。不要等。

\subsubsection{Q:什么时候卖?}\label{qux4ec0ux4e48ux65f6ux5019ux5356}

A:退休的时候。或者需要用钱的时候。不要因为涨了或跌了就卖。

\subsubsection{Q:这样赚得少吧?}\label{qux8fd9ux6837ux8d5aux5f97ux5c11ux5427}

A:S\&P 500 过去 30 年平均回报 \textasciitilde10\%/年。 10 万美元,30
年后变成 170 万。 足够了。

\begin{center}\rule{0.5\linewidth}{0.5pt}\end{center}

\subsection{一句话}\label{ux4e00ux53e5ux8bdd-4}

\begin{quote}
\textbf{不要试图打败市场,做市场。}
\end{quote}

\begin{center}\rule{0.5\linewidth}{0.5pt}\end{center}

\subsection{检查清单}\label{ux68c0ux67e5ux6e05ux5355-4}

\begin{itemize}
\tightlist
\item[$\square$]
  我有一个低费率的指数基金(费率 \textless{} 0.1\%)
\item[$\square$]
  我没有花时间选个股
\item[$\square$]
  我的投资组合不超过 4 个基金
\item[$\square$]
  我设置了自动定投
\end{itemize}

\begin{center}\rule{0.5\linewidth}{0.5pt}\end{center}

\textbf{下一章}:\href{06-dollar-cost-averaging.md}{分批买入不择时}

\newpage

\section{第六章:分批买入不择时}\label{ux7b2cux516dux7ae0ux5206ux6279ux4e70ux5165ux4e0dux62e9ux65f6}

\begin{quote}
``妈妈说:慢慢来,才能快。''
\end{quote}

\begin{center}\rule{0.5\linewidth}{0.5pt}\end{center}

\subsection{故事}\label{ux6545ux4e8b-5}

2000 年,我刚到硅谷。

机场大巴司机说他刚卖掉房子,全部买了网络股。

``Pets.com 要上天了!''

他择时了。他选在了最高点。

那年 4 月,纳斯达克开始跌。跌了 80\%。

司机亏光了。

\begin{center}\rule{0.5\linewidth}{0.5pt}\end{center}

如果他换一种方法呢?

不是一次 all in,而是每个月买一点:

{\def\LTcaptype{none} % do not increment counter
\begin{longtable}[]{@{}llll@{}}
\toprule\noalign{}
月份 & 纳斯达克 & 买入金额 & 买到份额 \\
\midrule\noalign{}
\endhead
\bottomrule\noalign{}
\endlastfoot
1 月 & 5,000 & \$1,000 & 0.20 \\
4 月 & 4,000 & \$1,000 & 0.25 \\
7 月 & 3,000 & \$1,000 & 0.33 \\
10 月 & 2,000 & \$1,000 & 0.50 \\
\end{longtable}
}

\textbf{跌得越多,同样的钱买得越多。}

这叫\textbf{定投}(Dollar Cost Averaging)。

\begin{center}\rule{0.5\linewidth}{0.5pt}\end{center}

\subsection{规则}\label{ux89c4ux5219-5}

\subsubsection{规则
1:不要试图择时}\label{ux89c4ux5219-1ux4e0dux8981ux8bd5ux56feux62e9ux65f6}

没有人知道明天市场涨还是跌。

专家也不知道。 基金经理也不知道。 你也不知道。

所以不要猜。

\subsubsection{规则
2:定期定额}\label{ux89c4ux5219-2ux5b9aux671fux5b9aux989d}

每个月固定日期,固定金额,自动买入。

\begin{itemize}
\tightlist
\item
  发工资那天买
\item
  不管市场涨还是跌
\item
  不看新闻
\item
  不思考
\end{itemize}

\textbf{把决策权交给系统,不要交给情绪。}

\subsubsection{规则 3:一次性 vs
定投}\label{ux89c4ux5219-3ux4e00ux6b21ux6027-vs-ux5b9aux6295}

如果你有一大笔钱(比如 10 万美元):

{\def\LTcaptype{none} % do not increment counter
\begin{longtable}[]{@{}lll@{}}
\toprule\noalign{}
方法 & 优点 & 缺点 \\
\midrule\noalign{}
\endhead
\bottomrule\noalign{}
\endlastfoot
一次性买入 & 统计上回报更高 & 可能买在高点 \\
12 个月定投 & 心理压力小 & 可能错过上涨 \\
\end{longtable}
}

\textbf{新手建议:12 个月定投。}

等你习惯了市场波动,再考虑一次性买入。

\begin{center}\rule{0.5\linewidth}{0.5pt}\end{center}

\subsection{怎么设置自动定投}\label{ux600eux4e48ux8bbeux7f6eux81eaux52a8ux5b9aux6295}

\subsubsection{美国券商}\label{ux7f8eux56fdux5238ux5546}

\begin{enumerate}
\def\labelenumi{\arabic{enumi}.}
\tightlist
\item
  登录 Fidelity / Schwab / Vanguard
\item
  找到 ``Automatic Investments''
\item
  设置:每月 X 日,买入 \$X 的 VOO
\item
  连接银行账户
\item
  完成
\end{enumerate}

\subsubsection{建议金额}\label{ux5efaux8baeux91d1ux989d}

{\def\LTcaptype{none} % do not increment counter
\begin{longtable}[]{@{}ll@{}}
\toprule\noalign{}
年收入 & 建议月定投 \\
\midrule\noalign{}
\endhead
\bottomrule\noalign{}
\endlastfoot
\$50,000 & \$500-750 \\
\$100,000 & \$1,000-1,500 \\
\$150,000 & \$1,500-2,500 \\
\$200,000+ & \$2,500-4,000 \\
\end{longtable}
}

目标:每年投入年收入的 \textbf{15-20\%}。

\begin{center}\rule{0.5\linewidth}{0.5pt}\end{center}

\subsection{一句话}\label{ux4e00ux53e5ux8bdd-5}

\begin{quote}
\textbf{不要问什么时候买,问每个月买多少。}
\end{quote}

\begin{center}\rule{0.5\linewidth}{0.5pt}\end{center}

\subsection{检查清单}\label{ux68c0ux67e5ux6e05ux5355-5}

\begin{itemize}
\tightlist
\item[$\square$]
  我设置了自动定投
\item[$\square$]
  每个月发工资后自动买入
\item[$\square$]
  我不会因为市场涨跌改变计划
\item[$\square$]
  我不看财经新闻做投资决策
\end{itemize}

\begin{center}\rule{0.5\linewidth}{0.5pt}\end{center}

\textbf{下一章}:\href{07-hold-through.md}{持有是最难的事}

\newpage

\section{第七章:持有是最难的事}\label{ux7b2cux4e03ux7ae0ux6301ux6709ux662fux6700ux96beux7684ux4e8b}

\begin{quote}
``Sarah 说:投资最难的不是买入,是持有。''
\end{quote}

\begin{center}\rule{0.5\linewidth}{0.5pt}\end{center}

\subsection{故事}\label{ux6545ux4e8b-6}

2009 年 3 月,股市跌到谷底。

我那 10 万块的指数基金,缩水成了 5.5 万。

我打电话给 Sarah。

``我是不是应该卖了?''

``你疯了?\textbf{现在是加仓的时候。}''

``可是我已经亏了 45\%。''

``那只是纸面亏损。你看看那些公司------Apple、Google、Microsoft------他们的业务消失了吗?''

``没有。''

``那就别卖。''

我没有卖。

2024 年,那笔投资涨了 4 倍多。

\textbf{如果我当时卖了,我会永远锁定 45\% 的亏损。}

\begin{center}\rule{0.5\linewidth}{0.5pt}\end{center}

\subsection{规则}\label{ux89c4ux5219-6}

\subsubsection{规则
1:区分纸面亏损和真实亏损}\label{ux89c4ux5219-1ux533aux5206ux7eb8ux9762ux4e8fux635fux548cux771fux5b9eux4e8fux635f}

{\def\LTcaptype{none} % do not increment counter
\begin{longtable}[]{@{}ll@{}}
\toprule\noalign{}
类型 & 意思 \\
\midrule\noalign{}
\endhead
\bottomrule\noalign{}
\endlastfoot
纸面亏损 & 账户显示红色,但你没卖 \\
真实亏损 & 你卖了,钱真的少了 \\
\end{longtable}
}

\textbf{只要你不卖,就只是数字波动。}

\subsubsection{规则
2:看公司,不看股价}\label{ux89c4ux5219-2ux770bux516cux53f8ux4e0dux770bux80a1ux4ef7}

股价跌了,问自己:

\begin{itemize}
\tightlist
\item
  公司还在赚钱吗?
\item
  产品还有人买吗?
\item
  长期竞争力还在吗?
\end{itemize}

如果答案是''是''------\textbf{那只是市场情绪波动,不是真的出问题了。}

\subsubsection{规则
3:时间是你的朋友}\label{ux89c4ux5219-3ux65f6ux95f4ux662fux4f60ux7684ux670bux53cb}

{\def\LTcaptype{none} % do not increment counter
\begin{longtable}[]{@{}ll@{}}
\toprule\noalign{}
持有时间 & 美股亏损概率 \\
\midrule\noalign{}
\endhead
\bottomrule\noalign{}
\endlastfoot
1 天 & \textasciitilde45\% \\
1 年 & \textasciitilde30\% \\
10 年 & \textasciitilde5\% \\
20 年 & \textasciitilde0\% \\
\end{longtable}
}

\textbf{持有得越久,亏损的可能性越低。}

\begin{center}\rule{0.5\linewidth}{0.5pt}\end{center}

\subsection{持有期间会发生的事}\label{ux6301ux6709ux671fux95f4ux4f1aux53d1ux751fux7684ux4e8b}

在你持有的 20 年里,你会经历:

\begin{itemize}
\tightlist
\item
  3-5 次 20\% 以上的下跌
\item
  1-2 次 40\% 以上的暴跌
\item
  无数次''这次不一样''的恐慌
\item
  无数次想卖掉的冲动
\end{itemize}

\textbf{这些都是正常的。}

S\&P 500 历史上每次暴跌后都创了新高。每一次。

\begin{center}\rule{0.5\linewidth}{0.5pt}\end{center}

\subsection{怎么做到}\label{ux600eux4e48ux505aux5230}

\subsubsection{方法
1:不看账户}\label{ux65b9ux6cd5-1ux4e0dux770bux8d26ux6237}

设一个规则:每月只看一次账户。

或者更狠:每年只看一次。

\textbf{看得越少,做错事的机会越少。}

\subsubsection{方法
2:写下你的理由}\label{ux65b9ux6cd5-2ux5199ux4e0bux4f60ux7684ux7406ux7531}

买入的时候,写下:

\begin{quote}
``我买 VOO 是因为我相信美国经济长期会增长。 我计划持有 20 年。 中途下跌
50\% 我也不会卖。''
\end{quote}

跌的时候,拿出来读一读。

\subsubsection{方法 3:自动化}\label{ux65b9ux6cd5-3ux81eaux52a8ux5316}

设置自动定投,然后: - 删掉股票 App - 取消财经新闻订阅 -
不要和朋友讨论股票

\textbf{无聊是正确的感觉。}

\begin{center}\rule{0.5\linewidth}{0.5pt}\end{center}

\subsection{一句话}\label{ux4e00ux53e5ux8bdd-6}

\begin{quote}
\textbf{会买的是徒弟,会持有的才是师傅。}
\end{quote}

\begin{center}\rule{0.5\linewidth}{0.5pt}\end{center}

\subsection{检查清单}\label{ux68c0ux67e5ux6e05ux5355-6}

\begin{itemize}
\tightlist
\item[$\square$]
  我知道我持有的是好公司/好指数
\item[$\square$]
  我有明确的持有期限(至少 5 年)
\item[$\square$]
  我不会因为下跌 20-50\% 而卖出
\item[$\square$]
  我减少了查看账户的频率
\end{itemize}

\begin{center}\rule{0.5\linewidth}{0.5pt}\end{center}

\textbf{下一章}:\href{08-when-to-sell.md}{什么时候该走}

\newpage

\section{第八章:什么时候该走}\label{ux7b2cux516bux7ae0ux4ec0ux4e48ux65f6ux5019ux8be5ux8d70}

\begin{quote}
``Sarah 说:后悔是投资者最大的敌人。做了决定就不要回头看。''
\end{quote}

\begin{center}\rule{0.5\linewidth}{0.5pt}\end{center}

\subsection{故事}\label{ux6545ux4e8b-7}

2004 年,Sarah 离开我们的小公司,去了 Google。

她给了我一个机会:``Jim,你要不要一起来?''

我没去。

Google 上市后,Sarah 的期权值了几百万。

那天晚上,我给 Sarah 打电话。

``我后悔了。''

``\textbf{后悔是投资者最大的敌人。做了决定就不要回头看。}''

``这是谁说的?''

``我爸。他炒股亏了很多钱之后说的。''

\begin{center}\rule{0.5\linewidth}{0.5pt}\end{center}

\subsection{规则}\label{ux89c4ux5219-7}

\subsubsection{规则
1:该卖的三种情况}\label{ux89c4ux5219-1ux8be5ux5356ux7684ux4e09ux79cdux60c5ux51b5}

{\def\LTcaptype{none} % do not increment counter
\begin{longtable}[]{@{}ll@{}}
\toprule\noalign{}
情况 & 例子 \\
\midrule\noalign{}
\endhead
\bottomrule\noalign{}
\endlastfoot
\textbf{需要用钱} & 买房、退休、紧急情况 \\
\textbf{投资逻辑变了} & 公司基本面恶化、行业消失 \\
\textbf{有更好的选择} & 机会成本更高 \\
\end{longtable}
}

\subsubsection{规则
2:不该卖的情况}\label{ux89c4ux5219-2ux4e0dux8be5ux5356ux7684ux60c5ux51b5}

{\def\LTcaptype{none} % do not increment counter
\begin{longtable}[]{@{}ll@{}}
\toprule\noalign{}
情况 & 为什么不该卖 \\
\midrule\noalign{}
\endhead
\bottomrule\noalign{}
\endlastfoot
股价跌了 & 可能只是情绪波动 \\
新闻说要崩盘 & 新闻通常滞后 \\
涨了很多想落袋为安 & 你会错过更多上涨 \\
朋友都在卖 & 群体不一定对 \\
\end{longtable}
}

\subsubsection{规则
3:做了决定不要回头看}\label{ux89c4ux5219-3ux505aux4e86ux51b3ux5b9aux4e0dux8981ux56deux5934ux770b}

卖了之后: - 不要看它后来涨了多少 - 不要计算''如果我没卖会怎样'' -
不要后悔

\textbf{钱已经不是你的了。向前看。}

\begin{center}\rule{0.5\linewidth}{0.5pt}\end{center}

\subsection{卖出检查清单}\label{ux5356ux51faux68c0ux67e5ux6e05ux5355}

在卖出之前,问自己:

\begin{enumerate}
\def\labelenumi{\arabic{enumi}.}
\item
  \textbf{我为什么买入?} 那个理由还成立吗?
\item
  \textbf{这是情绪还是理性?} 是因为恐惧/贪婪,还是真的有逻辑?
\item
  \textbf{卖出后钱去哪里?} 有更好的地方放吗?
\item
  \textbf{税的影响?} 短期持有税更高(美国:短期 = 收入税,长期 =
  资本利得税 15-20\%)
\item
  \textbf{我会后悔吗?} 如果它之后涨了 50\%,我能接受吗?
\end{enumerate}

\begin{center}\rule{0.5\linewidth}{0.5pt}\end{center}

\subsection{指数基金卖出时机}\label{ux6307ux6570ux57faux91d1ux5356ux51faux65f6ux673a}

如果你买的是指数基金,卖出时机很简单:

\begin{verbatim}
年轻时:不卖
中年时:不卖
退休时:每年卖 4%(4% rule)
\end{verbatim}

\textbf{就这么简单。}

\begin{center}\rule{0.5\linewidth}{0.5pt}\end{center}

\subsection{一句话}\label{ux4e00ux53e5ux8bdd-7}

\begin{quote}
\textbf{卖出的理由只有三个:需要用钱、逻辑变了、有更好选择。其他都是借口。}
\end{quote}

\begin{center}\rule{0.5\linewidth}{0.5pt}\end{center}

\subsection{检查清单}\label{ux68c0ux67e5ux6e05ux5355-7}

\begin{itemize}
\tightlist
\item[$\square$]
  我有明确的卖出理由(三个之一)
\item[$\square$]
  我不是因为情绪在卖
\item[$\square$]
  我考虑了税的影响
\item[$\square$]
  我做好了不后悔的准备
\end{itemize}

\begin{center}\rule{0.5\linewidth}{0.5pt}\end{center}

\textbf{下一章}:\href{09-no-chasing.md}{不要追涨杀跌}

\newpage

\section{第九章:不要追涨杀跌}\label{ux7b2cux4e5dux7ae0ux4e0dux8981ux8ffdux6da8ux6740ux8dcc}

\begin{quote}
``Sarah 说:你总是在高点买,在低点卖。这叫追涨杀跌。''
\end{quote}

\begin{center}\rule{0.5\linewidth}{0.5pt}\end{center}

\subsection{故事}\label{ux6545ux4e8b-8}

Mike 的投资历史:

{\def\LTcaptype{none} % do not increment counter
\begin{longtable}[]{@{}lll@{}}
\toprule\noalign{}
年份 & 操作 & 结果 \\
\midrule\noalign{}
\endhead
\bottomrule\noalign{}
\endlastfoot
2000 & 高点买网络股 & 亏了 \\
2007 & 高点买曼哈顿房子 & 亏了 40\% \\
2021 & 儿子高点买 Zoom & 学费没了 \\
\end{longtable}
}

Mike 的模式:

\begin{enumerate}
\def\labelenumi{\arabic{enumi}.}
\tightlist
\item
  看到什么涨,就去买
\item
  看到什么跌,就卖掉
\item
  重复
\end{enumerate}

\textbf{这叫追涨杀跌。这是亏钱的最可靠方法。}

\begin{center}\rule{0.5\linewidth}{0.5pt}\end{center}

\subsection{为什么追涨杀跌会亏钱}\label{ux4e3aux4ec0ux4e48ux8ffdux6da8ux6740ux8dccux4f1aux4e8fux94b1}

人类的本能:

\begin{itemize}
\tightlist
\item
  看到涨 → 觉得会继续涨 → 买
\item
  看到跌 → 觉得会继续跌 → 卖
\end{itemize}

但市场的规律是:

\begin{itemize}
\tightlist
\item
  涨太多 → 会跌
\item
  跌太多 → 会涨
\end{itemize}

\textbf{你的本能和市场规律是反的。}

\begin{center}\rule{0.5\linewidth}{0.5pt}\end{center}

\subsection{规则}\label{ux89c4ux5219-8}

\subsubsection{规则
1:看到热门的,不买}\label{ux89c4ux5219-1ux770bux5230ux70edux95e8ux7684ux4e0dux4e70}

当你听到所有人都在讨论一只股票:

\begin{itemize}
\tightlist
\item
  2000年:Pets.com
\item
  2020年:Zoom
\item
  2021年:GameStop
\item
  2021年:Dogecoin
\item
  2024年:Nvidia(要小心)
\end{itemize}

\textbf{等到你听说的时候,通常已经晚了。}

\subsubsection{规则
2:看到暴跌的,不卖}\label{ux89c4ux5219-2ux770bux5230ux66b4ux8dccux7684ux4e0dux5356}

当新闻头条全是恐慌:

\begin{itemize}
\tightlist
\item
  2008年:金融危机
\item
  2020年:新冠暴跌
\item
  2022年:科技股暴跌
\end{itemize}

\textbf{这通常是机会,不是逃跑的时候。}

\subsubsection{规则
3:用系统代替情绪}\label{ux89c4ux5219-3ux7528ux7cfbux7edfux4ee3ux66ffux60c5ux7eea}

不要根据感觉做决定。用规则:

\begin{verbatim}
买入规则:
- 每月定投 $X
- 不管涨跌
- 自动执行

卖出规则:
- 只在需要用钱时卖
- 只在投资逻辑变化时卖
- 不因为涨跌而卖
\end{verbatim}

\begin{center}\rule{0.5\linewidth}{0.5pt}\end{center}

\subsection{自我检测:你是不是在追涨杀跌?}\label{ux81eaux6211ux68c0ux6d4bux4f60ux662fux4e0dux662fux5728ux8ffdux6da8ux6740ux8dcc}

问自己这些问题:

{\def\LTcaptype{none} % do not increment counter
\begin{longtable}[]{@{}ll@{}}
\toprule\noalign{}
问题 & 如果是,你可能在追涨杀跌 \\
\midrule\noalign{}
\endhead
\bottomrule\noalign{}
\endlastfoot
你是在看到涨之后才买的吗? & ✓ \\
你是在看到跌之后想卖的吗? & ✓ \\
你是因为朋友在买才买的吗? & ✓ \\
你是因为新闻说要崩盘才卖的吗? & ✓ \\
你是在涨了之后后悔没早买吗? & ✓ \\
你是在跌了之后后悔没早卖吗? & ✓ \\
\end{longtable}
}

\textbf{如果有 2 个以上是''是'',你需要改变策略。}

\begin{center}\rule{0.5\linewidth}{0.5pt}\end{center}

\subsection{怎么改}\label{ux600eux4e48ux6539}

\subsubsection{方法 1:定投}\label{ux65b9ux6cd5-1ux5b9aux6295}

每月固定金额,不管涨跌。

\textbf{把决策权从情绪手中拿走。}

\subsubsection{方法
2:不看新闻}\label{ux65b9ux6cd5-2ux4e0dux770bux65b0ux95fb}

财经新闻的目的是吸引眼球,不是帮你赚钱。

\textbf{删掉财经 App。}

\subsubsection{方法
3:写投资日记}\label{ux65b9ux6cd5-3ux5199ux6295ux8d44ux65e5ux8bb0}

每次买卖之前,写下:

\begin{itemize}
\tightlist
\item
  我为什么买/卖?
\item
  我的情绪是什么?
\item
  这是系统决策还是冲动?
\end{itemize}

\textbf{强迫自己思考。}

\begin{center}\rule{0.5\linewidth}{0.5pt}\end{center}

\subsection{一句话}\label{ux4e00ux53e5ux8bdd-8}

\begin{quote}
\textbf{市场奖励纪律,惩罚冲动。}
\end{quote}

\begin{center}\rule{0.5\linewidth}{0.5pt}\end{center}

\subsection{检查清单}\label{ux68c0ux67e5ux6e05ux5355-8}

\begin{itemize}
\tightlist
\item[$\square$]
  我不会因为''大家都在买''而买
\item[$\square$]
  我不会因为''大家都在卖''而卖
\item[$\square$]
  我有一个系统化的买入/卖出规则
\item[$\square$]
  我不看财经新闻做投资决策
\end{itemize}

\begin{center}\rule{0.5\linewidth}{0.5pt}\end{center}

\textbf{下一章}:\href{10-sock-puppet.md}{袜子木偶的教训}

\newpage

\section{第十章:袜子木偶的教训}\label{ux7b2cux5341ux7ae0ux889cux5b50ux6728ux5076ux7684ux6559ux8bad}

\begin{quote}
\textbf{2000 年互联网泡沫}
\end{quote}

\begin{center}\rule{0.5\linewidth}{0.5pt}\end{center}

\subsection{发生了什么}\label{ux53d1ux751fux4e86ux4ec0ux4e48}

2000 年,互联网公司遍地开花。

Pets.com 卖狗粮。它的吉祥物是一只袜子木偶。

公司上市,股价冲上天。

机场大巴司机卖掉房子买网络股。他说:``Pets.com 要上天了!''

2000 年 3 月,纳斯达克到了 5,000 点。

4 月,开始跌。

Pets.com 倒闭了。CEO
在最后一次全员大会上挥舞着袜子木偶:``同志们,我们精神上颠覆了袜子行业!''

纳斯达克跌了 80\%。

\textbf{很多人的积蓄归零。}

\begin{center}\rule{0.5\linewidth}{0.5pt}\end{center}

\subsection{为什么会发生}\label{ux4e3aux4ec0ux4e48ux4f1aux53d1ux751f}

\subsubsection{1. 故事太好了}\label{ux6545ux4e8bux592aux597dux4e86}

``互联网改变一切!''

这个故事是对的。互联网确实改变了一切。

但故事对 ≠ 股票能赚钱。

\textbf{对的故事 + 错的价格 = 亏钱}

\subsubsection{2.
没有人看基本面}\label{ux6ca1ux6709ux4ebaux770bux57faux672cux9762}

Pets.com 没有利润。甚至收入都很少。

但所有人都在买。

``谁在乎利润?这是未来!''

\subsubsection{3. FOMO(错过恐惧)}\label{fomoux9519ux8fc7ux6050ux60e7}

``隔壁老王赚了 100\%!''

``我不买就亏了!''

当所有人都在赚钱的时候,不赚钱感觉像亏钱。

\begin{center}\rule{0.5\linewidth}{0.5pt}\end{center}

\subsection{教训}\label{ux6559ux8bad}

\subsubsection{教训
1:故事不等于价值}\label{ux6559ux8bad-1ux6545ux4e8bux4e0dux7b49ux4e8eux4ef7ux503c}

下次你听到''这是未来''的时候,问自己:

\begin{itemize}
\tightlist
\item
  这家公司赚钱吗?
\item
  如果不赚钱,什么时候会赚钱?
\item
  现在的股价合理吗?
\end{itemize}

\subsubsection{教训
2:泡沫的特征}\label{ux6559ux8bad-2ux6ce1ux6cabux7684ux7279ux5f81}

当你看到这些,要警惕:

{\def\LTcaptype{none} % do not increment counter
\begin{longtable}[]{@{}lll@{}}
\toprule\noalign{}
信号 & 2000 年例子 & 永恒的形式 \\
\midrule\noalign{}
\endhead
\bottomrule\noalign{}
\endlastfoot
司机在讨论股票 & ✓ & 不懂的人开始投资 \\
公司没有利润但股价飞涨 & ✓ & 估值脱离现实 \\
``这次不一样'' & ✓ & 新范式叙事 \\
所有人都在赚钱 & ✓ & FOMO 蔓延 \\
涨得最快的是最烂的公司 & ✓ & 投机盛行 \\
\end{longtable}
}

\subsubsection{教训
3:泡沫破裂后,好公司还在}\label{ux6559ux8bad-3ux6ce1ux6cabux7834ux88c2ux540eux597dux516cux53f8ux8fd8ux5728}

2000 年泡沫破裂后:

\begin{itemize}
\tightlist
\item
  Pets.com → 倒闭
\item
  Amazon → 跌了 90\%,然后涨了 100 倍
\item
  Google → 2004 年上市,成为巨头
\end{itemize}

\textbf{泡沫杀死垃圾,但好公司会活下来。}

\begin{center}\rule{0.5\linewidth}{0.5pt}\end{center}

\subsection{怎么避免}\label{ux600eux4e48ux907fux514d}

\subsubsection{避免方法
1:不买没有利润的公司}\label{ux907fux514dux65b9ux6cd5-1ux4e0dux4e70ux6ca1ux6709ux5229ux6da6ux7684ux516cux53f8}

或者至少,不要重仓。

\subsubsection{避免方法
2:问''司机问题''}\label{ux907fux514dux65b9ux6cd5-2ux95eeux53f8ux673aux95eeux9898}

如果出租车司机、理发师、你不懂股票的亲戚都在讨论一只股票------

\textbf{跑。}

\subsubsection{避免方法
3:买指数}\label{ux907fux514dux65b9ux6cd5-3ux4e70ux6307ux6570}

指数里有好公司也有坏公司。

泡沫破裂后,坏公司被踢出去,好公司留下来。

\textbf{你不用自己选。}

\begin{center}\rule{0.5\linewidth}{0.5pt}\end{center}

\subsection{一句话}\label{ux4e00ux53e5ux8bdd-9}

\begin{quote}
\textbf{当袜子木偶成为明星的时候,泡沫就快破了。}
\end{quote}

\begin{center}\rule{0.5\linewidth}{0.5pt}\end{center}

\subsection{检查清单}\label{ux68c0ux67e5ux6e05ux5355-9}

\begin{itemize}
\tightlist
\item[$\square$]
  我不买没有利润的''概念股''
\item[$\square$]
  我不因为''所有人都在买''而买
\item[$\square$]
  我能区分好故事和好投资
\item[$\square$]
  我在泡沫中保持冷静
\end{itemize}

\begin{center}\rule{0.5\linewidth}{0.5pt}\end{center}

\textbf{下一章}:\href{11-enron.md}{安然式的创新}

\newpage

\section{第十一章:安然式的创新}\label{ux7b2cux5341ux4e00ux7ae0ux5b89ux7136ux5f0fux7684ux521bux65b0}

\begin{quote}
\textbf{2001 年安然破产}
\end{quote}

\begin{center}\rule{0.5\linewidth}{0.5pt}\end{center}

\subsection{发生了什么}\label{ux53d1ux751fux4e86ux4ec0ux4e48-1}

安然是一家能源公司。

它被评为''全美最创新的公司''------连续六年。

Mike 在那里工作。他把 signing bonus 5 万美元全买了公司股票。

``他们在做能源交易的革命!''

我问:``你们具体做什么?''

Mike 想了想:``我们\ldots{} 创造价值。''

``怎么创造?''

``通过复杂的金融工具。''

``什么工具?''

``太复杂了你不会懂的。''

2001 年 11 月,安然破产。

\textbf{原来''创新''的意思是''会计造假''。}

Mike 那 5 万块的股票,变成了 37 美分。

\begin{center}\rule{0.5\linewidth}{0.5pt}\end{center}

\subsection{为什么会发生}\label{ux4e3aux4ec0ux4e48ux4f1aux53d1ux751f-1}

\subsubsection{1. 财报太复杂}\label{ux8d22ux62a5ux592aux590dux6742}

Sarah 的爸爸是 CPA。他说:

\begin{quote}
``如果一个公司的财报需要天才才能看懂,要么他们在做诺贝尔奖级别的创新,要么他们在做牢狱级别的造假。''
\end{quote}

安然的财报没有人能看懂。

\subsubsection{2. 光环效应}\label{ux5149ux73afux6548ux5e94}

\begin{itemize}
\tightlist
\item
  连续六年''最创新公司''
\item
  CEO 是商业杂志的封面人物
\item
  所有人都说安然是好公司
\end{itemize}

\textbf{当所有人都说好的时候,没有人在认真检查。}

\subsubsection{3.
员工相信公司}\label{ux5458ux5de5ux76f8ux4fe1ux516cux53f8}

Mike 把钱全押在安然。

为什么?因为他相信公司。

\textbf{相信 ≠ 正确。信任 ≠ 事实。}

\begin{center}\rule{0.5\linewidth}{0.5pt}\end{center}

\subsection{教训}\label{ux6559ux8bad-1}

\subsubsection{教训 1:听不懂 =
危险}\label{ux6559ux8bad-1ux542cux4e0dux61c2-ux5371ux9669}

如果一个公司的商业模式你听不懂:

\begin{itemize}
\tightlist
\item
  要么它太复杂,风险太高
\item
  要么它在骗人
\end{itemize}

无论哪种,\textbf{不要投}。

\subsubsection{教训
2:奖项不能当真}\label{ux6559ux8bad-2ux5956ux9879ux4e0dux80fdux5f53ux771f}

``全美最创新''、``最受尊敬''、``最佳雇主''\ldots{}

这些奖项不能保护你的钱。

\textbf{看财务数据,不看奖杯。}

\subsubsection{教训
3:不要买公司股票}\label{ux6559ux8bad-3ux4e0dux8981ux4e70ux516cux53f8ux80a1ux7968}

你在公司工作,已经把人力资本押进去了。

如果再买公司股票,你在双重押注。

公司倒了 → 失业 + 股票归零

\textbf{这是 Mike 犯的错误。}

\begin{center}\rule{0.5\linewidth}{0.5pt}\end{center}

\subsection{造假的信号}\label{ux9020ux5047ux7684ux4fe1ux53f7}

当你看到这些,要警惕:

{\def\LTcaptype{none} % do not increment counter
\begin{longtable}[]{@{}ll@{}}
\toprule\noalign{}
信号 & 意思 \\
\midrule\noalign{}
\endhead
\bottomrule\noalign{}
\endlastfoot
财报特别复杂 & 可能在藏东西 \\
解释不清楚怎么赚钱 & 可能没在赚钱 \\
高管频繁卖股票 & 他们知道什么 \\
审计师换了 & 可能有问题 \\
利润增长但现金流没增长 & 利润可能是假的 \\
\end{longtable}
}

\begin{center}\rule{0.5\linewidth}{0.5pt}\end{center}

\subsection{怎么避免}\label{ux600eux4e48ux907fux514d-1}

\subsubsection{避免方法
1:三句话测试}\label{ux907fux514dux65b9ux6cd5-1ux4e09ux53e5ux8bddux6d4bux8bd5}

如果你不能用三句话解释一家公司怎么赚钱,不要买它的股票。

\subsubsection{避免方法
2:看现金流}\label{ux907fux514dux65b9ux6cd5-2ux770bux73b0ux91d1ux6d41}

利润可以造假。现金流很难造假。

公司赚钱了,钱在哪里?在银行账户里吗?

\subsubsection{避免方法
3:分散投资}\label{ux907fux514dux65b9ux6cd5-3ux5206ux6563ux6295ux8d44}

就算你被骗了,如果只占投资组合的 5\%,你还能活。

\begin{center}\rule{0.5\linewidth}{0.5pt}\end{center}

\subsection{一句话}\label{ux4e00ux53e5ux8bdd-10}

\begin{quote}
\textbf{``创新''和''欺诈''有时候只差一份财报。}
\end{quote}

\begin{center}\rule{0.5\linewidth}{0.5pt}\end{center}

\subsection{检查清单}\label{ux68c0ux67e5ux6e05ux5355-10}

\begin{itemize}
\tightlist
\item[$\square$]
  我能用三句话解释公司怎么赚钱
\item[$\square$]
  我看了现金流,不只是利润
\item[$\square$]
  我没有重仓任何一只股票
\item[$\square$]
  我没有买自己公司的股票
\end{itemize}

\begin{center}\rule{0.5\linewidth}{0.5pt}\end{center}

\textbf{下一章}:\href{12-lehman.md}{大到不能倒也会倒}

\newpage

\section{第十二章:大到不能倒也会倒}\label{ux7b2cux5341ux4e8cux7ae0ux5927ux5230ux4e0dux80fdux5012ux4e5fux4f1aux5012}

\begin{quote}
\textbf{2008 年雷曼兄弟破产}
\end{quote}

\begin{center}\rule{0.5\linewidth}{0.5pt}\end{center}

\subsection{发生了什么}\label{ux53d1ux751fux4e86ux4ec0ux4e48-2}

2008 年 9 月 14 日,周日晚上。

Mike 打电话给我:

``Jim,Bear Stearns 已经被 JPMorgan 收购了。现在我在 Lehman Brothers。''

``挺好的啊。''

``明天可能就没了。''

``什么意思?''

``我们找不到买家。政府不救我们。''

周一,Lehman Brothers 宣布破产。

158 年历史。一夜之间倒闭。

\textbf{全球金融危机正式开始。}

\begin{center}\rule{0.5\linewidth}{0.5pt}\end{center}

\subsection{背景}\label{ux80ccux666f}

\subsubsection{什么是次贷危机}\label{ux4ec0ux4e48ux662fux6b21ux8d37ux5371ux673a}

\begin{enumerate}
\def\labelenumi{\arabic{enumi}.}
\tightlist
\item
  银行给没有还款能力的人贷款买房
\item
  把这些贷款打包成''金融产品''卖给投资人
\item
  房价涨的时候,所有人都赚钱
\item
  房价跌了\ldots{} 所有人都完蛋了
\end{enumerate}

\subsubsection{为什么雷曼会倒}\label{ux4e3aux4ec0ux4e48ux96f7ux66fcux4f1aux5012}

雷曼持有大量次贷相关产品。

房价跌了,这些产品变成废纸。

政府选择不救雷曼。

\textbf{158 年的百年老店,48 小时内消失。}

\begin{center}\rule{0.5\linewidth}{0.5pt}\end{center}

\subsection{Mike 的故事}\label{mike-ux7684ux6545ux4e8b}

Mike 安然之后去了雷曼。

``华尔街才是真正的金融!百年老店!''

他用所有 bonus 买了曼哈顿的房子。

``房价永远涨,这是纽约!''

2008 年,他失业了。房子亏了 40\%。

他又出现在我的门口。

Sarah 说:``Mike,你又犯了同样的错误。''

``我知道。又把所有钱押在一个地方了。''

``不只是这个。\textbf{你总是在高点买,在低点卖。这叫追涨杀跌。}''

\begin{center}\rule{0.5\linewidth}{0.5pt}\end{center}

\subsection{教训}\label{ux6559ux8bad-2}

\subsubsection{教训
1:大到不能倒也会倒}\label{ux6559ux8bad-1ux5927ux5230ux4e0dux80fdux5012ux4e5fux4f1aux5012}

{\def\LTcaptype{none} % do not increment counter
\begin{longtable}[]{@{}ll@{}}
\toprule\noalign{}
``不会倒''的东西 & 结果 \\
\midrule\noalign{}
\endhead
\bottomrule\noalign{}
\endlastfoot
安然 & 倒了 \\
雷曼兄弟 & 倒了 \\
硅谷银行 & 倒了 \\
\end{longtable}
}

\textbf{没有什么是绝对安全的。}

\subsubsection{教训
2:政府不一定救你}\label{ux6559ux8bad-2ux653fux5e9cux4e0dux4e00ux5b9aux6551ux4f60}

雷曼之后,政府救了 AIG、救了银行。

但雷曼本身没被救。

\textbf{不要把命运押在''政府会救''上。}

\subsubsection{教训
3:危机是机会}\label{ux6559ux8bad-3ux5371ux673aux662fux673aux4f1a}

2008 年,Sarah 让我买指数基金。

``现在是买股票的时候。''

``你疯了?世界要完蛋了!''

``\textbf{世界不会完蛋。恐惧是最好的买入信号。}''

我买了。15 年后,涨了 4 倍多。

\begin{center}\rule{0.5\linewidth}{0.5pt}\end{center}

\subsection{危机中做什么}\label{ux5371ux673aux4e2dux505aux4ec0ux4e48}

\subsubsection{如果你有钱}\label{ux5982ux679cux4f60ux6709ux94b1}

\begin{enumerate}
\def\labelenumi{\arabic{enumi}.}
\tightlist
\item
  \textbf{不要恐慌卖出} --- 锁定亏损是最蠢的事
\item
  \textbf{考虑加仓} --- 如果你相信经济会复苏
\item
  \textbf{分批买入} --- 不要一次 all in
\end{enumerate}

\subsubsection{如果你没钱}\label{ux5982ux679cux4f60ux6ca1ux94b1}

\begin{enumerate}
\def\labelenumi{\arabic{enumi}.}
\tightlist
\item
  \textbf{保护工作} --- 现金流比投资回报重要
\item
  \textbf{不要借钱投资} --- 这是赌博
\item
  \textbf{等待} --- 下一次机会会来
\end{enumerate}

\begin{center}\rule{0.5\linewidth}{0.5pt}\end{center}

\subsection{一句话}\label{ux4e00ux53e5ux8bdd-11}

\begin{quote}
\textbf{大到不能倒只是意味着倒得更慢,不是不会倒。}
\end{quote}

\begin{center}\rule{0.5\linewidth}{0.5pt}\end{center}

\subsection{检查清单}\label{ux68c0ux67e5ux6e05ux5355-11}

\begin{itemize}
\tightlist
\item[$\square$]
  我不相信任何''绝对安全''的投资
\item[$\square$]
  我分散了投资,不把所有钱放一个地方
\item[$\square$]
  我有现金储备,可以在危机中加仓
\item[$\square$]
  我不会在恐慌中卖出
\end{itemize}

\begin{center}\rule{0.5\linewidth}{0.5pt}\end{center}

\textbf{下一章}:\href{13-meme-stocks.md}{表情包不是投资}

\newpage

\section{第十三章:表情包不是投资}\label{ux7b2cux5341ux4e09ux7ae0ux8868ux60c5ux5305ux4e0dux662fux6295ux8d44}

\begin{quote}
\textbf{2021 年 GameStop 和加密货币狂潮}
\end{quote}

\begin{center}\rule{0.5\linewidth}{0.5pt}\end{center}

\subsection{发生了什么}\label{ux53d1ux751fux4e86ux4ec0ux4e48-3}

\subsubsection{GameStop (GME)}\label{gamestop-gme}

2021 年,Reddit 上的散户决定联合起来。

目标:让做空 GameStop 的华尔街空头爆仓。

口号:\textbf{``To the moon!🚀''}

GameStop 从 \$20 涨到 \$483。一周之内。

Mike 的儿子 Tommy 把大学学费买了 GME。

他赚了钱。然后亏了。然后又赚了。然后全亏了。

``至少,'' Tommy 说,``我们发了一条信息。''

``什么信息?''

``呃\ldots{} 散户的力量?''

\subsubsection{加密货币}\label{ux52a0ux5bc6ux8d27ux5e01}

同一年,Elon Musk 在推特上发了一只狗。

Dogecoin 从 \$0.01 涨到 \$0.70。

Tommy 说:``Jim 叔叔,你必须买 Dogecoin。''

``为什么?''

``因为 Elon Musk 发了一只狗!''

我没买。

Dogecoin 现在 \$0.08。

\begin{center}\rule{0.5\linewidth}{0.5pt}\end{center}

\subsection{为什么会发生}\label{ux4e3aux4ec0ux4e48ux4f1aux53d1ux751f-2}

\subsubsection{1. 疫情无聊}\label{ux75abux60c5ux65e0ux804a}

2020-2021 年,大家都在家里。

没事干,就炒股。

\subsubsection{2. 免费交易}\label{ux514dux8d39ux4ea4ux6613}

Robinhood 让交易变成游戏。

没有手续费。点一下就买入。

\textbf{太容易了,所以太随便了。}

\subsubsection{3. 社交媒体}\label{ux793eux4ea4ux5a92ux4f53}

Reddit、Twitter、TikTok\ldots{}

``有人在 r/wallstreetbets 说要买这个!''

\textbf{投资决策变成了从众行为。}

\subsubsection{4. FOMO + 表情包}\label{fomo-ux8868ux60c5ux5305}

🚀 💎 🙌

\textbf{当投资用表情包表达的时候,那不是投资。}

\begin{center}\rule{0.5\linewidth}{0.5pt}\end{center}

\subsection{教训}\label{ux6559ux8bad-3}

\subsubsection{教训 1:运动 ≠
投资}\label{ux6559ux8bad-1ux8fd0ux52a8-ux6295ux8d44}

GME 散户想''打败华尔街''。

这是社会运动,不是投资策略。

\textbf{运动可能是对的,但你的钱可能是错的。}

\subsubsection{教训
2:表情包不是分析}\label{ux6559ux8bad-2ux8868ux60c5ux5305ux4e0dux662fux5206ux6790}

``🚀 To the moon'' 不是投资理由。

``💎🙌 Diamond hands'' 不是持有策略。

\textbf{真正的投资需要数字,不是表情。}

\subsubsection{教训
3:短期交易是赌博}\label{ux6559ux8bad-3ux77edux671fux4ea4ux6613ux662fux8d4cux535a}

{\def\LTcaptype{none} % do not increment counter
\begin{longtable}[]{@{}ll@{}}
\toprule\noalign{}
投资 & 赌博 \\
\midrule\noalign{}
\endhead
\bottomrule\noalign{}
\endlastfoot
基于价值 & 基于运气 \\
长期持有 & 短期波动 \\
分析决策 & 情绪决策 \\
大多数人能赢 & 大多数人会输 \\
\end{longtable}
}

GME、Dogecoin 这种短期炒作,大多数人亏钱。

\begin{center}\rule{0.5\linewidth}{0.5pt}\end{center}

\subsection{Sarah 的评价}\label{sarah-ux7684ux8bc4ux4ef7}

Sarah 说:``\textbf{任何用表情包定价的东西,都不是投资,是赌博。}''

她还说:``\textbf{因为他追的是热度,不是价值。永远在追,永远在接盘。}''

\begin{center}\rule{0.5\linewidth}{0.5pt}\end{center}

\subsection{怎么避免}\label{ux600eux4e48ux907fux514d-2}

\subsubsection{避免方法 1:不看 Reddit/Twitter
做投资}\label{ux907fux514dux65b9ux6cd5-1ux4e0dux770b-reddittwitter-ux505aux6295ux8d44}

社交媒体是娱乐,不是投资建议。

\subsubsection{避免方法
2:问自己''这是投资还是赌博''}\label{ux907fux514dux65b9ux6cd5-2ux95eeux81eaux5df1ux8fd9ux662fux6295ux8d44ux8fd8ux662fux8d4cux535a}

{\def\LTcaptype{none} % do not increment counter
\begin{longtable}[]{@{}ll@{}}
\toprule\noalign{}
问题 & 如果是,可能是赌博 \\
\midrule\noalign{}
\endhead
\bottomrule\noalign{}
\endlastfoot
我是因为表情包买的吗? & ✓ \\
我期望一周翻倍吗? & ✓ \\
我不知道这个公司做什么? & ✓ \\
我买是因为别人都在买? & ✓ \\
\end{longtable}
}

\subsubsection{避免方法
3:设一个''赌博账户''}\label{ux907fux514dux65b9ux6cd5-3ux8bbeux4e00ux4e2aux8d4cux535aux8d26ux6237}

如果你真的想玩,拿出投资组合的 5\% 当''赌博账户''。

亏光了也不影响生活。

\textbf{但剩下的 95\%,用来认真投资。}

\begin{center}\rule{0.5\linewidth}{0.5pt}\end{center}

\subsection{一句话}\label{ux4e00ux53e5ux8bdd-12}

\begin{quote}
\textbf{表情包能带来笑声,但不能带来回报。}
\end{quote}

\begin{center}\rule{0.5\linewidth}{0.5pt}\end{center}

\subsection{检查清单}\label{ux68c0ux67e5ux6e05ux5355-12}

\begin{itemize}
\tightlist
\item[$\square$]
  我不根据社交媒体热度买股票
\item[$\square$]
  我不期望一周翻倍
\item[$\square$]
  我能区分投资和赌博
\item[$\square$]
  我的''赌博账户''不超过 5\%
\end{itemize}

\begin{center}\rule{0.5\linewidth}{0.5pt}\end{center}

\textbf{下一章}:\href{14-ftx.md}{有效利他还是有效挪用}

\newpage

\section{第十四章:有效利他还是有效挪用}\label{ux7b2cux5341ux56dbux7ae0ux6709ux6548ux5229ux4ed6ux8fd8ux662fux6709ux6548ux632aux7528}

\begin{quote}
\textbf{2022 年 FTX 破产}
\end{quote}

\begin{center}\rule{0.5\linewidth}{0.5pt}\end{center}

\subsection{发生了什么}\label{ux53d1ux751fux4e86ux4ec0ux4e48-4}

Sam Bankman-Fried,30 岁,身价 260 亿美元。

他创办了 FTX,一个加密货币交易所。

他说他信奉''有效利他主义''------赚钱是为了捐给慈善。

他穿 T 恤短裤,住在巴哈马的''公社''里,和九个室友。

``我们是简朴的理想主义者。''

投资人排队给他送钱。红杉资本、软银、Tom Brady\ldots{}

2022 年 11 月,FTX 破产。

\textbf{客户的钱\ldots{} 不见了。}

原来他用客户的存款去投资、去买房子、去给自己发薪水。

``有效利他主义''变成了''有效挪用''。

Tommy 也亏了钱。他把剩下的几千块放在 FTX 上。

\begin{center}\rule{0.5\linewidth}{0.5pt}\end{center}

\subsection{为什么会发生}\label{ux4e3aux4ec0ux4e48ux4f1aux53d1ux751f-3}

\subsubsection{1. 光环太耀眼}\label{ux5149ux73afux592aux8000ux773c}

\begin{itemize}
\tightlist
\item
  30 岁,260 亿身价
\item
  登上杂志封面
\item
  和名人合影
\item
  说要把钱全捐给慈善
\end{itemize}

\textbf{谁会怀疑一个说要把钱全捐掉的人?}

\subsubsection{2. 没有人审计}\label{ux6ca1ux6709ux4ebaux5ba1ux8ba1}

加密货币行业几乎没有监管。

没有人检查 FTX 的账本。

等发现问题的时候,钱已经没了。

\subsubsection{3.
投资人没有做功课}\label{ux6295ux8d44ux4ebaux6ca1ux6709ux505aux529fux8bfe}

红杉资本投了 FTX,但没有认真尽职调查。

``他太聪明了,我们相信他。''

\textbf{聪明 ≠ 诚实。光环 ≠ 能力。}

\begin{center}\rule{0.5\linewidth}{0.5pt}\end{center}

\subsection{教训}\label{ux6559ux8bad-4}

\subsubsection{教训
1:看行动,不看口号}\label{ux6559ux8bad-1ux770bux884cux52a8ux4e0dux770bux53e3ux53f7}

{\def\LTcaptype{none} % do not increment counter
\begin{longtable}[]{@{}ll@{}}
\toprule\noalign{}
他说的 & 他做的 \\
\midrule\noalign{}
\endhead
\bottomrule\noalign{}
\endlastfoot
我要把钱全捐掉 & 用客户的钱买豪宅 \\
我们是简朴的理想主义者 & 住在巴哈马的豪华公寓 \\
我信奉有效利他主义 & 有效挪用客户资金 \\
\end{longtable}
}

\textbf{当一个人说他赚钱是为了帮助别人时,先看看他怎么花自己的钱。}

\subsubsection{教训
2:监管存在是有原因的}\label{ux6559ux8bad-2ux76d1ux7ba1ux5b58ux5728ux662fux6709ux539fux56e0ux7684}

传统银行有: - 存款保险(FDIC) - 定期审计 - 资本要求 - 监管检查

加密货币交易所没有这些。

\textbf{``去中心化''听起来很酷,但你的钱没有保障。}

\subsubsection{教训
3:不要把钱放在不受监管的地方}\label{ux6559ux8bad-3ux4e0dux8981ux628aux94b1ux653eux5728ux4e0dux53d7ux76d1ux7ba1ux7684ux5730ux65b9}

如果一个平台: - 不受监管 - 不能证明钱在哪里 - 创始人说得太好听

\textbf{那不是投资平台,那是赌博平台。}

\begin{center}\rule{0.5\linewidth}{0.5pt}\end{center}

\subsection{危险信号}\label{ux5371ux9669ux4fe1ux53f7}

{\def\LTcaptype{none} % do not increment counter
\begin{longtable}[]{@{}ll@{}}
\toprule\noalign{}
信号 & 意思 \\
\midrule\noalign{}
\endhead
\bottomrule\noalign{}
\endlastfoot
创始人说要把钱全捐掉 & 可能在道德绑架 \\
公司不受任何监管 & 你的钱没有保障 \\
回报高得离谱 & 可能是庞氏骗局 \\
审计报告不存在或可疑 & 钱可能不在那里 \\
创始人穿得太随便 & 可能是刻意的形象包装 \\
\end{longtable}
}

\begin{center}\rule{0.5\linewidth}{0.5pt}\end{center}

\subsection{怎么避免}\label{ux600eux4e48ux907fux514d-3}

\subsubsection{避免方法
1:只用受监管的平台}\label{ux907fux514dux65b9ux6cd5-1ux53eaux7528ux53d7ux76d1ux7ba1ux7684ux5e73ux53f0}

在美国: - 银行:FDIC 保险 - 券商:SIPC 保险 - 加密货币:Coinbase
是上市公司,有一定保障

\subsubsection{避免方法
2:不要把大量资金放在加密货币交易所}\label{ux907fux514dux65b9ux6cd5-2ux4e0dux8981ux628aux5927ux91cfux8d44ux91d1ux653eux5728ux52a0ux5bc6ux8d27ux5e01ux4ea4ux6613ux6240}

用来交易可以,但买完就转到自己的钱包。

``Not your keys, not your coins.''

\subsubsection{避免方法
3:对''好人''保持怀疑}\label{ux907fux514dux65b9ux6cd5-3ux5bf9ux597dux4ebaux4fddux6301ux6000ux7591}

骗子最喜欢装成好人。

\textbf{真正的好人不需要一直强调自己是好人。}

\begin{center}\rule{0.5\linewidth}{0.5pt}\end{center}

\subsection{一句话}\label{ux4e00ux53e5ux8bdd-13}

\begin{quote}
\textbf{当一个人说他赚钱是为了帮助世界时,先检查他的账本。}
\end{quote}

\begin{center}\rule{0.5\linewidth}{0.5pt}\end{center}

\subsection{检查清单}\label{ux68c0ux67e5ux6e05ux5355-13}

\begin{itemize}
\tightlist
\item[$\square$]
  我只用受监管的平台
\item[$\square$]
  我不把大量资金放在加密货币交易所
\item[$\square$]
  我对''太好的人''保持怀疑
\item[$\square$]
  我检查平台是否有审计和保险
\end{itemize}

\begin{center}\rule{0.5\linewidth}{0.5pt}\end{center}

\textbf{下一章}:\href{15-dont-buy-company-stock.md}{别把人力资本和金融资本放一起}

\newpage

\section{第十五章:别把人力资本和金融资本放一起}\label{ux7b2cux5341ux4e94ux7ae0ux522bux628aux4ebaux529bux8d44ux672cux548cux91d1ux878dux8d44ux672cux653eux4e00ux8d77}

\begin{quote}
``Sarah 说:你已经把人力资本押在那里了,不要再押金融资本。''
\end{quote}

\begin{center}\rule{0.5\linewidth}{0.5pt}\end{center}

\subsection{故事}\label{ux6545ux4e8b-9}

Mike 在安然工作。

公司给了他 5 万美元的 bonus。他全买了公司股票。

``我相信公司啊。''

2001 年,安然破产。

Mike 失业了。

Mike 的股票归零了。

\textbf{双杀。}

\begin{center}\rule{0.5\linewidth}{0.5pt}\end{center}

后来,Mike 去了雷曼。

他又把所有 bonus 放在公司。

2008 年,雷曼破产。

Mike 又失业了。

Mike 的投资又缩水了。

\textbf{同样的错误,犯了两次。}

\begin{center}\rule{0.5\linewidth}{0.5pt}\end{center}

\subsection{为什么这是危险的}\label{ux4e3aux4ec0ux4e48ux8fd9ux662fux5371ux9669ux7684}

\subsubsection{你在公司有两种资本}\label{ux4f60ux5728ux516cux53f8ux6709ux4e24ux79cdux8d44ux672c}

{\def\LTcaptype{none} % do not increment counter
\begin{longtable}[]{@{}ll@{}}
\toprule\noalign{}
资本类型 & 意思 \\
\midrule\noalign{}
\endhead
\bottomrule\noalign{}
\endlastfoot
\textbf{人力资本} & 你的工资、技能、职业发展 \\
\textbf{金融资本} & 你投资的钱 \\
\end{longtable}
}

如果你买公司股票,你在\textbf{双重押注}同一家公司。

\subsubsection{风险计算}\label{ux98ceux9669ux8ba1ux7b97}

\begin{verbatim}
正常情况:
- 人力资本在公司 A
- 金融资本分散在 A、B、C、D...
- A 出问题 → 你失业,但投资还在

Mike 的情况:
- 人力资本在安然
- 金融资本也在安然
- 安然出问题 → 失业 + 投资归零
\end{verbatim}

\textbf{这是最糟糕的分散。}

\begin{center}\rule{0.5\linewidth}{0.5pt}\end{center}

\subsection{规则}\label{ux89c4ux5219-9}

\subsubsection{规则
1:不要买公司股票}\label{ux89c4ux5219-1ux4e0dux8981ux4e70ux516cux53f8ux80a1ux7968}

除非: - 公司给你的 RSU/ESPP 是免费的 - 买了立刻卖掉

\subsubsection{规则 2:如果有
RSU,尽早卖}\label{ux89c4ux5219-2ux5982ux679cux6709-rsuux5c3dux65e9ux5356}

很多公司给员工 RSU(限制性股票单位)。

\textbf{拿到就卖。}

不要因为''我相信公司''而持有。

你已经用你的时间和职业押在公司了,不要再用钱押注。

\subsubsection{规则 3:如果有
ESPP,买了就卖}\label{ux89c4ux5219-3ux5982ux679cux6709-esppux4e70ux4e86ux5c31ux5356}

很多公司让员工用折扣买股票(通常 15\% 折扣)。

策略: 1. 用最大额度买 2. 买到就卖 3. 锁定 15\% 收益

\textbf{不要持有。}

\begin{center}\rule{0.5\linewidth}{0.5pt}\end{center}

\subsection{例外情况}\label{ux4f8bux5916ux60c5ux51b5}

\subsubsection{如果你在 FAANG
呢?}\label{ux5982ux679cux4f60ux5728-faang-ux5462}

``可是我在 Google/Apple/Meta 工作,股票一直涨啊!''

问自己:

\begin{enumerate}
\def\labelenumi{\arabic{enumi}.}
\tightlist
\item
  你能预测公司 10 年后的股价吗?
\item
  如果股票跌 50\%,你还失业,你能承受吗?
\item
  你比市场更了解公司吗?
\end{enumerate}

\textbf{如果答案是''不确定'',就卖。}

Sarah 在 Google 工作了 16 年,但她: - 定期卖出 RSU -
用卖出的钱买指数基金 - 保持分散

\begin{center}\rule{0.5\linewidth}{0.5pt}\end{center}

\subsection{真实数据}\label{ux771fux5b9eux6570ux636e}

公司股票占投资组合的最大比例:

{\def\LTcaptype{none} % do not increment counter
\begin{longtable}[]{@{}ll@{}}
\toprule\noalign{}
风险承受力 & 最大占比 \\
\midrule\noalign{}
\endhead
\bottomrule\noalign{}
\endlastfoot
保守 & 5\% \\
中等 & 10\% \\
激进 & 15\% \\
\end{longtable}
}

\textbf{超过这个比例就该卖了。}

\begin{center}\rule{0.5\linewidth}{0.5pt}\end{center}

\subsection{一句话}\label{ux4e00ux53e5ux8bdd-14}

\begin{quote}
\textbf{你的工资已经押在公司了,别再用存款押注。}
\end{quote}

\begin{center}\rule{0.5\linewidth}{0.5pt}\end{center}

\subsection{检查清单}\label{ux68c0ux67e5ux6e05ux5355-14}

\begin{itemize}
\tightlist
\item[$\square$]
  我没有持有大量公司股票(\textless{} 10\%)
\item[$\square$]
  我拿到 RSU 就卖
\item[$\square$]
  我参加 ESPP 是为了折扣,不是长期持有
\item[$\square$]
  我的人力资本和金融资本是分开的
\end{itemize}

\begin{center}\rule{0.5\linewidth}{0.5pt}\end{center}

\textbf{下一章}:\href{16-plan-b.md}{永远要有 Plan B}

\newpage

\section{第十六章:永远要有 Plan
B}\label{ux7b2cux5341ux516dux7ae0ux6c38ux8fdcux8981ux6709-plan-b}

\begin{quote}
``Sarah 说:在职场上,永远要有 Plan B。不要把你的命运完全交给别人。''
\end{quote}

\begin{center}\rule{0.5\linewidth}{0.5pt}\end{center}

\subsection{故事}\label{ux6545ux4e8b-10}

2011 年,我在 DataSphere 工作。

老板是印度人。他的老板也是印度人。我不是他们的''自己人''。

六个月后,绩效评估。

``Jim,你的技术能力很好,但你的文化融合需要提升。''

我明白了。我永远不会是他们中的一员。

我辞职了。

那天晚上,我给 Sarah 打电话。

``我被赶出来了。''

``\textbf{但你有 Plan B 吗?}''

``什么 Plan B?''

``\textbf{存款。技能。人脉。期权只是锦上添花,真正的安全感来自你银行账户里的数字。}''

\begin{center}\rule{0.5\linewidth}{0.5pt}\end{center}

\subsection{什么是 Plan B}\label{ux4ec0ux4e48ux662f-plan-b}

Plan B 不是''万一被裁员怎么办''。

Plan B 是\textbf{任何情况下你都能生存的底牌}。

\subsubsection{Plan B
的三个支柱}\label{plan-b-ux7684ux4e09ux4e2aux652fux67f1}

{\def\LTcaptype{none} % do not increment counter
\begin{longtable}[]{@{}lll@{}}
\toprule\noalign{}
支柱 & 内容 & 为什么重要 \\
\midrule\noalign{}
\endhead
\bottomrule\noalign{}
\endlastfoot
\textbf{存款} & 6-12 个月生活费 & 给你时间找下一步 \\
\textbf{技能} & 可转移的能力 & 让你有选择 \\
\textbf{人脉} & 可以帮你的人 & 获得机会 \\
\end{longtable}
}

\begin{center}\rule{0.5\linewidth}{0.5pt}\end{center}

\subsection{存款:你的安全垫}\label{ux5b58ux6b3eux4f60ux7684ux5b89ux5168ux57ab}

\subsubsection{规则}\label{ux89c4ux5219-10}

\begin{itemize}
\tightlist
\item
  至少 6 个月生活费
\item
  放在随时能取的地方
\item
  不用来投资
\end{itemize}

\subsubsection{为什么是 6
个月}\label{ux4e3aux4ec0ux4e48ux662f-6-ux4e2aux6708}

找工作平均需要 3-6 个月。

如果经济不好,可能更久。

\textbf{没有存款,你会被迫接受第一个 offer,哪怕很烂。}

有存款,你可以挑。

\begin{center}\rule{0.5\linewidth}{0.5pt}\end{center}

\subsection{技能:你的可转移能力}\label{ux6280ux80fdux4f60ux7684ux53efux8f6cux79fbux80fdux529b}

\subsubsection{什么是可转移技能}\label{ux4ec0ux4e48ux662fux53efux8f6cux79fbux6280ux80fd}

不依赖某家公司的能力:

{\def\LTcaptype{none} % do not increment counter
\begin{longtable}[]{@{}ll@{}}
\toprule\noalign{}
不可转移 & 可转移 \\
\midrule\noalign{}
\endhead
\bottomrule\noalign{}
\endlastfoot
公司内部系统 & 编程语言 \\
公司流程 & 解决问题能力 \\
公司人脉 & 行业人脉 \\
公司 title & 实际能力 \\
\end{longtable}
}

\subsubsection{怎么积累}\label{ux600eux4e48ux79efux7d2f}

\begin{enumerate}
\def\labelenumi{\arabic{enumi}.}
\tightlist
\item
  \textbf{学习通用技术} --- 不要只会公司的工具
\item
  \textbf{做 side project} --- 证明你能独立完成东西
\item
  \textbf{写作/演讲} --- 建立个人品牌
\item
  \textbf{考证书} --- 有时候有用
\end{enumerate}

\begin{center}\rule{0.5\linewidth}{0.5pt}\end{center}

\subsection{人脉:你的机会来源}\label{ux4ebaux8109ux4f60ux7684ux673aux4f1aux6765ux6e90}

\subsubsection{统计}\label{ux7edfux8ba1}

\begin{quote}
70\% 的工作是通过人脉找到的,不是招聘网站。
\end{quote}

\subsubsection{怎么建立}\label{ux600eux4e48ux5efaux7acb}

\begin{enumerate}
\def\labelenumi{\arabic{enumi}.}
\tightlist
\item
  \textbf{保持联系} --- 不要只在需要帮忙时才联系人
\item
  \textbf{帮助别人} --- 先给予,再索取
\item
  \textbf{参加行业活动} --- 线上线下都可以
\item
  \textbf{LinkedIn} --- 保持活跃,但不要太烦人
\end{enumerate}

\subsubsection{什么时候用}\label{ux4ec0ux4e48ux65f6ux5019ux7528}

\textbf{不是在你需要工作的时候才开始建立人脉。}

是在你不需要的时候就开始。

\begin{center}\rule{0.5\linewidth}{0.5pt}\end{center}

\subsection{2024 年的故事}\label{ux5e74ux7684ux6545ux4e8b}

我被新公司 PIP 了。

三个月后,HR 给我遣散方案。

24 年的职业生涯,一张支票就结束了。

但这次不一样。

\textbf{这次我有存款。我有技能。我有人脉。}

Sarah 说:``\textbf{这次你有钱了。你可以走了。}''

\begin{center}\rule{0.5\linewidth}{0.5pt}\end{center}

\subsection{一句话}\label{ux4e00ux53e5ux8bdd-15}

\begin{quote}
\textbf{不要把你的命运完全交给别人。你的安全感来自你自己。}
\end{quote}

\begin{center}\rule{0.5\linewidth}{0.5pt}\end{center}

\subsection{检查清单}\label{ux68c0ux67e5ux6e05ux5355-15}

\begin{itemize}
\tightlist
\item[$\square$]
  我有 6-12 个月生活费的存款
\item[$\square$]
  我的技能在其他公司也有用
\item[$\square$]
  我有行业内的人脉
\item[$\square$]
  我不依赖这份工作生存
\end{itemize}

\begin{center}\rule{0.5\linewidth}{0.5pt}\end{center}

\textbf{下一章}:\href{17-best-revenge.md}{最好的报复是过得好}

\newpage

\section{第十七章:最好的报复是过得好}\label{ux7b2cux5341ux4e03ux7ae0ux6700ux597dux7684ux62a5ux590dux662fux8fc7ux5f97ux597d}

\begin{quote}
``Sarah 说:最好的报复是过得好。''
\end{quote}

\begin{center}\rule{0.5\linewidth}{0.5pt}\end{center}

\subsection{故事}\label{ux6545ux4e8b-11}

DataSphere,八个月。

每天的羞辱: - 我的设计被偷走 - 空降的''表妹''抢了我的项目 -
绩效评估说我''文化融合不好''

我辞职了。

那天晚上,我想着那些人。想着我受到的不公平对待。想着要不要去 Glassdoor
上写差评。想着要不要告诉所有人这家公司有多烂。

妈妈说:\textbf{``儿子,有些地方不适合你。这不意味着你错了。是那个地方错了。''}

Sarah 说:\textbf{``最好的报复是过得好。''}

\begin{center}\rule{0.5\linewidth}{0.5pt}\end{center}

后来,我回到了 AI 领域。

NeuralMind 2.0。我热爱的技术。尊重我的人。

2024 年,公司被收购。我的期权变成了一套房子。

我不知道 DataSphere 的那些人现在怎么样了。

\textbf{我也不在乎。}

\begin{center}\rule{0.5\linewidth}{0.5pt}\end{center}

\subsection{为什么''过得好''是最好的报复}\label{ux4e3aux4ec0ux4e48ux8fc7ux5f97ux597dux662fux6700ux597dux7684ux62a5ux590d}

\subsubsection{1.
愤怒消耗的是你的能量}\label{ux6124ux6012ux6d88ux8017ux7684ux662fux4f60ux7684ux80fdux91cf}

当你花时间恨一个人、一家公司: - 你在想他们 - 你在给他们你的注意力 -
你在浪费你的生命

\textbf{他们不在乎。你在消耗自己。}

\subsubsection{2.
成功是沉默的}\label{ux6210ux529fux662fux6c89ux9ed8ux7684}

你不需要告诉任何人你过得好。

当你真的过得好的时候,他们自己会知道。

\subsubsection{3.
时间会证明一切}\label{ux65f6ux95f4ux4f1aux8bc1ux660eux4e00ux5207}

DataSphere 后来怎么样了?我不知道。

我只知道我过得挺好的。

\textbf{这就够了。}

\begin{center}\rule{0.5\linewidth}{0.5pt}\end{center}

\subsection{怎么''过得好''}\label{ux600eux4e48ux8fc7ux5f97ux597d}

\subsubsection{第一步:快速离开}\label{ux7b2cux4e00ux6b65ux5febux901fux79bbux5f00}

不要纠缠。不要抗争。不要试图改变一个不值得的地方。

\textbf{走就是了。}

\subsubsection{第二步:专注自己}\label{ux7b2cux4e8cux6b65ux4e13ux6ce8ux81eaux5df1}

离开后: - 不要在社交媒体上发负面内容 - 不要和每个人抱怨 - 不要活在过去

\textbf{把精力放在下一步上。}

\subsubsection{第三步:做好下一件事}\label{ux7b2cux4e09ux6b65ux505aux597dux4e0bux4e00ux4ef6ux4e8b}

找到值得你付出的地方。

做出成绩。

\textbf{成功是最响亮的回应。}

\begin{center}\rule{0.5\linewidth}{0.5pt}\end{center}

\subsection{什么时候该走}\label{ux4ec0ux4e48ux65f6ux5019ux8be5ux8d70}

不是所有战斗都值得打。

{\def\LTcaptype{none} % do not increment counter
\begin{longtable}[]{@{}ll@{}}
\toprule\noalign{}
值得打的仗 & 不值得打的仗 \\
\midrule\noalign{}
\endhead
\bottomrule\noalign{}
\endlastfoot
你能改变结果 & 结果已经决定了 \\
有明确的利益 & 只是为了''争口气'' \\
对方是讲理的人 & 对方根本不在乎 \\
胜算 \textgreater{} 50\% & 注定会输 \\
\end{longtable}
}

DataSphere 是一场不值得打的仗。

走开,是最聪明的选择。

\begin{center}\rule{0.5\linewidth}{0.5pt}\end{center}

\subsection{妈妈的话}\label{ux5988ux5988ux7684ux8bdd}

\begin{quote}
``儿子,不是每一场仗都值得打。有时候最聪明的做法是走开。''
\end{quote}

\begin{quote}
``有些地方不适合你。这不意味着你错了。是那个地方错了。''
\end{quote}

\begin{center}\rule{0.5\linewidth}{0.5pt}\end{center}

\subsection{一句话}\label{ux4e00ux53e5ux8bdd-16}

\begin{quote}
\textbf{活得好,是最有力的回应。}
\end{quote}

\begin{center}\rule{0.5\linewidth}{0.5pt}\end{center}

\subsection{检查清单}\label{ux68c0ux67e5ux6e05ux5355-16}

\begin{itemize}
\tightlist
\item[$\square$]
  我不在社交媒体上发负面内容
\item[$\square$]
  我不花时间恨任何人
\item[$\square$]
  我把精力放在自己的成长上
\item[$\square$]
  我已经走出了过去
\end{itemize}

\begin{center}\rule{0.5\linewidth}{0.5pt}\end{center}

\textbf{下一章}:\href{18-529-plan.md}{529 计划:税务利器}

\newpage

\section{第十八章:529
计划:税务利器}\label{ux7b2cux5341ux516bux7ae0529-ux8ba1ux5212ux7a0eux52a1ux5229ux5668}

\begin{quote}
``妈妈说:给孩子最好的投资,不是房子,是教育。但别忘了,政府也可以帮忙。''
\end{quote}

\begin{center}\rule{0.5\linewidth}{0.5pt}\end{center}

\subsection{什么是 529 计划}\label{ux4ec0ux4e48ux662f-529-ux8ba1ux5212}

529 计划是美国的教育储蓄账户。

\textbf{简单说:投进去的钱用于教育时免税。}

{\def\LTcaptype{none} % do not increment counter
\begin{longtable}[]{@{}ll@{}}
\toprule\noalign{}
特点 & 说明 \\
\midrule\noalign{}
\endhead
\bottomrule\noalign{}
\endlastfoot
联邦税 & 增长和取出都免税(用于教育时) \\
州税 & 很多州有额外减税优惠 \\
用途 & 学费、住宿、书本、电脑 \\
限制 & 只能用于教育,否则有罚款 \\
\end{longtable}
}

\begin{center}\rule{0.5\linewidth}{0.5pt}\end{center}

\subsection{为什么要用 529}\label{ux4e3aux4ec0ux4e48ux8981ux7528-529}

\subsubsection{例子}\label{ux4f8bux5b50}

假设你为孩子存 \$500/月,18 年:

{\def\LTcaptype{none} % do not increment counter
\begin{longtable}[]{@{}lll@{}}
\toprule\noalign{}
账户类型 & 18 年后总值 & 备注 \\
\midrule\noalign{}
\endhead
\bottomrule\noalign{}
\endlastfoot
普通投资账户 & \textasciitilde\$190,000 & 要交资本利得税 \\
529 账户 & \textasciitilde\$220,000 & 全部免税 \\
\end{longtable}
}

\textbf{差距:\$30,000}

这是税的力量。

\begin{center}\rule{0.5\linewidth}{0.5pt}\end{center}

\subsection{怎么开 529}\label{ux600eux4e48ux5f00-529}

\subsubsection{第一步:选择州}\label{ux7b2cux4e00ux6b65ux9009ux62e9ux5dde}

你可以开任何州的 529,但有些州给本州居民额外减税。

{\def\LTcaptype{none} % do not increment counter
\begin{longtable}[]{@{}ll@{}}
\toprule\noalign{}
你住的州 & 建议 \\
\midrule\noalign{}
\endhead
\bottomrule\noalign{}
\endlastfoot
有州税减免的州 & 开本州的 529 \\
没有州税的州(如 Texas) & 选费用最低的(如 Utah、Nevada) \\
州税减免不多 & 选投资选项最好的 \\
\end{longtable}
}

\subsubsection{第二步:开户}\label{ux7b2cux4e8cux6b65ux5f00ux6237}

\begin{enumerate}
\def\labelenumi{\arabic{enumi}.}
\tightlist
\item
  去州 529 官网
\item
  填写信息(你是账户持有人,孩子是受益人)
\item
  选择投资组合(推荐:目标日期基金)
\item
  设置自动转账
\end{enumerate}

\subsubsection{第三步:每月定投}\label{ux7b2cux4e09ux6b65ux6bcfux6708ux5b9aux6295}

设一个自动转账:

\begin{itemize}
\tightlist
\item
  发工资 → 自动转 \$X 到 529
\item
  不要想,不要看,让它自己涨
\end{itemize}

\begin{center}\rule{0.5\linewidth}{0.5pt}\end{center}

\subsection{常见问题}\label{ux5e38ux89c1ux95eeux9898-1}

\subsubsection{Q:孩子不上大学怎么办?}\label{qux5b69ux5b50ux4e0dux4e0aux5927ux5b66ux600eux4e48ux529e}

三个选择:

\begin{enumerate}
\def\labelenumi{\arabic{enumi}.}
\tightlist
\item
  \textbf{转给其他家庭成员} --- 兄弟姐妹、侄子侄女、甚至你自己
\item
  \textbf{用于 K-12 私立学校} --- 每年最多 \$10,000
\item
  \textbf{取出来交罚款} --- 10\% 罚款 + 补税(不推荐)
\end{enumerate}

\subsubsection{Q:存多少够?}\label{qux5b58ux591aux5c11ux591f}

{\def\LTcaptype{none} % do not increment counter
\begin{longtable}[]{@{}lll@{}}
\toprule\noalign{}
学校类型 & 估算 4 年费用(2025年) & 每月定投 18 年 \\
\midrule\noalign{}
\endhead
\bottomrule\noalign{}
\endlastfoot
公立(州内) & \$100,000 & \textasciitilde\$300/月 \\
公立(州外) & \$180,000 & \textasciitilde\$550/月 \\
私立 & \$320,000 & \textasciitilde\$950/月 \\
\end{longtable}
}

\textbf{不需要存满全部。有助学金、奖学金、学生贷款。}

存能存的,不要有压力。

\subsubsection{Q:影响助学金吗?}\label{qux5f71ux54cdux52a9ux5b66ux91d1ux5417}

529 账户在助学金计算中影响很小。

父母名下的 529 只算约 5.64\%。

影响不大。

\begin{center}\rule{0.5\linewidth}{0.5pt}\end{center}

\subsection{规则}\label{ux89c4ux5219-11}

\subsubsection{规则
1:尽早开始}\label{ux89c4ux5219-1ux5c3dux65e9ux5f00ux59cb}

复利需要时间。

孩子出生就开 529,不要等。

\subsubsection{规则 2:自动化}\label{ux89c4ux5219-2ux81eaux52a8ux5316}

设置每月自动转账。

不要靠意志力。

\subsubsection{规则
3:选择低费率}\label{ux89c4ux5219-3ux9009ux62e9ux4f4eux8d39ux7387}

529 也有费率差异。

选费率 \textless{} 0.5\% 的。

\subsubsection{规则
4:投资选择简单化}\label{ux89c4ux5219-4ux6295ux8d44ux9009ux62e9ux7b80ux5355ux5316}

选''目标日期基金''(Target Date Fund)。

孩子 2042 年上大学?选 ``2042 Target Date'' 基金。

\textbf{自动调整风险,不用管。}

\begin{center}\rule{0.5\linewidth}{0.5pt}\end{center}

\subsection{一句话}\label{ux4e00ux53e5ux8bdd-17}

\begin{quote}
\textbf{529 是最简单的税务优惠。能用就用。}
\end{quote}

\begin{center}\rule{0.5\linewidth}{0.5pt}\end{center}

\subsection{检查清单}\label{ux68c0ux67e5ux6e05ux5355-17}

\begin{itemize}
\tightlist
\item[$\square$]
  我开了 529 账户
\item[$\square$]
  我设置了每月自动转账
\item[$\square$]
  我选择了低费率的投资选项
\item[$\square$]
  我不会为了存 529 影响自己的退休储蓄
\end{itemize}

\begin{center}\rule{0.5\linewidth}{0.5pt}\end{center}

\textbf{下一章}:\href{19-public-vs-private.md}{公立还是私立}

\newpage

\section{第十九章:公立还是私立}\label{ux7b2cux5341ux4e5dux7ae0ux516cux7acbux8fd8ux662fux79c1ux7acb}

\begin{quote}
``妈妈说:花钱能买到好教育,但最好的老师是免费的------那就是你自己。''
\end{quote}

\begin{center}\rule{0.5\linewidth}{0.5pt}\end{center}

\subsection{问题}\label{ux95eeux9898}

``公立学校还是私立学校?''

这是每个硅谷家长的问题。

答案不是''哪个更好'',而是''什么适合你''。

\begin{center}\rule{0.5\linewidth}{0.5pt}\end{center}

\subsection{数字说话}\label{ux6570ux5b57ux8bf4ux8bdd}

\subsubsection{费用对比(加州 2025
年)}\label{ux8d39ux7528ux5bf9ux6bd4ux52a0ux5dde-2025-ux5e74}

{\def\LTcaptype{none} % do not increment counter
\begin{longtable}[]{@{}lll@{}}
\toprule\noalign{}
类型 & 年费用 & K-12 总费用 \\
\midrule\noalign{}
\endhead
\bottomrule\noalign{}
\endlastfoot
公立 & \$0 & \$0 \\
私立普通 & \$25,000-40,000 & \$325,000-520,000 \\
私立顶尖 & \$50,000-65,000 & \$650,000-845,000 \\
\end{longtable}
}

\textbf{私立 K-12 可能比大学还贵。}

\subsubsection{机会成本}\label{ux673aux4f1aux6210ux672c}

如果把 \$50,000/年 投到指数基金,18 年后:

\begin{verbatim}
每年 $50,000
× 18 年
× 7% 年化回报
= ~$1,700,000
\end{verbatim}

\textbf{一百七十万美元。}

这是选择私立学校的真实成本。

\begin{center}\rule{0.5\linewidth}{0.5pt}\end{center}

\subsection{什么时候值得}\label{ux4ec0ux4e48ux65f6ux5019ux503cux5f97}

\subsubsection{公立学校足够好的情况}\label{ux516cux7acbux5b66ux6821ux8db3ux591fux597dux7684ux60c5ux51b5}

{\def\LTcaptype{none} % do not increment counter
\begin{longtable}[]{@{}ll@{}}
\toprule\noalign{}
条件 & 说明 \\
\midrule\noalign{}
\endhead
\bottomrule\noalign{}
\endlastfoot
你住在好学区 & Cupertino、Palo Alto、Fremont \\
孩子没有特殊需求 & 标准课程能满足 \\
你愿意补充课外 & 课后班、夏令营、家庭教育 \\
\end{longtable}
}

\subsubsection{私立学校值得考虑的情况}\label{ux79c1ux7acbux5b66ux6821ux503cux5f97ux8003ux8651ux7684ux60c5ux51b5}

{\def\LTcaptype{none} % do not increment counter
\begin{longtable}[]{@{}ll@{}}
\toprule\noalign{}
条件 & 说明 \\
\midrule\noalign{}
\endhead
\bottomrule\noalign{}
\endlastfoot
孩子有特殊需求 & 学习障碍、天才儿童 \\
公立学区很差 & 搬家成本太高 \\
宗教/价值观 & 特定教育理念 \\
你真的负担得起 & 不影响退休储蓄 \\
\end{longtable}
}

\begin{center}\rule{0.5\linewidth}{0.5pt}\end{center}

\subsection{隐藏的选项}\label{ux9690ux85cfux7684ux9009ux9879}

\subsubsection{选项
1:买房进好学区}\label{ux9009ux9879-1ux4e70ux623fux8fdbux597dux5b66ux533a}

在好学区买房可能比私立学校更划算:

\begin{verbatim}
Cupertino 房子 vs Los Gatos 房子差价:~$500,000
但这是资产,不是消费
18 年后房子可能增值
\end{verbatim}

\textbf{房子会涨,学费不会回来。}

\subsubsection{选项 2:公立 +
补充}\label{ux9009ux9879-2ux516cux7acb-ux8865ux5145}

公立学校 + 课外活动:

\begin{verbatim}
公立学校:$0
钢琴课:$5,000/年
编程课:$3,000/年
夏令营:$5,000/年
家教:$5,000/年
---
总计:~$18,000/年
\end{verbatim}

\textbf{比私立便宜一半,灵活度更高。}

\subsubsection{选项
3:只念私立高中}\label{ux9009ux9879-3ux53eaux5ff5ux79c1ux7acbux9ad8ux4e2d}

有些家庭的策略:

\begin{itemize}
\tightlist
\item
  K-8:公立
\item
  9-12:私立
\end{itemize}

高中 4 年私立:\$200,000 vs K-12 私立:\$700,000

\textbf{省了 \$500,000。}

\begin{center}\rule{0.5\linewidth}{0.5pt}\end{center}

\subsection{真正的问题}\label{ux771fux6b63ux7684ux95eeux9898}

问自己:

\begin{enumerate}
\def\labelenumi{\arabic{enumi}.}
\tightlist
\item
  \textbf{为什么想让孩子上私立?}

  \begin{itemize}
  \tightlist
  \item
    如果是''大家都上'',那不是好理由
  \item
    如果是''孩子有特殊需求'',那值得考虑
  \end{itemize}
\item
  \textbf{这会影响你的财务安全吗?}

  \begin{itemize}
  \tightlist
  \item
    如果要借钱、动用退休金 → 不值得
  \item
    如果轻松负担 → 可以考虑
  \end{itemize}
\item
  \textbf{孩子真的需要吗?}

  \begin{itemize}
  \tightlist
  \item
    有些孩子在任何环境都能成功
  \item
    有些孩子需要特定环境
  \end{itemize}
\end{enumerate}

\begin{center}\rule{0.5\linewidth}{0.5pt}\end{center}

\subsection{妈妈的观点}\label{ux5988ux5988ux7684ux89c2ux70b9}

妈妈在 Baton Rouge 的公立学校读书。

她说:``\textbf{学校教知识,家庭教做人。好学校是加分,不是必须。}''

Sarah 从 Berkeley 毕业,公立大学。

她说:``\textbf{我认识的最成功的人,不是从最好的学校出来的。是那些最努力的人。}''

\begin{center}\rule{0.5\linewidth}{0.5pt}\end{center}

\subsection{一句话}\label{ux4e00ux53e5ux8bdd-18}

\begin{quote}
\textbf{教育投资要算账,但最好的投资是你花在孩子身上的时间。}
\end{quote}

\begin{center}\rule{0.5\linewidth}{0.5pt}\end{center}

\subsection{检查清单}\label{ux68c0ux67e5ux6e05ux5355-18}

\begin{itemize}
\tightlist
\item[$\square$]
  我知道公立和私立的真实费用差异
\item[$\square$]
  我考虑了机会成本
\item[$\square$]
  我的选择不会影响退休储蓄
\item[$\square$]
  我问过孩子需要什么
\end{itemize}

\begin{center}\rule{0.5\linewidth}{0.5pt}\end{center}

\textbf{下一章}:\href{20-beyond-ivy.md}{爬藤不是唯一的路}

\newpage

\section{第二十章:爬藤不是唯一的路}\label{ux7b2cux4e8cux5341ux7ae0ux722cux85e4ux4e0dux662fux552fux4e00ux7684ux8def}

\begin{quote}
``妈妈说:成功的路有很多条。藤校只是其中一条,还是最拥挤的那条。''
\end{quote}

\begin{center}\rule{0.5\linewidth}{0.5pt}\end{center}

\subsection{硅谷的焦虑}\label{ux7845ux8c37ux7684ux7126ux8651}

在硅谷,有一种病叫''爬藤焦虑''。

症状: - 从幼儿园开始规划''藤校路径'' - 花 \$500/小时找升学顾问 -
让孩子学 10 种乐器和 5 种运动 - 凌晨 2 点还在改孩子的申请文书

最终结果:

\textbf{藤校录取率 \textless{} 5\%。}

\textbf{95\% 的孩子会''失败''。}

但他们真的失败了吗?

\begin{center}\rule{0.5\linewidth}{0.5pt}\end{center}

\subsection{数字说话}\label{ux6570ux5b57ux8bf4ux8bdd-1}

\subsubsection{藤校 vs
州立大学}\label{ux85e4ux6821-vs-ux5ddeux7acbux5927ux5b66}

{\def\LTcaptype{none} % do not increment counter
\begin{longtable}[]{@{}lll@{}}
\toprule\noalign{}
学校 & 4 年费用 & 毕业后薪资 \\
\midrule\noalign{}
\endhead
\bottomrule\noalign{}
\endlastfoot
哈佛 & \$320,000 & \$90,000 起薪 \\
UC Berkeley & \$120,000(州内) & \$85,000 起薪 \\
差距 & \$200,000 & \$5,000/年 \\
\end{longtable}
}

\textbf{要多少年才能赚回 \$200,000 的差价?}

\subsubsection{真实的成功案例}\label{ux771fux5b9eux7684ux6210ux529fux6848ux4f8b}

{\def\LTcaptype{none} % do not increment counter
\begin{longtable}[]{@{}lll@{}}
\toprule\noalign{}
名字 & 学校 & 成就 \\
\midrule\noalign{}
\endhead
\bottomrule\noalign{}
\endlastfoot
Steve Jobs & Reed College(辍学) & Apple 创始人 \\
Larry Ellison & UIUC(辍学) & Oracle 创始人 \\
Jensen Huang & Oregon State & Nvidia 创始人 \\
Sarah Chen & UC Berkeley & Google 退休,财务自由 \\
\end{longtable}
}

\textbf{藤校不是成功的必要条件。}

\begin{center}\rule{0.5\linewidth}{0.5pt}\end{center}

\subsection{真正重要的是什么}\label{ux771fux6b63ux91cdux8981ux7684ux662fux4ec0ux4e48}

\subsubsection{1. 专业选择 \textgreater{}
学校排名}\label{ux4e13ux4e1aux9009ux62e9-ux5b66ux6821ux6392ux540d}

{\def\LTcaptype{none} % do not increment counter
\begin{longtable}[]{@{}lll@{}}
\toprule\noalign{}
专业 & 州立大学薪资 & 藤校薪资 \\
\midrule\noalign{}
\endhead
\bottomrule\noalign{}
\endlastfoot
计算机 & \$110,000 & \$120,000 \\
社会学 & \$45,000 & \$50,000 \\
\end{longtable}
}

\textbf{计算机专业的州立毕业生比社会学的藤校毕业生赚得多。}

\subsubsection{2. 技能 \textgreater{}
学历}\label{ux6280ux80fd-ux5b66ux5386}

雇主最看重的: 1. 能解决问题 2. 能和人合作 3. 能持续学习 4.
有实际项目经验

\textbf{学历只是敲门砖。能力才是护城河。}

\subsubsection{3.
人脉可以后天建立}\label{ux4ebaux8109ux53efux4ee5ux540eux5929ux5efaux7acb}

``藤校的价值是人脉。''

这是真的。但人脉也可以在工作中建立。

Sarah 在 Google 16 年,她的人脉比任何藤校都强。

\begin{center}\rule{0.5\linewidth}{0.5pt}\end{center}

\subsection{更聪明的路径}\label{ux66f4ux806aux660eux7684ux8defux5f84}

\subsubsection{路径 1:州立名校 +
存钱}\label{ux8defux5f84-1ux5ddeux7acbux540dux6821-ux5b58ux94b1}

去 UC Berkeley / UCLA / UT Austin / Georgia Tech。

省下 \$200,000 差价。

用这笔钱创业、投资、或者读研。

\subsubsection{路径 2:社区大学 →
转学}\label{ux8defux5f84-2ux793eux533aux5927ux5b66-ux8f6cux5b66}

加州社区大学 → UC 系统转学。

2 年社区大学:\$10,000 2 年 UC Berkeley:\$60,000 总计:\$70,000

\textbf{一样的学位,三分之一的价格。}

\subsubsection{路径
3:技术路线}\label{ux8defux5f84-3ux6280ux672fux8defux7ebf}

Coding Bootcamp:\$15,000-20,000 培训时间:3-6 个月
起薪:\$80,000-120,000

\textbf{4 年 vs 6 个月。\$300,000 vs \$20,000。}

\begin{center}\rule{0.5\linewidth}{0.5pt}\end{center}

\subsection{给家长的话}\label{ux7ed9ux5bb6ux957fux7684ux8bdd}

\begin{enumerate}
\def\labelenumi{\arabic{enumi}.}
\tightlist
\item
  \textbf{别把你的焦虑传给孩子}

  \begin{itemize}
  \tightlist
  \item
    你的面子不应该成为孩子的压力
  \end{itemize}
\item
  \textbf{成功有很多定义}

  \begin{itemize}
  \tightlist
  \item
    赚钱?快乐?有意义的工作?
  \item
    藤校不保证任何一个
  \end{itemize}
\item
  \textbf{最好的投资是陪伴}

  \begin{itemize}
  \tightlist
  \item
    花 \$50 万请顾问 vs 每天陪孩子 1 小时
  \item
    后者的回报更高
  \end{itemize}
\end{enumerate}

\begin{center}\rule{0.5\linewidth}{0.5pt}\end{center}

\subsection{妈妈的话}\label{ux5988ux5988ux7684ux8bdd-1}

妈妈没读过大学。

她在 Baton Rouge 开了一家餐馆,养大了我。

她说:``\textbf{儿子,我没有给你藤校,但我给了你正确的价值观。这比任何学位都值钱。}''

\begin{center}\rule{0.5\linewidth}{0.5pt}\end{center}

\subsection{一句话}\label{ux4e00ux53e5ux8bdd-19}

\begin{quote}
\textbf{藤校是通往成功的一条路,但不是唯一的路,也不是最好的路。}
\end{quote}

\begin{center}\rule{0.5\linewidth}{0.5pt}\end{center}

\subsection{检查清单}\label{ux68c0ux67e5ux6e05ux5355-19}

\begin{itemize}
\tightlist
\item[$\square$]
  我没有把自己的焦虑传给孩子
\item[$\square$]
  我考虑了投资回报率,不只是排名
\item[$\square$]
  我支持孩子找到适合自己的路
\item[$\square$]
  我知道成功的定义不只是名校
\end{itemize}

\begin{center}\rule{0.5\linewidth}{0.5pt}\end{center}

\textbf{下一章}:\href{21-kids-financial-literacy.md}{财商教育从小开始}

\newpage

\section{第二十一章:财商教育从小开始}\label{ux7b2cux4e8cux5341ux4e00ux7ae0ux8d22ux5546ux6559ux80b2ux4eceux5c0fux5f00ux59cb}

\begin{quote}
``妈妈说:给孩子鱼,不如教孩子钓鱼。给孩子钱,不如教孩子理钱。''
\end{quote}

\begin{center}\rule{0.5\linewidth}{0.5pt}\end{center}

\subsection{问题}\label{ux95eeux9898-1}

Tommy,Mike 的儿子。

他用大学学费买了 GameStop。 他把剩下的钱放在 FTX 上。 他问爸爸借钱买
Nvidia。

为什么?

\textbf{因为没有人教过他怎么理财。}

学校不教。父母假设他''长大就会懂''。

结果就是:一代又一代人在同样的地方踩同样的坑。

\begin{center}\rule{0.5\linewidth}{0.5pt}\end{center}

\subsection{什么时候开始}\label{ux4ec0ux4e48ux65f6ux5019ux5f00ux59cb}

{\def\LTcaptype{none} % do not increment counter
\begin{longtable}[]{@{}ll@{}}
\toprule\noalign{}
年龄 & 可以教的概念 \\
\midrule\noalign{}
\endhead
\bottomrule\noalign{}
\endlastfoot
3-5 岁 & 钱是什么,不同硬币的价值 \\
6-8 岁 & 储蓄 vs 花费,想要 vs 需要 \\
9-12 岁 & 预算,复利,目标储蓄 \\
13-15 岁 & 打工赚钱,银行账户,借记卡 \\
16-18 岁 & 投资基础,信用,税 \\
\end{longtable}
}

\textbf{越早开始,习惯越容易形成。}

\begin{center}\rule{0.5\linewidth}{0.5pt}\end{center}

\subsection{实际做法}\label{ux5b9eux9645ux505aux6cd5}

\subsubsection{1.
给零花钱,但有规则}\label{ux7ed9ux96f6ux82b1ux94b1ux4f46ux6709ux89c4ux5219}

从 6 岁开始,每周固定零花钱。

规则: - \textbf{存 30\%} --- 长期储蓄 - \textbf{捐 10\%} --- 慈善 -
\textbf{花 60\%} --- 自由支配

\textbf{让他们体验选择的代价。}

\subsubsection{2. 三个罐子}\label{ux4e09ux4e2aux7f50ux5b50}

买三个透明罐子:

{\def\LTcaptype{none} % do not increment counter
\begin{longtable}[]{@{}lll@{}}
\toprule\noalign{}
罐子 & 用途 & 规则 \\
\midrule\noalign{}
\endhead
\bottomrule\noalign{}
\endlastfoot
🏦 储蓄 & 长期目标 & 不能随便动 \\
🎁 花费 & 日常想要的东西 & 可以自由花 \\
❤️ 捐赠 & 帮助别人 & 定期捐出去 \\
\end{longtable}
}

\textbf{可视化让孩子理解钱的流向。}

\subsubsection{3. 让他们犯错}\label{ux8ba9ux4ed6ux4eecux72afux9519}

孩子想用零花钱买一个垃圾玩具?

\textbf{让他买。}

三天后玩具坏了,他会学到教训。

\textbf{10 块钱的教训比 10 万块钱的教训便宜多了。}

\subsubsection{4.
打开你的财务}\label{ux6253ux5f00ux4f60ux7684ux8d22ux52a1}

很多家庭对钱讳莫如深。

但孩子需要看到真实的数字:

\begin{itemize}
\tightlist
\item
  家庭收入多少
\item
  房贷多少
\item
  每月存多少
\item
  投资账户长什么样
\end{itemize}

\textbf{不是炫耀,是教育。}

\subsubsection{5. 一起投资}\label{ux4e00ux8d77ux6295ux8d44}

给孩子开一个托管账户(Custodial Account)。

让他们选一只股票或 ETF。

\begin{itemize}
\tightlist
\item
  一起看财报(简化版)
\item
  一起讨论公司在做什么
\item
  一起经历涨跌
\end{itemize}

\textbf{体验是最好的老师。}

\begin{center}\rule{0.5\linewidth}{0.5pt}\end{center}

\subsection{关键对话}\label{ux5173ux952eux5bf9ux8bdd}

\subsubsection{对话 1:想要 vs
需要}\label{ux5bf9ux8bdd-1ux60f3ux8981-vs-ux9700ux8981}

孩子:``我想要这个玩具!''

你:``这是'想要'还是'需要'?''

孩子:``想要。''

你:``好,那你愿意用你的零花钱买吗?''

\subsubsection{对话 2:复利}\label{ux5bf9ux8bdd-2ux590dux5229}

用 \$100 解释复利:

``如果你存 \$100,每年涨 10\%: - 1 年后:\$110 - 10 年后:\$259 - 30
年后:\$1,745

\textbf{你的钱会帮你赚钱。这就是为什么要早存。}''

\subsubsection{对话 3:风险}\label{ux5bf9ux8bdd-3ux98ceux9669}

``Tommy 叔叔买了 GameStop,赚了钱,又亏光了。为什么?''

``因为那是赌博,不是投资。\textbf{投资是买好公司,长期持有。赌博是猜明天涨还是跌。}''

\begin{center}\rule{0.5\linewidth}{0.5pt}\end{center}

\subsection{推荐资源}\label{ux63a8ux8350ux8d44ux6e90}

{\def\LTcaptype{none} % do not increment counter
\begin{longtable}[]{@{}lll@{}}
\toprule\noalign{}
资源 & 适合年龄 & 类型 \\
\midrule\noalign{}
\endhead
\bottomrule\noalign{}
\endlastfoot
《小狗钱钱》 & 8-12 & 书 \\
Greenlight 卡 & 8+ & 儿童借记卡 \\
Stockpile & 13+ & 托管投资账户 \\
``The Money Guy'' 播客 & 16+ & 播客 \\
\end{longtable}
}

\begin{center}\rule{0.5\linewidth}{0.5pt}\end{center}

\subsection{一句话}\label{ux4e00ux53e5ux8bdd-20}

\begin{quote}
\textbf{你能给孩子最好的遗产不是钱,是正确的金钱观。}
\end{quote}

\begin{center}\rule{0.5\linewidth}{0.5pt}\end{center}

\subsection{检查清单}\label{ux68c0ux67e5ux6e05ux5355-20}

\begin{itemize}
\tightlist
\item[$\square$]
  孩子有零花钱,并学会分配
\item[$\square$]
  我允许孩子犯小错误
\item[$\square$]
  我和孩子讨论家庭财务
\item[$\square$]
  孩子知道投资和赌博的区别
\end{itemize}

\begin{center}\rule{0.5\linewidth}{0.5pt}\end{center}

\textbf{下一章}:\href{22-compound-time.md}{复利需要时间}

\newpage

\section{第二十二章:复利需要时间}\label{ux7b2cux4e8cux5341ux4e8cux7ae0ux590dux5229ux9700ux8981ux65f6ux95f4}

\begin{quote}
``妈妈说:慢慢来,才能快。''
\end{quote}

\begin{center}\rule{0.5\linewidth}{0.5pt}\end{center}

\subsection{故事}\label{ux6545ux4e8b-12}

2008 年,我用 15 万美元买了指数基金。

2009 年,它跌到了 8 万。

2024 年,它变成了 60 多万。

\textbf{16 年。4 倍。}

中间我做了什么?

\textbf{什么都没做。}

买入。持有。等待。

这就是复利。

\begin{center}\rule{0.5\linewidth}{0.5pt}\end{center}

\subsection{复利的数学}\label{ux590dux5229ux7684ux6570ux5b66}

\subsubsection{简单版}\label{ux7b80ux5355ux7248}

假设年回报 10\%:

{\def\LTcaptype{none} % do not increment counter
\begin{longtable}[]{@{}ll@{}}
\toprule\noalign{}
年数 & \$10,000 变成 \\
\midrule\noalign{}
\endhead
\bottomrule\noalign{}
\endlastfoot
1 年 & \$11,000 \\
5 年 & \$16,105 \\
10 年 & \$25,937 \\
20 年 & \$67,275 \\
30 年 & \$174,494 \\
40 年 & \$452,593 \\
\end{longtable}
}

\textbf{最后 10 年赚的比前 30 年加起来还多。}

这就是为什么时间这么重要。

\subsubsection{72 法则}\label{ux6cd5ux5219}

想知道多少年钱翻倍?

\textbf{72 ÷ 年回报率 = 翻倍年数}

{\def\LTcaptype{none} % do not increment counter
\begin{longtable}[]{@{}ll@{}}
\toprule\noalign{}
年回报率 & 翻倍年数 \\
\midrule\noalign{}
\endhead
\bottomrule\noalign{}
\endlastfoot
6\% & 12 年 \\
8\% & 9 年 \\
10\% & 7.2 年 \\
12\% & 6 年 \\
\end{longtable}
}

\begin{center}\rule{0.5\linewidth}{0.5pt}\end{center}

\subsection{时间 vs 金额}\label{ux65f6ux95f4-vs-ux91d1ux989d}

\subsubsection{问题}\label{ux95eeux9898-2}

谁最后有更多钱?

{\def\LTcaptype{none} % do not increment counter
\begin{longtable}[]{@{}lllll@{}}
\toprule\noalign{}
人物 & 开始年龄 & 每月投 & 投资年数 & 60 岁时 \\
\midrule\noalign{}
\endhead
\bottomrule\noalign{}
\endlastfoot
A & 25 岁 & \$500 & 35 年 & \$1,150,000 \\
B & 35 岁 & \$1,000 & 25 年 & \$790,000 \\
\end{longtable}
}

\textbf{A 每个月只投一半,但最后多 36 万。}

\textbf{因为 A 早开始了 10 年。}

\subsubsection{教训}\label{ux6559ux8bad-5}

\begin{itemize}
\tightlist
\item
  开始比金额重要
\item
  早 10 年 \textgreater{} 多投一倍
\item
  最好的投资时机是 10 年前,其次是现在
\end{itemize}

\begin{center}\rule{0.5\linewidth}{0.5pt}\end{center}

\subsection{复利的敌人}\label{ux590dux5229ux7684ux654cux4eba}

\subsubsection{敌人 1:中断}\label{ux654cux4eba-1ux4e2dux65ad}

停止定投,哪怕只有几年,损失巨大。

{\def\LTcaptype{none} % do not increment counter
\begin{longtable}[]{@{}ll@{}}
\toprule\noalign{}
情况 & 60 岁时 \\
\midrule\noalign{}
\endhead
\bottomrule\noalign{}
\endlastfoot
25 岁开始,不中断 & \$1,150,000 \\
25 岁开始,35-40 岁停了 5 年 & \$750,000 \\
\end{longtable}
}

\textbf{损失 \$400,000。}

\subsubsection{敌人 2:择时}\label{ux654cux4eba-2ux62e9ux65f6}

试图''低买高卖'',结果错过最好的日子。

{\def\LTcaptype{none} % do not increment counter
\begin{longtable}[]{@{}ll@{}}
\toprule\noalign{}
情况 & 20 年回报 \\
\midrule\noalign{}
\endhead
\bottomrule\noalign{}
\endlastfoot
一直持有 & \$64,000 → \$400,000 \\
错过最好的 10 天 & \$64,000 → \$200,000 \\
错过最好的 20 天 & \$64,000 → \$130,000 \\
\end{longtable}
}

\textbf{错过 20 天,回报少了 68\%。}

\subsubsection{敌人 3:高费率}\label{ux654cux4eba-3ux9ad8ux8d39ux7387}

1\% 的费率差距,30 年后差 30\% 的回报。

{\def\LTcaptype{none} % do not increment counter
\begin{longtable}[]{@{}ll@{}}
\toprule\noalign{}
费率 & \$100,000 30 年后 \\
\midrule\noalign{}
\endhead
\bottomrule\noalign{}
\endlastfoot
0.1\% & \$574,000 \\
1.0\% & \$432,000 \\
差距 & \$142,000 \\
\end{longtable}
}

\textbf{选低费率的基金。}

\begin{center}\rule{0.5\linewidth}{0.5pt}\end{center}

\subsection{怎么利用复利}\label{ux600eux4e48ux5229ux7528ux590dux5229}

\subsubsection{规则
1:尽早开始}\label{ux89c4ux5219-1ux5c3dux65e9ux5f00ux59cb-1}

不要等''有钱了再投资''。

\textbf{今天开始。哪怕每月 \$100。}

\subsubsection{规则
2:永不中断}\label{ux89c4ux5219-2ux6c38ux4e0dux4e2dux65ad}

自动化你的投资。

发工资 → 自动转账 → 自动买入

\textbf{把决策权从自己手里拿走。}

\subsubsection{规则
3:永不卖出}\label{ux89c4ux5219-3ux6c38ux4e0dux5356ux51fa}

除非需要用钱,否则不卖。

\textbf{每次卖出都打断复利。}

\subsubsection{规则 4:低费率}\label{ux89c4ux5219-4ux4f4eux8d39ux7387}

选费率 \textless{} 0.1\% 的指数基金。

\textbf{Vanguard、Fidelity、Schwab。}

\begin{center}\rule{0.5\linewidth}{0.5pt}\end{center}

\subsection{一句话}\label{ux4e00ux53e5ux8bdd-21}

\begin{quote}
\textbf{复利是世界第八大奇迹。但它需要时间。给它时间。}
\end{quote}

\begin{center}\rule{0.5\linewidth}{0.5pt}\end{center}

\subsection{检查清单}\label{ux68c0ux67e5ux6e05ux5355-21}

\begin{itemize}
\tightlist
\item[$\square$]
  我已经开始投资了(不管金额多少)
\item[$\square$]
  我设置了自动定投
\item[$\square$]
  我不会中途停止
\item[$\square$]
  我选择了低费率的基金
\end{itemize}

\begin{center}\rule{0.5\linewidth}{0.5pt}\end{center}

\textbf{下一章}:\href{23-real-wealth.md}{真正的财富是时间}

\newpage

\section{第二十三章:真正的财富是时间}\label{ux7b2cux4e8cux5341ux4e09ux7ae0ux771fux6b63ux7684ux8d22ux5bccux662fux65f6ux95f4}

\begin{quote}
``Sarah 说:真正的财富是时间,不是钱。钱可以买时间,但时间不能买回来。''
\end{quote}

\begin{center}\rule{0.5\linewidth}{0.5pt}\end{center}

\subsection{故事}\label{ux6545ux4e8b-13}

2024 年,Sarah 给我打电话。

``我上个月退休了。''

``退休?你才 48 岁。''

``\textbf{Google 的股票够我活三辈子。我想去画画了。}''

她在 Google 工作了 16 年。选对了公司,拿了期权,持有,不动。

现在她可以做任何想做的事。

\textbf{这才是真正的财富------选择的自由。}

\begin{center}\rule{0.5\linewidth}{0.5pt}\end{center}

\subsection{什么是''财务自由''}\label{ux4ec0ux4e48ux662fux8d22ux52a1ux81eaux7531}

不是: - 拥有一千万美元 - 住豪宅开豪车 - 不用工作

而是: - \textbf{可以不工作},但选择做喜欢的事 -
\textbf{不用担心钱},可以专注于真正重要的事 -
\textbf{有时间},给家人、给自己、给世界

\begin{center}\rule{0.5\linewidth}{0.5pt}\end{center}

\subsection{财务自由的数学}\label{ux8d22ux52a1ux81eaux7531ux7684ux6570ux5b66}

\subsubsection{4\% 法则}\label{ux6cd5ux5219-1}

你每年需要多少钱生活?

\textbf{年支出 × 25 = 财务自由所需资金}

{\def\LTcaptype{none} % do not increment counter
\begin{longtable}[]{@{}ll@{}}
\toprule\noalign{}
年支出 & 需要的资金 \\
\midrule\noalign{}
\endhead
\bottomrule\noalign{}
\endlastfoot
\$40,000 & \$1,000,000 \\
\$80,000 & \$2,000,000 \\
\$120,000 & \$3,000,000 \\
\$200,000 & \$5,000,000 \\
\end{longtable}
}

\subsubsection{为什么是 25 倍}\label{ux4e3aux4ec0ux4e48ux662f-25-ux500d}

历史数据显示,每年取 4\%,大概率钱不会用完。

这叫\textbf{安全取款率}。

\begin{center}\rule{0.5\linewidth}{0.5pt}\end{center}

\subsection{时间 vs 金钱}\label{ux65f6ux95f4-vs-ux91d1ux94b1}

\subsubsection{问题}\label{ux95eeux9898-3}

{\def\LTcaptype{none} % do not increment counter
\begin{longtable}[]{@{}ll@{}}
\toprule\noalign{}
选项 A & 选项 B \\
\midrule\noalign{}
\endhead
\bottomrule\noalign{}
\endlastfoot
年薪 \$200,000 & 年薪 \$150,000 \\
每周工作 60 小时 & 每周工作 40 小时 \\
通勤 1 小时 & 远程办公 \\
没时间陪孩子 & 每天接孩子放学 \\
\end{longtable}
}

\textbf{哪个更富有?}

很多人选 A。

但 A 的''时薪''其实更低。

\subsubsection{真正的时薪}\label{ux771fux6b63ux7684ux65f6ux85aa}

{\def\LTcaptype{none} % do not increment counter
\begin{longtable}[]{@{}llll@{}}
\toprule\noalign{}
选项 & 年薪 & 实际工作时间 & 真实时薪 \\
\midrule\noalign{}
\endhead
\bottomrule\noalign{}
\endlastfoot
A & \$200,000 & 60h/周 + 10h 通勤 = 3,640h/年 & \$55/h \\
B & \$150,000 & 40h/周 = 2,080h/年 & \$72/h \\
\end{longtable}
}

\textbf{B 的时薪更高。而且有生活。}

\begin{center}\rule{0.5\linewidth}{0.5pt}\end{center}

\subsection{时间的价值}\label{ux65f6ux95f4ux7684ux4ef7ux503c}

你的时间值多少?

\subsubsection{计算方法}\label{ux8ba1ux7b97ux65b9ux6cd5}

你的年薪 ÷ 2,080 = 你的时薪

{\def\LTcaptype{none} % do not increment counter
\begin{longtable}[]{@{}ll@{}}
\toprule\noalign{}
年薪 & 时薪 \\
\midrule\noalign{}
\endhead
\bottomrule\noalign{}
\endlastfoot
\$100,000 & \$48 \\
\$150,000 & \$72 \\
\$200,000 & \$96 \\
\end{longtable}
}

\subsubsection{用时薪做决策}\label{ux7528ux65f6ux85aaux505aux51b3ux7b56}

{\def\LTcaptype{none} % do not increment counter
\begin{longtable}[]{@{}llll@{}}
\toprule\noalign{}
任务 & 时间 & 花多少钱外包? & 决策 \\
\midrule\noalign{}
\endhead
\bottomrule\noalign{}
\endlastfoot
打扫房子 & 3 小时 & \$100 & 外包(省 \$144) \\
自己报税 & 10 小时 & \$500 & 外包(省 \$460) \\
做饭 & 1 小时 & 但我喜欢 & 自己做(不是钱的问题) \\
\end{longtable}
}

\textbf{有些事情花钱比花时间更值得。}

\begin{center}\rule{0.5\linewidth}{0.5pt}\end{center}

\subsection{怎么买回时间}\label{ux600eux4e48ux4e70ux56deux65f6ux95f4}

\subsubsection{1. 花钱买时间}\label{ux82b1ux94b1ux4e70ux65f6ux95f4}

\begin{itemize}
\tightlist
\item
  请清洁阿姨
\item
  用外卖省做饭时间(偶尔)
\item
  住近公司省通勤
\item
  请会计报税
\end{itemize}

\subsubsection{2.
减少消费买时间}\label{ux51cfux5c11ux6d88ux8d39ux4e70ux65f6ux95f4}

\begin{itemize}
\tightlist
\item
  少买东西 = 少花钱 = 可以少工作
\item
  \$50,000 年支出 vs \$100,000 年支出
\item
  前者需要 \$1.25M 退休,后者需要 \$2.5M
\end{itemize}

\textbf{花得越少,越早自由。}

\subsubsection{3. 投资买时间}\label{ux6295ux8d44ux4e70ux65f6ux95f4}

\begin{itemize}
\tightlist
\item
  今天投资 \$1,000
\item
  30 年后变成 \$17,000
\item
  那是几个月的自由
\end{itemize}

\textbf{今天的 \$1,000 = 未来的几个月。}

\begin{center}\rule{0.5\linewidth}{0.5pt}\end{center}

\subsection{一句话}\label{ux4e00ux53e5ux8bdd-22}

\begin{quote}
\textbf{钱是工具,时间才是目的。用钱买时间,不要用时间换钱。}
\end{quote}

\begin{center}\rule{0.5\linewidth}{0.5pt}\end{center}

\subsection{检查清单}\label{ux68c0ux67e5ux6e05ux5355-22}

\begin{itemize}
\tightlist
\item[$\square$]
  我知道我的''真实时薪''
\item[$\square$]
  我用时薪来做花钱决策
\item[$\square$]
  我在用钱买回时间
\item[$\square$]
  我知道我的财务自由数字
\end{itemize}

\begin{center}\rule{0.5\linewidth}{0.5pt}\end{center}

\textbf{下一章}:\href{24-listen-to-mama.md}{听妈妈的话}

\newpage

\section{第二十四章:听妈妈的话}\label{ux7b2cux4e8cux5341ux56dbux7ae0ux542cux5988ux5988ux7684ux8bdd}

\begin{quote}
``妈妈总是对的。''
\end{quote}

\begin{center}\rule{0.5\linewidth}{0.5pt}\end{center}

\subsection{25 年后}\label{ux5e74ux540e}

2025 年,我和 Mike、Sarah 在帕洛阿尔托的咖啡馆。

我们都老了。Mike 头发白了,我腰不太好,Sarah 开始画画了。

``25 年了,'' Mike 说。

``是啊,25 年。''

``你们学到了什么?'' Mike 问。

我想了想。

``听妈妈的话。''

\begin{center}\rule{0.5\linewidth}{0.5pt}\end{center}

\subsection{妈妈的智慧}\label{ux5988ux5988ux7684ux667aux6167}

这 25 年,我犯了很多错。但每次回头看,妈妈早就说过了。

\subsubsection{关于投资}\label{ux5173ux4e8eux6295ux8d44}

\begin{quote}
``鸡蛋不要放一个篮子里。''
\end{quote}

分散投资。Mike 没听,亏了两次。

\begin{quote}
``泡沫就像可乐,喝的时候很爽,打嗝的时候很尴尬。''
\end{quote}

2000 年互联网泡沫,2021 年加密货币,每一代都有自己的泡沫。

\begin{quote}
``如果一个人说的话你听不懂,要么他是天才,要么他是骗子。天才很少。''
\end{quote}

安然、Theranos、FTX,每一个都在说''复杂的创新''。

\subsubsection{关于人生}\label{ux5173ux4e8eux4ebaux751f}

\begin{quote}
``慢慢来,才能快。''
\end{quote}

复利需要时间。职业需要积累。没有捷径。

\begin{quote}
``不是每一场仗都值得打。有时候最聪明的做法是走开。''
\end{quote}

离开 DataSphere 是我做过最正确的决定之一。

\begin{quote}
``熬过冬天的树,根扎得最深。''
\end{quote}

2000 年泡沫、2008 年危机、2020 年疫情,每一次我都活下来了。

\subsubsection{关于常识}\label{ux5173ux4e8eux5e38ux8bc6}

\begin{quote}
``当所有人都说一件事不可能发生的时候,它通常正在发生。''
\end{quote}

``房价不会跌。'' 2008 年跌了。 ``雷曼不会倒。'' 倒了。
``加密货币是未来。'' FTX 没了。

\begin{quote}
``永远不要相信任何包含'永远'这个词的预测。''
\end{quote}

没有什么是永远的。

\begin{center}\rule{0.5\linewidth}{0.5pt}\end{center}

\subsection{常识 vs 聪明}\label{ux5e38ux8bc6-vs-ux806aux660e}

华尔街有很多''聪明人''。

他们有 MBA、博士、复杂的模型。

但他们很多人亏得比普通人还惨。

为什么?

\textbf{因为他们太聪明了,聪明到忘记了常识。}

妈妈没读过大学。她在 Baton Rouge 开餐馆。

但她的建议比任何投资书都有用。

\begin{center}\rule{0.5\linewidth}{0.5pt}\end{center}

\subsection{简单的规则}\label{ux7b80ux5355ux7684ux89c4ux5219}

25 年来,我学到的东西可以总结成几条:

\begin{enumerate}
\def\labelenumi{\arabic{enumi}.}
\tightlist
\item
  \textbf{分散投资} --- 不要把所有鸡蛋放一个篮子
\item
  \textbf{长期持有} --- 复利需要时间
\item
  \textbf{听不懂就不投} --- 能力圈之外的钱不是你的
\item
  \textbf{别人恐惧时贪婪} --- 但首先要有现金
\item
  \textbf{不追涨杀跌} --- 系统比情绪可靠
\item
  \textbf{存钱} --- 真正的安全感来自银行账户
\item
  \textbf{人力资本和金融资本分开} --- 不买公司股票
\item
  \textbf{时间比钱重要} --- 钱是工具,不是目的
\item
  \textbf{教孩子理财} --- 最好的遗产是正确的价值观
\item
  \textbf{听妈妈的话} --- 常识胜过聪明
\end{enumerate}

\begin{center}\rule{0.5\linewidth}{0.5pt}\end{center}

\subsection{最后的故事}\label{ux6700ux540eux7684ux6545ux4e8b}

去年圣诞节,我回了一趟 Baton Rouge。

妈妈的餐馆还在,虽然她已经不亲自掌勺了。

我坐在老位置上,吃了一碗 Gumbo。

Sarah 也在。她专门从旧金山飞来。

``儿子,'' 妈妈问,``这是你女朋友吗?''

我和 Sarah 对视了一眼,都笑了。

``妈妈,我们认识 25 年了。''

``那你还等什么?''

Sarah 脸红了。

我也脸红了。

妈妈说:``\textbf{慢慢来没问题,但也不能慢到老死。}''

\begin{center}\rule{0.5\linewidth}{0.5pt}\end{center}

\subsection{一句话}\label{ux4e00ux53e5ux8bdd-23}

\begin{quote}
\textbf{最复杂的问题,通常有最简单的答案。妈妈早就告诉你了。}
\end{quote}

\begin{center}\rule{0.5\linewidth}{0.5pt}\end{center}

\subsection{检查清单}\label{ux68c0ux67e5ux6e05ux5355-23}

\begin{itemize}
\tightlist
\item[$\square$]
  我遵循了简单的常识
\item[$\square$]
  我没有试图变得太''聪明''
\item[$\square$]
  我在重要的事情上花时间
\item[$\square$]
  我打电话给妈妈了
\end{itemize}

\begin{center}\rule{0.5\linewidth}{0.5pt}\end{center}

\textbf{附录}:\href{appendix-b-rules.md}{阿甘的 10 条铁律}

\newpage

\section{25.
退休不是终点,是转折点}\label{ux9000ux4f11ux4e0dux662fux7ec8ux70b9ux662fux8f6cux6298ux70b9}

\begin{quote}
\textbf{妈妈说:``儿子,退休不是停止工作,是可以选择工作。''}
\end{quote}

\begin{center}\rule{0.5\linewidth}{0.5pt}\end{center}

\subsection{Sarah 的算术课}\label{sarah-ux7684ux7b97ux672fux8bfe}

2020 年,Sarah 48 岁。

``我要退休了。''

``你疯了吗?你才 48。''

``我算过了。''她拿出一张纸:

\begin{verbatim}
我的数学:

Google 股票:够用 15 年
指数基金:够用 10 年
房子:付清了
医保:ACA 搞定
每年开支:$80,000

结论:我可以活到 95 岁。
\end{verbatim}

``可是\ldots 你不怕没事做?''

``\textbf{我要去画画。我等这一天等了 16 年。}''

\begin{center}\rule{0.5\linewidth}{0.5pt}\end{center}

\subsection{退休的三个数字}\label{ux9000ux4f11ux7684ux4e09ux4e2aux6570ux5b57}

\subsubsection{数字一:你需要多少钱?}\label{ux6570ux5b57ux4e00ux4f60ux9700ux8981ux591aux5c11ux94b1}

\textbf{简单公式}:

\begin{verbatim}
退休储蓄 = 年支出 × 25
\end{verbatim}

这叫 \textbf{4\% 法则}------每年从退休储蓄中提取 4\%,理论上可以用 30
年以上。

\textbf{举例}: - 年支出 \$60,000 → 需要 \$1,500,000 - 年支出 \$80,000 →
需要 \$2,000,000 - 年支出 \$100,000 → 需要 \$2,500,000

妈妈说: \textgreater{}
``儿子,这只是理论。实际上,你不会花一样多的钱。60 岁和 80
岁,开支完全不同。''

\begin{center}\rule{0.5\linewidth}{0.5pt}\end{center}

\subsubsection{数字二:你还有多少时间?}\label{ux6570ux5b57ux4e8cux4f60ux8fd8ux6709ux591aux5c11ux65f6ux95f4}

\textbf{复利的魔法}:

\begin{verbatim}
25 岁开始 vs 35 岁开始

假设每月存 $500,年回报 7%:

25 岁开始 → 65 岁有 $1,200,000
35 岁开始 → 65 岁有 $567,000
\end{verbatim}

\textbf{10 年的差距 = 63 万美元}

Sarah 说: \textgreater{} ``复利是时间的函数。开始得越早,终点越远。''

\begin{center}\rule{0.5\linewidth}{0.5pt}\end{center}

\subsubsection{数字三:你的收入来源有几个?}\label{ux6570ux5b57ux4e09ux4f60ux7684ux6536ux5165ux6765ux6e90ux6709ux51e0ux4e2a}

\textbf{理想的退休收入来源}:

\begin{verbatim}
1. 社保(Social Security)  —— 基础保障
2. 退休账户(401k/IRA)    —— 税务优惠
3. 投资收入(股息/利息)    —— 被动收入
4. 兼职/顾问                —— 如果你想
\end{verbatim}

\textbf{Mike 的教训}: \textgreater{} ``我所有的钱都在 Enron
股票里。退休金、401k、个人账户。 \textgreater{} 我以为我是百万富翁。
\textgreater{} 三个月后,我一无所有。''

\textbf{规则}:退休收入至少要有 2-3 个独立来源。

\begin{center}\rule{0.5\linewidth}{0.5pt}\end{center}

\subsection{退休账户:硅谷人的税务武器}\label{ux9000ux4f11ux8d26ux6237ux7845ux8c37ux4ebaux7684ux7a0eux52a1ux6b66ux5668}

\subsubsection{401(k):公司帮你存}\label{kux516cux53f8ux5e2eux4f60ux5b58}

\begin{verbatim}
2024 年限额:$23,000(50 岁以上 +$7,500)

优点:
✓ 税前存入,当年省税
✓ 公司 Match(免费的钱!)
✓ 自动扣款,无需纪律

缺点:
✗ 59.5 岁前取出要罚款
✗ 取出时按收入纳税
✗ 投资选择有限
\end{verbatim}

\textbf{妈妈的命令}: \textgreater{} ``儿子,401k 至少存到公司 Match
的上限。不存就是白送钱给政府。''

\begin{center}\rule{0.5\linewidth}{0.5pt}\end{center}

\subsubsection{Roth
IRA:年轻人的最佳选择}\label{roth-iraux5e74ux8f7bux4ebaux7684ux6700ux4f73ux9009ux62e9}

\begin{verbatim}
2024 年限额:$7,000(50 岁以上 +$1,000)

优点:
✓ 税后存入,增长免税
✓ 取出时完全免税
✓ 退休后灵活度高
✓ 没有强制取出要求

缺点:
✗ 收入太高不能存($161k 单身 / $240k 夫妻)
✗ 限额较低
\end{verbatim}

\textbf{Sarah 的建议}: \textgreater{} ``年轻时收入低,税率低。用 Roth
IRA。 \textgreater{} 老了收入高,税率高。用 Traditional 401k。
\textgreater{} 这叫税务套利。''

\begin{center}\rule{0.5\linewidth}{0.5pt}\end{center}

\subsubsection{Backdoor
Roth:高收入者的后门}\label{backdoor-rothux9ad8ux6536ux5165ux8005ux7684ux540eux95e8}

如果收入超过 Roth IRA 限额:

\begin{verbatim}
步骤:
1. 存入 Traditional IRA(非抵税)
2. 立即转换成 Roth IRA
3. 缴少量转换税
4. 享受终身免税增长
\end{verbatim}

\textbf{注意}:要咨询税务专家,有 Pro Rata 规则。

\begin{center}\rule{0.5\linewidth}{0.5pt}\end{center}

\subsubsection{Peter Thiel 的 Roth IRA
神话}\label{peter-thiel-ux7684-roth-ira-ux795eux8bdd}

2021 年,ProPublica 爆出一个惊天秘密:

\textbf{Peter Thiel 的 Roth IRA 账户里有 50 亿美元。}

``等等,''Mike 说,``Roth IRA 每年限额才几千块,怎么可能变成 50 亿?''

\textbf{答案:他买了自己公司的股票。}

\begin{verbatim}
Peter Thiel 的操作(1999 年):

1. 开设 Roth IRA 账户
2. 用账户买入 PayPal 创始人股份
3. 价格:每股 $0.001(是的,0.1 美分)
4. 买入数量:170 万股
5. 总成本:$1,700(在年度限额内)

然后...

2002 年:eBay 以 $15 亿收购 PayPal
Peter Thiel 的股份价值:数千万美元

继续...

用 Roth IRA 投资 Facebook(2004 年)
再投资 Palantir
再投资其他创业公司

2021 年:账户价值 $50 亿
全部免税。
\end{verbatim}

\begin{center}\rule{0.5\linewidth}{0.5pt}\end{center}

\textbf{Sarah 的分析}:

\begin{quote}
``这合法吗?技术上是合法的。

但普通人能复制吗?不能。

关键是他能用 \$0.001 买到自己创办的公司股票。 你我去买
PayPal,要按市场价。

这不是投资技巧,这是创始人特权。''
\end{quote}

\begin{center}\rule{0.5\linewidth}{0.5pt}\end{center}

\textbf{IRS 后来的反应}:

\begin{verbatim}
2021 年后,国会开始调查:

拟议的限制(尚未通过):
- Roth IRA 超过 $10M 部分强制取出
- 禁止买非公开交易的股票
- 禁止买自己公司的股票

现状(2024 年):
- 这些限制还没通过
- 但 IRS 开始严查"低估值"交易
- 创始人股份的估值必须合理
\end{verbatim}

\begin{center}\rule{0.5\linewidth}{0.5pt}\end{center}

\textbf{妈妈的评论}:

\begin{quote}
``儿子,Peter Thiel 是天才,但他的方法你学不来。

与其羡慕别人的 50 亿, 不如老老实实每年存满 \$7,000。

40 年后,你也有 100 万。 100 万免税,也很香了。''
\end{quote}

\begin{center}\rule{0.5\linewidth}{0.5pt}\end{center}

\textbf{阿甘的 Roth IRA 策略}:

\begin{verbatim}
普通人的正确姿势:

1. 每年存满限额($7,000)
2. 投指数基金(不要瞎折腾)
3. 40 年不动(复利做功)
4. 退休后免税取出

预期结果(7% 年回报):
25 岁开始 → 65 岁有 $1,500,000
全部免税。

不是 50 亿,但足够了。
\end{verbatim}

\begin{center}\rule{0.5\linewidth}{0.5pt}\end{center}

\subsection{退休的四个阶段}\label{ux9000ux4f11ux7684ux56dbux4e2aux9636ux6bb5}

\subsubsection{第一阶段:积累期(25-45
岁)}\label{ux7b2cux4e00ux9636ux6bb5ux79efux7d2fux671f25-45-ux5c81}

\textbf{目标}:存得越多越好

\begin{verbatim}
策略:
- 最大化 401k + Roth IRA
- 投资主要放在股票
- 承受波动,追求增长
- 不要碰退休账户
\end{verbatim}

\begin{center}\rule{0.5\linewidth}{0.5pt}\end{center}

\subsubsection{第二阶段:过渡期(45-55
岁)}\label{ux7b2cux4e8cux9636ux6bb5ux8fc7ux6e21ux671f45-55-ux5c81}

\textbf{目标}:降低风险,保护本金

\begin{verbatim}
策略:
- 逐步从股票转向债券
- 开始规划退休后的生活
- 计算你的"数字"
- 考虑提前退休的可能
\end{verbatim}

Sarah 在 45 岁时: \textgreater{} ``我开始认真算了。股票从 80\% 降到
60\%。债券从 10\% 升到 30\%。现金保持 10\%。''

\begin{center}\rule{0.5\linewidth}{0.5pt}\end{center}

\subsubsection{第三阶段:退休初期(55-70
岁)}\label{ux7b2cux4e09ux9636ux6bb5ux9000ux4f11ux521dux671f55-70-ux5c81}

\textbf{目标}:活跃、自由、有选择

\begin{verbatim}
策略:
- 健康是最大的投资
- 可以做自己想做的事
- 支出可能比工作时更高(旅行、爱好)
- 保持一定的股票配置(对抗通胀)
\end{verbatim}

\begin{center}\rule{0.5\linewidth}{0.5pt}\end{center}

\subsubsection{第四阶段:退休后期(70
岁以后)}\label{ux7b2cux56dbux9636ux6bb5ux9000ux4f11ux540eux671f70-ux5c81ux4ee5ux540e}

\textbf{目标}:简化、安心、传承

\begin{verbatim}
策略:
- 支出通常下降
- 医疗费用上升
- 考虑遗产规划
- 简化投资组合
\end{verbatim}

妈妈说: \textgreater{}
``儿子,到了这个年纪,钱不是最重要的了。重要的是有人陪你吃饭。''

\begin{center}\rule{0.5\linewidth}{0.5pt}\end{center}

\subsection{提前退休?FIRE
运动}\label{ux63d0ux524dux9000ux4f11fire-ux8fd0ux52a8}

\textbf{FIRE = Financial Independence, Retire Early}

\begin{verbatim}
核心公式:
储蓄率决定退休年限

储蓄率 10% → 需要工作 51 年
储蓄率 25% → 需要工作 32 年
储蓄率 50% → 需要工作 17 年
储蓄率 75% → 需要工作 7 年
\end{verbatim}

\textbf{Sarah 的看法}: \textgreater{} ``FIRE 很好,但不是每个人都适合。
\textgreater{} 有些人需要工作的意义。 \textgreater{}
有些人需要同事的社交。 \textgreater{} 财务自由只是工具,不是目的。''

\begin{center}\rule{0.5\linewidth}{0.5pt}\end{center}

\subsection{比特币披萨:史上最贵的一餐}\label{ux6bd4ux7279ux5e01ux62abux8428ux53f2ux4e0aux6700ux8d35ux7684ux4e00ux9910}

2010 年 5 月 22 日,程序员 Laszlo Hanyecz 在 Bitcoin Talk 论坛发帖:

\begin{quote}
``我愿意用 10,000 个比特币换两个披萨。''
\end{quote}

有人接单了。两个 Papa John's 披萨,价值 \$41。

\textbf{那 10,000 个比特币今天值多少?}

\begin{verbatim}
比特币披萨的代价:

2010 年:10,000 BTC = $41(两个披萨)
2013 年:10,000 BTC = $1,000,000
2017 年:10,000 BTC = $200,000,000
2021 年:10,000 BTC = $690,000,000
2024 年:10,000 BTC = $1,000,000,000

这是人类历史上最贵的披萨。
\end{verbatim}

\begin{center}\rule{0.5\linewidth}{0.5pt}\end{center}

\textbf{Mike 听完这个故事}:

``所以我应该买比特币?''

Sarah 说:

\begin{quote}
``你错过了重点。

2010 年,Laszlo 有 10,000 个比特币。 2024 年,他还有比特币吗?

\textbf{答案是:有。他还持有一些。}

但大多数早期持有者? 在 \$100 卖了。 在 \$1,000 卖了。 在 \$10,000
卖了。

\textbf{买对不难,拿住才难。}''
\end{quote}

\begin{center}\rule{0.5\linewidth}{0.5pt}\end{center}

\textbf{妈妈的评论}:

\begin{quote}
``儿子,每个人都会讲'如果当年我买了 XXX'的故事。

但没人告诉你: - 当年买了的人,99\% 早就卖了 - 拿住 14
年不动的人,万中无一 - 那些讲故事的人,自己也没买

\textbf{与其后悔没买比特币,不如想想你现在拿着什么能拿 14 年。}''
\end{quote}

\begin{center}\rule{0.5\linewidth}{0.5pt}\end{center}

\textbf{阿甘的教训}:

\begin{verbatim}
比特币披萨的真正教训:

1. 早买不如拿得住
2. 涨 10 倍你会卖,涨 100 倍你早跑了
3. 真正的财富来自时间,不是时机
4. 指数基金涨得慢,但你真的能拿 40 年
\end{verbatim}

\begin{center}\rule{0.5\linewidth}{0.5pt}\end{center}

\subsection{中国资金大逃亡:谁推高了湾区房价}\label{ux4e2dux56fdux8d44ux91d1ux5927ux9003ux4ea1ux8c01ux63a8ux9ad8ux4e86ux6e7eux533aux623fux4ef7}

2012 年,Sarah 想在 Palo Alto 买房。

Open House 那天,她看到一个现象:

\begin{quote}
``房子标价 \$1,200,000。 来了 30 组人看房。 其中 20 组是中国人。
最后成交价:\$1,680,000。 \textbf{全现金,无贷款。}''
\end{quote}

\begin{center}\rule{0.5\linewidth}{0.5pt}\end{center}

\subsubsection{2010-2016:中国热钱涌入}\label{ux4e2dux56fdux70edux94b1ux6d8cux5165}

\begin{verbatim}
背景:

2010 年:中国 GDP 超越日本,成为世界第二
2012 年:习近平上台,反腐开始
2013 年:中国富人开始"资产出海"
2015 年:人民币贬值,资金加速外逃

目的地:
- 温哥华(太贵了)
- 洛杉矶(华人太多)
- 湾区(孩子要读书 + 科技圈)

特别是:Palo Alto、Los Altos、Cupertino
为什么?学区。
\end{verbatim}

\begin{center}\rule{0.5\linewidth}{0.5pt}\end{center}

\subsubsection{为什么是 Palo Alto 和 Los
Altos?}\label{ux4e3aux4ec0ux4e48ux662f-palo-alto-ux548c-los-altos}

\textbf{Mike 的观察}:

\begin{quote}
``2014 年,我去 Los Altos 看房。

一栋 1960 年代的老房子,3000 sqft。 标价 \$2,800,000。

我问经纪人:`这房子值这个价吗?'

她说:`你不是在买房子,你是在买学区。 \textbf{Los Altos High School,API
950+。 这个学区的房子,永远有人买。}'\,''
\end{quote}

\begin{center}\rule{0.5\linewidth}{0.5pt}\end{center}

\textbf{中国买家的逻辑}:

\begin{verbatim}
中国家长的算法:

1. 孩子要上好大学
2. 好大学要好高中
3. 好高中要好学区
4. 好学区 = Palo Alto / Los Altos / Cupertino

价格?不重要。
贷款?不需要。
房子旧?可以推倒重建。

目标:让孩子进 Stanford / Berkeley / MIT
\end{verbatim}

\begin{center}\rule{0.5\linewidth}{0.5pt}\end{center}

\subsubsection{房价的疯狂年代}\label{ux623fux4ef7ux7684ux75afux72c2ux5e74ux4ee3}

\begin{verbatim}
Palo Alto 房价中位数:

2010 年:$1,200,000
2012 年:$1,500,000
2014 年:$2,200,000
2016 年:$2,800,000
2018 年:$3,200,000

8 年涨了 167%。
同期标普 500 涨了 150%。

但股票不能让你的孩子上好学校。
\end{verbatim}

\begin{center}\rule{0.5\linewidth}{0.5pt}\end{center}

\textbf{Sarah 的分析}:

\begin{quote}
``中国资金改变了湾区房产的游戏规则。

以前:房价 = 收入 × 合理倍数 现在:房价 = 全球资金竞价

你用贷款,人家全现金。 你按收入买,人家按资产买。
你算投资回报,人家算孩子前途。

\textbf{不同的游戏,不同的规则。}''
\end{quote}

\begin{center}\rule{0.5\linewidth}{0.5pt}\end{center}

\subsubsection{2017
年后:潮水退去}\label{ux5e74ux540eux6f6eux6c34ux9000ux53bb}

\begin{verbatim}
转折点:

2017 年:中国加强外汇管制,每人每年 $50,000 限额
2018 年:中美贸易战开始
2020 年:疫情,中国买家无法来看房
2022 年:利率暴涨,房价终于停滞

结果:
- 中国买家大幅减少
- 但房价没跌多少
- 因为库存也少了
- 老业主有 Prop 13 保护,不想卖
\end{verbatim}

\begin{center}\rule{0.5\linewidth}{0.5pt}\end{center}

\textbf{David 的感慨}:

\begin{quote}
``我 1995 年花 \$300,000 买的房子, 2015 年值 \$2,500,000。

我以为是我投资眼光好。 其实是中国人帮我抬的价。

\textbf{我不是股神,我只是运气好。}''
\end{quote}

\begin{center}\rule{0.5\linewidth}{0.5pt}\end{center}

\textbf{妈妈的总结}:

\begin{quote}
``儿子,房价这东西,短期看资金,长期看人口。

中国资金来了,房价涨。 中国资金走了,房价稳。

但硅谷的工作机会还在。 好学区的吸引力还在。 全世界想来美国的人还在。

\textbf{你买不起?正常。这不是给工薪阶层准备的游戏。}''
\end{quote}

\begin{center}\rule{0.5\linewidth}{0.5pt}\end{center}

\subsection{加州房产:退休的隐形炸弹}\label{ux52a0ux5ddeux623fux4ea7ux9000ux4f11ux7684ux9690ux5f62ux70b8ux5f39}

2022 年,Mike 的邻居 David 要退休了。

``我 1995 年买的房子,\$300,000。现在值 \$2,500,000。''

``恭喜啊!''

``\textbf{恭喜什么?我想卖掉搬去德州,但联邦资本利得税要交
\$440,000。}''

这就是加州退休的第一课:\textbf{你的房子可能是金手铐}。

\begin{center}\rule{0.5\linewidth}{0.5pt}\end{center}

\subsubsection{Prop
13:加州最伟大的法案}\label{prop-13ux52a0ux5ddeux6700ux4f1fux5927ux7684ux6cd5ux6848}

\textbf{1978 年的规则}:

\begin{verbatim}
Prop 13 核心:
- 房产税基于购买价,不是市场价
- 每年最多涨 2%
- 房子不卖,税基不变
\end{verbatim}

\textbf{David 的例子}:

\begin{verbatim}
1995 年买入:$300,000
2024 年市值:$2,500,000

按 Prop 13:每年房产税 ≈ $5,000(基于调整后的购买价 ~$420,000)
按市价算:每年房产税 ≈ $30,000

每年省:$25,000
\end{verbatim}

妈妈说: \textgreater{}
``儿子,在加州买房,就像买了终身打折卡。但你一卖,折扣就没了。''

\begin{center}\rule{0.5\linewidth}{0.5pt}\end{center}

\subsubsection{1031
Exchange:延税的艺术}\label{exchangeux5ef6ux7a0eux7684ux827aux672f}

\textbf{联邦税法第 1031 条}:

\begin{verbatim}
规则:
- 卖投资房产,买同类房产
- 资本利得税可以延迟
- 不是免税,是递延

时间要求:
- 45 天内确定新房产
- 180 天内完成交易

限制:
- 只适用于投资房产
- 不适用于自住房(有其他免税额)
- 必须"同类交换"(Like-Kind)
\end{verbatim}

\textbf{David 的想法}: \textgreater{} ``我能不能用 1031 把 Palo Alto
的房子换成德州的房子?''

\textbf{答案}:可以。但你会失去 Prop 13
的保护。而且德州虽然没有州税,但房产税是加州的 3 倍。

\begin{center}\rule{0.5\linewidth}{0.5pt}\end{center}

\subsubsection{Prop 19 陷阱:2020
年的糖衣毒药}\label{prop-19-ux9677ux96312020-ux5e74ux7684ux7cd6ux8863ux6bd2ux836f}

\textbf{表面的''好处''}:

\begin{verbatim}
Prop 19(2020 年生效):

"好处":
- 55 岁以上可以带税基搬家(全州范围)
- 残疾人、灾民也可以
- 最多可用 3 次

听起来不错?等等...
\end{verbatim}

\textbf{真正的陷阱}:

\begin{verbatim}
Prop 19 的代价:

旧规则(Prop 58):
- 父母房产传给子女,税基不变
- 子女继承 $300,000 的税基
- 继续享受低房产税

新规则(Prop 19):
- 继承后如果不自住,重新评估市值
- 子女继承 $2,500,000 市值的房子
- 房产税从 $5,000 跳到 $30,000
- 除非子女搬进去住
\end{verbatim}

\textbf{Mike 的分析}: \textgreater{} ``Prop 19 就是政府的把戏。
\textgreater{} 表面上给老人搬家的便利。 \textgreater{}
实际上取消了子女继承的税基保护。 \textgreater{}
加州政府每年多收几十亿房产税。''

\begin{center}\rule{0.5\linewidth}{0.5pt}\end{center}

\subsubsection{退休房产策略:三条路}\label{ux9000ux4f11ux623fux4ea7ux7b56ux7565ux4e09ux6761ux8def}

\textbf{路线一:死守加州}

\begin{verbatim}
适合:
- 喜欢加州气候
- 子女在本地
- 房子已经付清

策略:
- 继续享受 Prop 13
- 用房子做 HELOC 应急
- 让子女继承(如果他们愿意住)
\end{verbatim}

\textbf{路线二:1031 换房}

\begin{verbatim}
适合:
- 有投资房产
- 想换到其他州
- 可以接受管理新房产

策略:
- 投资房用 1031 延税
- 自住房用 $250k/$500k 免税额
- 分批处理,分散税务冲击
\end{verbatim}

\textbf{路线三:直接卖掉}

\begin{verbatim}
适合:
- 想彻底简化
- 不在乎税
- 子女不需要房子

策略:
- 卖房,交税,拿现金
- 搬到低税州租房或买便宜房子
- 用省下的生活费覆盖税

算账:
自住房卖 $2,500,000
成本基础 $300,000
增值 $2,200,000
夫妻免税额 -$500,000
应税增值 $1,700,000
联邦税(20%) $340,000
加州税(13.3%) $226,000
总税 ~$566,000

听起来很多?但如果你 75 岁,
剩下的 $1,934,000 现金,
搬到德州每年省 $25,000 房产税,
活到 90 岁省 $375,000。
\end{verbatim}

\begin{center}\rule{0.5\linewidth}{0.5pt}\end{center}

\subsubsection{David
的最终选择}\label{david-ux7684ux6700ux7ec8ux9009ux62e9}

2023 年,David 65 岁。

他算了三个月。

最后他说:

\begin{quote}
``\textbf{我不搬了。}

这房子我住了 30 年。 我老婆的骨灰撒在后院。 我女儿每周来吃晚饭。

Prop 13 每年给我省 \$25,000。 这笔钱够我活得很好。

我死后,让孩子们自己决定。 要么住进来,要么卖掉交税。
那是他们的问题,不是我的。''
\end{quote}

妈妈说: \textgreater{} ``儿子,有时候最好的税务策略,是根本不卖。''

\begin{center}\rule{0.5\linewidth}{0.5pt}\end{center}

\subsection{退休的真正意义}\label{ux9000ux4f11ux7684ux771fux6b63ux610fux4e49}

2024 年,Sarah 退休两年后。

我问她:``退休后最大的发现是什么?''

她说:

\begin{quote}
``\textbf{时间比钱贵。}

我以前以为退休是关于钱的。 现在我知道,退休是关于时间的。

钱可以再赚,时间不能。

我现在每天画画四小时,比在 Google 开会快乐一百倍。''
\end{quote}

\begin{center}\rule{0.5\linewidth}{0.5pt}\end{center}

\subsection{阿甘的退休清单}\label{ux963fux7518ux7684ux9000ux4f11ux6e05ux5355}

\begin{verbatim}
退休账户:
□ 401k 至少存到公司 Match 上限
□ 开设 Roth IRA 或 Backdoor Roth
□ 了解你的退休"数字"(年支出 × 25)
□ 退休收入至少 2-3 个来源

投资配置:
□ 股债配置随年龄调整
□ 不要把退休金全押一个股票
□ 指数基金为核心,长期持有

加州房产(如适用):
□ 理解 Prop 13 的价值
□ 了解 Prop 19 对继承的影响
□ 考虑 1031 Exchange 的利弊
□ 计算卖房 vs 持有的税务成本

心态准备:
□ 想清楚退休后要做什么
□ 买对不难,拿住才难
□ 时间比钱贵
\end{verbatim}

\begin{center}\rule{0.5\linewidth}{0.5pt}\end{center}

\subsection{一句话总结}\label{ux4e00ux53e5ux8bddux603bux7ed3}

\begin{quote}
\textbf{妈妈说:``儿子,退休不是活够了。是终于可以好好活了。''}
\end{quote}

\begin{center}\rule{0.5\linewidth}{0.5pt}\end{center}

\subsection{进一步阅读}\label{ux8fdbux4e00ux6b65ux9605ux8bfb}

\begin{itemize}
\tightlist
\item
  \href{01-never-all-in.md}{第 1 章:永远不要全押一个篮子} - 分散投资
\item
  \href{05-index-funds.md}{第 5 章:不知道买什么就买指数} - 简单策略
\item
  \href{22-compound-time.md}{第 22 章:复利需要时间} - 耐心
\item
  \href{23-real-wealth.md}{第 23 章:真正的财富是时间} - 自由
\end{itemize}

\begin{center}\rule{0.5\linewidth}{0.5pt}\end{center}

\textbf{上一章}:\href{24-listen-to-mama.md}{第 24 章:听妈妈的话}
\textbf{下一章}:\href{26-healthcare.md}{第 26
章:医保是隐形的财富}(即将推出)

\begin{center}\rule{0.5\linewidth}{0.5pt}\end{center}

\textbf{版本}:v0.1 \textbf{更新日期}:2025-12-30

\newpage

\section{附录
A:投资决策检查清单}\label{ux9644ux5f55-aux6295ux8d44ux51b3ux7b56ux68c0ux67e5ux6e05ux5355}

\begin{quote}
每次投资前,过一遍这个清单。
\end{quote}

\begin{center}\rule{0.5\linewidth}{0.5pt}\end{center}

\subsection{投资前检查}\label{ux6295ux8d44ux524dux68c0ux67e5}

\subsubsection{1. 基础条件 ✓}\label{ux57faux7840ux6761ux4ef6}

\begin{itemize}
\tightlist
\item[$\square$]
  我有 6 个月以上的紧急备用金
\item[$\square$]
  我没有高息债务(信用卡等)
\item[$\square$]
  这是''闲钱'',5 年内不需要用
\item[$\square$]
  我设置了自动定投
\end{itemize}

\subsubsection{2. 理解检查 ✓}\label{ux7406ux89e3ux68c0ux67e5}

\begin{itemize}
\tightlist
\item[$\square$]
  我能用三句话解释这个投资怎么赚钱
\item[$\square$]
  我知道如果错了会亏多少
\item[$\square$]
  我不是因为''别人都在买''而买
\item[$\square$]
  我没有被''复杂''的术语吓住
\end{itemize}

\subsubsection{3. 情绪检查 ✓}\label{ux60c5ux7eeaux68c0ux67e5}

\begin{itemize}
\tightlist
\item[$\square$]
  我不是在恐慌中卖出
\item[$\square$]
  我不是在狂热中买入
\item[$\square$]
  这是计划内的决策,不是冲动
\item[$\square$]
  我没有 FOMO(错过恐惧)
\end{itemize}

\subsubsection{4. 分散检查 ✓}\label{ux5206ux6563ux68c0ux67e5}

\begin{itemize}
\tightlist
\item[$\square$]
  这笔投资不超过总资产的 10\%
\item[$\square$]
  我没有买自己公司的股票(或很少)
\item[$\square$]
  我的投资组合是分散的
\item[$\square$]
  我有股票、债券、现金的配置
\end{itemize}

\begin{center}\rule{0.5\linewidth}{0.5pt}\end{center}

\subsection{买入检查}\label{ux4e70ux5165ux68c0ux67e5}

在点击''买入''之前:

{\def\LTcaptype{none} % do not increment counter
\begin{longtable}[]{@{}ll@{}}
\toprule\noalign{}
问题 & 答案 \\
\midrule\noalign{}
\endhead
\bottomrule\noalign{}
\endlastfoot
为什么买? & \_\_\_\_\_\_\_\_\_\_\_\_\_\_\_\_\_ \\
持有多久? & \_\_\_\_\_\_\_\_\_\_\_\_\_\_\_\_\_ \\
跌多少会卖? & \_\_\_\_\_\_\_\_\_\_\_\_\_\_\_\_\_ \\
涨多少会卖? & \_\_\_\_\_\_\_\_\_\_\_\_\_\_\_\_\_ \\
这是投资还是赌博? & \_\_\_\_\_\_\_\_\_\_\_\_\_\_\_\_\_ \\
\end{longtable}
}

\begin{center}\rule{0.5\linewidth}{0.5pt}\end{center}

\subsection{卖出检查}\label{ux5356ux51faux68c0ux67e5}

在点击''卖出''之前:

{\def\LTcaptype{none} % do not increment counter
\begin{longtable}[]{@{}ll@{}}
\toprule\noalign{}
问题 & 如果是''是'',可能不该卖 \\
\midrule\noalign{}
\endhead
\bottomrule\noalign{}
\endlastfoot
我是因为价格跌了而卖? & ⚠️ \\
我是因为新闻吓到了而卖? & ⚠️ \\
我是因为想落袋为安而卖? & ⚠️ \\
我是因为朋友都在卖而卖? & ⚠️ \\
\end{longtable}
}

\textbf{合理的卖出理由:} - {[} {]} 我需要用这笔钱 - {[} {]}
投资逻辑发生了根本变化 - {[} {]} 有更好的投资机会

\begin{center}\rule{0.5\linewidth}{0.5pt}\end{center}

\subsection{年度检查}\label{ux5e74ux5ea6ux68c0ux67e5}

每年做一次:

\subsubsection{资产配置}\label{ux8d44ux4ea7ux914dux7f6e}

{\def\LTcaptype{none} % do not increment counter
\begin{longtable}[]{@{}llll@{}}
\toprule\noalign{}
类别 & 目标占比 & 实际占比 & 需要调整 \\
\midrule\noalign{}
\endhead
\bottomrule\noalign{}
\endlastfoot
股票 & \_\_\_\% & \_\_\_\% & ☐ \\
债券 & \_\_\_\% & \_\_\_\% & ☐ \\
现金 & \_\_\_\% & \_\_\_\% & ☐ \\
其他 & \_\_\_\% & \_\_\_\% & ☐ \\
\end{longtable}
}

\subsubsection{费用检查}\label{ux8d39ux7528ux68c0ux67e5}

{\def\LTcaptype{none} % do not increment counter
\begin{longtable}[]{@{}lll@{}}
\toprule\noalign{}
账户 & 年费率 & 是否 \textless{} 0.5\%? \\
\midrule\noalign{}
\endhead
\bottomrule\noalign{}
\endlastfoot
\_\_\_\_\_\_\_ & \_\_\_\% & ☐ \\
\_\_\_\_\_\_\_ & \_\_\_\% & ☐ \\
\_\_\_\_\_\_\_ & \_\_\_\% & ☐ \\
\end{longtable}
}

\subsubsection{安全检查}\label{ux5b89ux5168ux68c0ux67e5}

\begin{itemize}
\tightlist
\item[$\square$]
  紧急备用金还在
\item[$\square$]
  保险还有效
\item[$\square$]
  受益人信息是最新的
\item[$\square$]
  密码和 2FA 设置好了
\end{itemize}

\begin{center}\rule{0.5\linewidth}{0.5pt}\end{center}

\subsection{危机时检查}\label{ux5371ux673aux65f6ux68c0ux67e5}

当市场大跌(\textgreater{} 20\%)时:

\begin{enumerate}
\def\labelenumi{\arabic{enumi}.}
\tightlist
\item
  \textbf{不要打开账户看数字}
\item
  \textbf{不要看财经新闻}
\item
  \textbf{问自己这些问题:}
\end{enumerate}

{\def\LTcaptype{none} % do not increment counter
\begin{longtable}[]{@{}ll@{}}
\toprule\noalign{}
问题 & 答案 \\
\midrule\noalign{}
\endhead
\bottomrule\noalign{}
\endlastfoot
这是暂时的恐慌还是真的出问题了? & \_\_\_\_\_\_\_\_\_\_\_\_\_\_\_\_\_ \\
我持有的公司还在赚钱吗? & \_\_\_\_\_\_\_\_\_\_\_\_\_\_\_\_\_ \\
我有现金可以加仓吗? & \_\_\_\_\_\_\_\_\_\_\_\_\_\_\_\_\_ \\
我能承受再跌 30\% 吗? & \_\_\_\_\_\_\_\_\_\_\_\_\_\_\_\_\_ \\
\end{longtable}
}

\textbf{如果公司基本面没变,不要卖。}

\begin{center}\rule{0.5\linewidth}{0.5pt}\end{center}

\subsection{打印这页}\label{ux6253ux5370ux8fd9ux9875}

把这个检查清单打印出来。

贴在你的电脑旁边。

每次做投资决策前,过一遍。

\textbf{纪律 \textgreater{} 情绪。}

\begin{center}\rule{0.5\linewidth}{0.5pt}\end{center}

\textbf{返回}:\href{00-index.md}{目录}

\newpage

\section{附录 B:阿甘的 10
条铁律}\label{ux9644ux5f55-bux963fux7518ux7684-10-ux6761ux94c1ux5f8b}

\begin{quote}
``25 年的智慧,浓缩成 10 条。''
\end{quote}

\begin{center}\rule{0.5\linewidth}{0.5pt}\end{center}

\subsection{铁律一:分散投资}\label{ux94c1ux5f8bux4e00ux5206ux6563ux6295ux8d44}

\begin{quote}
\textbf{不要把所有鸡蛋放在一个篮子里,尤其是你工作的那个篮子。}
\end{quote}

\begin{itemize}
\tightlist
\item
  单只股票不超过 10\%
\item
  不买自己公司的股票
\item
  股票 + 债券 + 现金
\end{itemize}

\textbf{Mike 犯了两次错。你不要犯。}

\begin{center}\rule{0.5\linewidth}{0.5pt}\end{center}

\subsection{铁律二:安全第一}\label{ux94c1ux5f8bux4e8cux5b89ux5168ux7b2cux4e00}

\begin{quote}
\textbf{先存够六个月生活费,再想投资的事。}
\end{quote}

\begin{itemize}
\tightlist
\item
  紧急备用金放在随时能取的地方
\item
  不用这笔钱投资
\item
  这是你的安全垫
\end{itemize}

\textbf{没有安全垫,不要上战场。}

\begin{center}\rule{0.5\linewidth}{0.5pt}\end{center}

\subsection{铁律三:能力圈}\label{ux94c1ux5f8bux4e09ux80fdux529bux5708}

\begin{quote}
\textbf{听不懂就不要投。}
\end{quote}

\begin{itemize}
\tightlist
\item
  用三句话解释不清楚,不买
\item
  太复杂通常意味着太危险
\item
  大佬投资了 ≠ 安全
\end{itemize}

\textbf{安然、Theranos、FTX,都是''创新''。}

\begin{center}\rule{0.5\linewidth}{0.5pt}\end{center}

\subsection{铁律四:逆向思维}\label{ux94c1ux5f8bux56dbux9006ux5411ux601dux7ef4}

\begin{quote}
\textbf{别人恐惧时贪婪,别人贪婪时恐惧。}
\end{quote}

\begin{itemize}
\tightlist
\item
  市场暴跌时,考虑加仓
\item
  市场狂热时,保持警惕
\item
  但首先,你要有现金
\end{itemize}

\textbf{2008 年买入的人,赚了 4 倍。}

\begin{center}\rule{0.5\linewidth}{0.5pt}\end{center}

\subsection{铁律五:简单为王}\label{ux94c1ux5f8bux4e94ux7b80ux5355ux4e3aux738b}

\begin{quote}
\textbf{不知道买什么,就买指数基金。}
\end{quote}

\begin{itemize}
\tightlist
\item
  VOO / VTI / SPY
\item
  费率 \textless{} 0.1\%
\item
  买入,持有,不动
\end{itemize}

\textbf{无聊才能赚钱。}

\begin{center}\rule{0.5\linewidth}{0.5pt}\end{center}

\subsection{铁律六:自动化}\label{ux94c1ux5f8bux516dux81eaux52a8ux5316}

\begin{quote}
\textbf{把决策权从情绪手里拿走。}
\end{quote}

\begin{itemize}
\tightlist
\item
  每月定投,自动执行
\item
  不看新闻,不择时
\item
  设好规则,然后遵守
\end{itemize}

\textbf{系统 \textgreater{} 冲动。}

\begin{center}\rule{0.5\linewidth}{0.5pt}\end{center}

\subsection{铁律七:持有是关键}\label{ux94c1ux5f8bux4e03ux6301ux6709ux662fux5173ux952e}

\begin{quote}
\textbf{会买的是徒弟,会持有的才是师傅。}
\end{quote}

\begin{itemize}
\tightlist
\item
  不要因为跌了就卖
\item
  不要因为涨了就跑
\item
  复利需要时间
\end{itemize}

\textbf{16 年持有,4 倍回报。}

\begin{center}\rule{0.5\linewidth}{0.5pt}\end{center}

\subsection{铁律八:Plan B}\label{ux94c1ux5f8bux516bplan-b}

\begin{quote}
\textbf{永远不要把命运完全交给别人。}
\end{quote}

\begin{itemize}
\tightlist
\item
  存款:6-12 个月生活费
\item
  技能:可转移的能力
\item
  人脉:可以帮你的人
\end{itemize}

\textbf{期权是锦上添花,存款是雪中送炭。}

\begin{center}\rule{0.5\linewidth}{0.5pt}\end{center}

\subsection{铁律九:时间最贵}\label{ux94c1ux5f8bux4e5dux65f6ux95f4ux6700ux8d35}

\begin{quote}
\textbf{钱是工具,时间才是目的。}
\end{quote}

\begin{itemize}
\tightlist
\item
  用钱买时间,不要用时间换钱
\item
  财务自由 = 选择的自由
\item
  复利需要时间,给它时间
\end{itemize}

\textbf{Sarah 48 岁退休去画画。这才是富有。}

\begin{center}\rule{0.5\linewidth}{0.5pt}\end{center}

\subsection{铁律十:听妈妈的话}\label{ux94c1ux5f8bux5341ux542cux5988ux5988ux7684ux8bdd}

\begin{quote}
\textbf{常识胜过聪明。}
\end{quote}

\begin{itemize}
\tightlist
\item
  简单的规则,长期遵守
\item
  不要试图变得太''聪明''
\item
  最复杂的问题,通常有最简单的答案
\end{itemize}

\textbf{妈妈没读过大学,但她的建议最有用。}

\begin{center}\rule{0.5\linewidth}{0.5pt}\end{center}

\subsection{一页总结}\label{ux4e00ux9875ux603bux7ed3}

{\def\LTcaptype{none} % do not increment counter
\begin{longtable}[]{@{}lll@{}}
\toprule\noalign{}
\# & 铁律 & 一句话 \\
\midrule\noalign{}
\endhead
\bottomrule\noalign{}
\endlastfoot
1 & 分散投资 & 不要全押一个篮子 \\
2 & 安全第一 & 先有安全垫 \\
3 & 能力圈 & 听不懂就不投 \\
4 & 逆向思维 & 恐惧时贪婪 \\
5 & 简单为王 & 买指数基金 \\
6 & 自动化 & 定投不择时 \\
7 & 持有是关键 & 不要乱动 \\
8 & Plan B & 存款技能人脉 \\
9 & 时间最贵 & 用钱买时间 \\
10 & 听妈妈的话 & 常识胜过聪明 \\
\end{longtable}
}

\begin{center}\rule{0.5\linewidth}{0.5pt}\end{center}

\subsection{打印这页}\label{ux6253ux5370ux8fd9ux9875-1}

贴在你能看到的地方。

每当你想做投资决策时,看一眼。

\textbf{简单的规则,长期遵守,就能赢。}

\begin{center}\rule{0.5\linewidth}{0.5pt}\end{center}

\textbf{返回}:\href{00-index.md}{目录}

\newpage

\section{附录
C:妈妈语录全集}\label{ux9644ux5f55-cux5988ux5988ux8bedux5f55ux5168ux96c6}

\begin{quote}
妈妈在 Baton Rouge 开餐馆。她没读过大学。但她的话比任何投资书都有用。
\end{quote}

\begin{center}\rule{0.5\linewidth}{0.5pt}\end{center}

\subsection{关于投资}\label{ux5173ux4e8eux6295ux8d44-1}

\begin{quote}
``鸡蛋不要放一个篮子里。''
\end{quote}

分散投资的最简单版本。

\begin{quote}
``泡沫就像可乐,喝的时候很爽,打嗝的时候很尴尬。''
\end{quote}

每一代人都有自己的泡沫。享受时小心,泡沫总会破。

\begin{quote}
``如果一个人说的话你听不懂,要么他是天才,要么他是骗子。天才很少。''
\end{quote}

安然、Theranos、FTX 的共同点:他们的商业模式没人能解释清楚。

\begin{quote}
``熬过冬天的树,根扎得最深。''
\end{quote}

2000 年、2008 年、2020 年的危机都过去了。活下来的人更强。

\begin{quote}
``当所有人都说一件事不可能发生的时候,它通常正在发生。''
\end{quote}

``房价不会跌''、``雷曼不会倒''------这些话说完没多久就应验了。

\begin{quote}
``永远不要相信任何包含'永远'这个词的预测。''
\end{quote}

没有什么是永远的。

\begin{center}\rule{0.5\linewidth}{0.5pt}\end{center}

\subsection{关于人生}\label{ux5173ux4e8eux4ebaux751f-1}

\begin{quote}
``慢慢来,才能快。''
\end{quote}

妈妈在我去硅谷时说的。复利需要时间。职业需要积累。没有捷径。

\begin{quote}
``不是每一场仗都值得打。有时候最聪明的做法是走开。''
\end{quote}

离开不适合你的地方,不是失败,是智慧。

\begin{quote}
``有些地方不适合你。这不意味着你错了。是那个地方错了。''
\end{quote}

DataSphere 的经历教会我这个道理。

\begin{quote}
``你种的树,不一定是你乘凉的树。但至少你种过。''
\end{quote}

关于 cross-training 那些年轻人。给予不一定有回报,但给予本身有意义。

\begin{quote}
``慢慢来没问题,但也不能慢到老死。''
\end{quote}

关于我和 Sarah 的关系。在重要的事情上,不要拖太久。

\begin{center}\rule{0.5\linewidth}{0.5pt}\end{center}

\subsection{关于常识}\label{ux5173ux4e8eux5e38ux8bc6-1}

\begin{quote}
``做生意就像熬 Gumbo,火候要对,材料要好,最重要的是------要有耐心。''
\end{quote}

耐心是一切的基础。

\begin{quote}
``机会来的时候,通常长得不像机会。''
\end{quote}

2004 年 Google 的 offer,2004 年 Kevin 去的
Facebook。当时没人觉得是机会。

\begin{quote}
``当你因为一只狗的图片买股票时,你买的不是投资,是彩票。''
\end{quote}

关于 Dogecoin。表情包不是投资理由。

\begin{quote}
``当一个人说他赚钱是为了帮助别人时,先看看他怎么花自己的钱。''
\end{quote}

关于 SBF 和''有效利他主义''。行动比口号更真实。

\begin{quote}
``自信和能力是两回事。骗子通常自信满满,因为他们知道自己在骗你。''
\end{quote}

关于 Elizabeth Holmes 和 Theranos。自信不等于能力。

\begin{quote}
``银行就像医生------你觉得他们很可靠,直到你真正需要他们的时候。''
\end{quote}

2023 年硅谷银行倒闭。没有什么是绝对安全的。

\begin{quote}
``工具不会取代人,但会用工具的人会取代不会用工具的人。''
\end{quote}

关于 AI。学会使用新工具,不要害怕它们。

\begin{center}\rule{0.5\linewidth}{0.5pt}\end{center}

\subsection{关于教育}\label{ux5173ux4e8eux6559ux80b2}

\begin{quote}
``给孩子鱼,不如教孩子钓鱼。给孩子钱,不如教孩子理钱。''
\end{quote}

财商教育比留下遗产更重要。

\begin{quote}
``学校教知识,家庭教做人。好学校是加分,不是必须。''
\end{quote}

关于公立 vs 私立的争论。

\begin{quote}
``花钱能买到好教育,但最好的老师是免费的------那就是你自己。''
\end{quote}

父母的陪伴比学费更重要。

\begin{quote}
``儿子,我没有给你藤校,但我给了你正确的价值观。这比任何学位都值钱。''
\end{quote}

妈妈对教育的总结。

\begin{center}\rule{0.5\linewidth}{0.5pt}\end{center}

\subsection{最后一句}\label{ux6700ux540eux4e00ux53e5}

\begin{quote}
``儿子,你确定你在投资,不是在赌博?''
\end{quote}

每次做投资决策前,问自己这个问题。

\begin{center}\rule{0.5\linewidth}{0.5pt}\end{center}

\textbf{返回}:\href{00-index.md}{目录}

\newpage

\section{附录 D:Sarah
语录全集}\label{ux9644ux5f55-dsarah-ux8bedux5f55ux5168ux96c6}

\begin{quote}
Sarah 在 Google 工作了 16 年。48
岁退休去画画。她的投资建议比任何财经专家都靠谱。
\end{quote}

\begin{center}\rule{0.5\linewidth}{0.5pt}\end{center}

\subsection{关于投资}\label{ux5173ux4e8eux6295ux8d44-2}

\begin{quote}
``永远不要用你的工资去买你公司的股票。你已经把人力资本押在那里了,不要再押金融资本。''
\end{quote}

Mike 听了两次还没学会。你不要犯同样的错误。

\begin{quote}
``看,这就是分散投资的意义。你的人力资本归零了,但你的金融资本还在。''
\end{quote}

2011 年 NeuralMind 倒闭时说的。期权归零,但指数基金还在。

\begin{quote}
``世界不会完蛋。恐惧是最好的买入信号。''
\end{quote}

2008 年金融危机时说的。她让我买指数基金,那笔投资涨了 4 倍。

\begin{quote}
``别人恐惧时贪婪。但首先你要有贪婪的资本------现金。''
\end{quote}

逆向投资的前提是有钱。没有现金储备,机会来了也抓不住。

\begin{quote}
``当你不知道该买什么的时候,就买整个市场。''
\end{quote}

关于指数基金。不会选股?买 VOO 就行。

\begin{quote}
``投资最难的不是买入,是持有。''
\end{quote}

2009 年市场跌到谷底时说的。很多人在最低点卖出,错过了之后的反弹。

\begin{center}\rule{0.5\linewidth}{0.5pt}\end{center}

\subsection{关于卖出}\label{ux5173ux4e8eux5356ux51fa}

\begin{quote}
``后悔是投资者最大的敌人。做了决定就不要回头看。''
\end{quote}

2004 年我没去 Google。她说不要后悔,向前看。

\begin{quote}
``你不可能抓住每一个机会。重要的是抓住属于你的那个。''
\end{quote}

错过 Facebook、错过 Google,不重要。重要的是抓住了 AI。

\begin{quote}
``你总是在高点买,在低点卖。这叫追涨杀跌。''
\end{quote}

对 Mike 说的。他从安然到雷曼,每次都在高点进场。

\begin{center}\rule{0.5\linewidth}{0.5pt}\end{center}

\subsection{关于职场}\label{ux5173ux4e8eux804cux573a}

\begin{quote}
``在职场上,永远要有 Plan B。不要把你的命运完全交给别人。''
\end{quote}

DataSphere 之后说的。存款、技能、人脉是真正的安全感。

\begin{quote}
``期权只是锦上添花,真正的安全感来自你银行账户里的数字。''
\end{quote}

不要把希望寄托在期权上。大部分期权最后都是废纸。

\begin{quote}
``最好的报复是过得好。''
\end{quote}

关于离开不好的公司。不要浪费精力恨人,把精力放在自己身上。

\begin{center}\rule{0.5\linewidth}{0.5pt}\end{center}

\subsection{关于财务自由}\label{ux5173ux4e8eux8d22ux52a1ux81eaux7531}

\begin{quote}
``真正的财富是时间,不是钱。钱可以买时间,但时间不能买回来。''
\end{quote}

2025 年咖啡馆里说的。她 48 岁退休去画画,这才是真正的富有。

\begin{quote}
``硅谷最大的秘密是:真正有钱的人,不是创始人,是早期员工加上长期持有股票的人。''
\end{quote}

创始人要融资会被稀释。早期员工拿期权,上市后持有。复利。时间。耐心。

\begin{quote}
``Google 的股票够我活三辈子。我想去画画了。''
\end{quote}

财务自由的定义:可以做任何想做的事。

\begin{quote}
``找一家好公司,待很久,不乱动。''
\end{quote}

她的投资策略总结。无聊,但有效。

\begin{center}\rule{0.5\linewidth}{0.5pt}\end{center}

\subsection{关于判断}\label{ux5173ux4e8eux5224ux65ad}

\begin{quote}
``如果一个公司的财报需要天才才能看懂,要么他们在做诺贝尔奖级别的创新,要么他们在做牢狱级别的造假。''
\end{quote}

她爸是 CPA。这是关于安然的评价。

\begin{quote}
``当所有人都在用复杂的方法解释一个简单的问题时,那个问题通常不是他们说的那个问题。''
\end{quote}

关于次贷危机。复杂是用来掩盖真相的。

\begin{quote}
``这家医疗科技公司,没有发过一篇经过同行评审的论文。''
\end{quote}

关于 Theranos。技术公司没有技术论文,很可疑。

\begin{quote}
``任何用表情包定价的东西,都不是投资,是赌博。''
\end{quote}

关于 Dogecoin 和 meme 股票。

\begin{quote}
``因为他追的是热度,不是价值。永远在追,永远在接盘。''
\end{quote}

关于 Tommy 为什么总是亏钱。

\begin{quote}
``骗子的共同特点是:他们说的话你听不懂,但他们自己很自信。''
\end{quote}

如何识别骗子的简单方法。

\begin{center}\rule{0.5\linewidth}{0.5pt}\end{center}

\subsection{关于分散}\label{ux5173ux4e8eux5206ux6563}

\begin{quote}
``这就是为什么你要分散存款。永远不要把所有钱放在一家银行。''
\end{quote}

2023 年硅谷银行倒闭后说的。

\begin{quote}
``大佬有钱可以赌。你没有。''
\end{quote}

关于那些投资 Theranos 的名人。他们亏得起,你亏不起。

\begin{center}\rule{0.5\linewidth}{0.5pt}\end{center}

\subsection{最后一句}\label{ux6700ux540eux4e00ux53e5-1}

\begin{quote}
``泡沫和革命的区别是------革命最终会创造真正的价值。但在那之前,它们看起来一模一样。''
\end{quote}

关于 2000 年互联网泡沫和 2024 年 AI
热潮。区分的方法:看公司是不是真的在赚钱。

\begin{center}\rule{0.5\linewidth}{0.5pt}\end{center}

\textbf{返回}:\href{00-index.md}{目录}

\newpage

\end{document}
